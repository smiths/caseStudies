\documentclass[12pt]{article}
\usepackage{fontspec}
\usepackage{fullpage}
\usepackage{hyperref}
\hypersetup{bookmarks=true,colorlinks=true,linkcolor=red,citecolor=blue,filecolor=magenta,urlcolor=cyan}
\usepackage{amsmath}
\usepackage{amssymb}
\usepackage{mathtools}
\usepackage{unicode-math}
\usepackage{tabularray}
\usepackage{tabularx}
\usepackage{booktabs}
\usepackage{caption}
\usepackage{graphics}
\usepackage{svg}
\usepackage{enumitem}
\usepackage{filecontents}
\usepackage[backend=bibtex]{biblatex}
\usepackage{url}

\usepackage{color}

\newif\ifcomments\commentstrue

\ifcomments
\newcommand{\authornote}[3]{\textcolor{#1}{[#3 ---#2]}}
\newcommand{\todo}[1]{\textcolor{red}{[TODO: #1]}}
\else
\newcommand{\authornote}[3]{}
\newcommand{\todo}[1]{}
\fi

\newcommand{\wss}[1]{\authornote{blue}{SS}{#1}} 

\setmathfont{Latin Modern Math}
\newcommand{\gt}{\ensuremath >}
\newcommand{\lt}{\ensuremath <}
\newlist{symbDescription}{description}{1}
\setlist[symbDescription]{noitemsep, topsep=0pt, parsep=0pt, partopsep=0pt}
\bibliography{bibfile}
\title{Software Requirements Specification for Projectile}
\author{Samuel J. Crawford, Brooks MacLachlan, and W. Spencer Smith}
\begin{document}
\maketitle
\tableofcontents
\newpage
\section{Reference Material}
\label{Sec:RefMat}
This section records information for easy reference.

\subsection{Table of Units}
\label{Sec:ToU}
The unit system used throughout is SI (Système International d'Unités). In addition to the basic units, several derived units are also used. For each unit, the \hyperref[Table:ToU]{Table of Units} lists the symbol, a description, and the SI name.

\begin{longtblr}
[caption={Table of Units}]
{colspec={l l l}, rowhead=1, hline{1,Z}=\heavyrulewidth, hline{2}=\lightrulewidth}
\textbf{Symbol} & \textbf{Description} & \textbf{SI Name}
\\
${\text{m}}$ & length & metre
\\
${\text{rad}}$ & angle & radian
\\
${\text{s}}$ & time & second
\label{Table:ToU}
\end{longtblr}
\subsection{Table of Symbols}
\label{Sec:ToS}
The symbols used in this document are summarized in the \hyperref[Table:ToS]{Table of Symbols} along with their units. Throughout the document, symbols in bold will represent vectors, and scalars otherwise. The symbols are listed in alphabetical order. For vector quantities, the units shown are for each component of the vector.

\begin{longtblr}
[caption={Table of Symbols}]
{colspec={l X[l] l}, rowhead=1, hline{1,Z}=\heavyrulewidth, hline{2}=\lightrulewidth}
\textbf{Symbol} & \textbf{Description} & \textbf{Units}
\\
$a$ & Scalar acceleration & $\frac{\text{m}}{\text{s}^{2}}$
\\
${a^{c}}$ & Constant acceleration & $\frac{\text{m}}{\text{s}^{2}}$
\\
${a_{\text{x}}}$ & $x$-component of acceleration & $\frac{\text{m}}{\text{s}^{2}}$
\\
${{a_{\text{x}}}^{\text{c}}}$ & $x$-component of constant acceleration & $\frac{\text{m}}{\text{s}^{2}}$
\\
${a_{\text{y}}}$ & $y$-component of acceleration & $\frac{\text{m}}{\text{s}^{2}}$
\\
${{a_{\text{y}}}^{\text{c}}}$ & $y$-component of constant acceleration & $\frac{\text{m}}{\text{s}^{2}}$
\\
$\symbf{a}\text{(}t\text{)}$ & Acceleration & $\frac{\text{m}}{\text{s}^{2}}$
\\
${\symbf{a}^{\text{c}}}$ & Constant acceleration vector & $\frac{\text{m}}{\text{s}^{2}}$
\\
${d_{\text{offset}}}$ & Distance between the target position and the landing position & ${\text{m}}$
\\
$g$ & Magnitude of gravitational acceleration & $\frac{\text{m}}{\text{s}^{2}}$
\\
$p$ & Scalar position & ${\text{m}}$
\\
$p\text{(}t\text{)}$ & 1D position & ${\text{m}}$
\\
${p^{\text{i}}}$ & Initial position & ${\text{m}}$
\\
${p_{\text{land}}}$ & Landing position & ${\text{m}}$
\\
${p_{\text{target}}}$ & Target position & ${\text{m}}$
\\
${p_{\text{x}}}$ & $x$-component of position & ${\text{m}}$
\\
${{p_{\text{x}}}^{\text{i}}}$ & $x$-component of initial position & ${\text{m}}$
\\
${p_{\text{y}}}$ & $y$-component of position & ${\text{m}}$
\\
${{p_{\text{y}}}^{\text{i}}}$ & $y$-component of initial position & ${\text{m}}$
\\
$\symbf{p}\text{(}t\text{)}$ & Position & ${\text{m}}$
\\
$s$ & Output message as a string & --
\\
$t$ & Time & ${\text{s}}$
\\
${t_{\text{flight}}}$ & Flight duration & ${\text{s}}$
\\
$v$ & Speed & $\frac{\text{m}}{\text{s}}$
\\
$v\text{(}t\text{)}$ & 1D speed & $\frac{\text{m}}{\text{s}}$
\\
${v^{\text{i}}}$ & Initial speed & $\frac{\text{m}}{\text{s}}$
\\
${v_{\text{launch}}}$ & Launch speed & $\frac{\text{m}}{\text{s}}$
\\
${v_{\text{x}}}$ & $x$-component of velocity & $\frac{\text{m}}{\text{s}}$
\\
${{v_{\text{x}}}^{\text{i}}}$ & $x$-component of initial velocity & $\frac{\text{m}}{\text{s}}$
\\
${v_{\text{y}}}$ & $y$-component of velocity & $\frac{\text{m}}{\text{s}}$
\\
${{v_{\text{y}}}^{\text{i}}}$ & $y$-component of initial velocity & $\frac{\text{m}}{\text{s}}$
\\
$\symbf{v}\text{(}t\text{)}$ & Velocity & $\frac{\text{m}}{\text{s}}$
\\
${\symbf{v}^{\text{i}}}$ & Initial velocity & $\frac{\text{m}}{\text{s}}$
\\
$ε$ & Hit tolerance & --
\\
$θ$ & Launch angle & ${\text{rad}}$
\\
$π$ & Ratio of circumference to diameter for any circle & --
\label{Table:ToS}
\end{longtblr}
\subsection{Abbreviations and Acronyms}
\label{Sec:TAbbAcc}
\begin{longtblr}
[caption={Abbreviations and Acronyms}]
{colspec={l l}, rowhead=1, hline{1,Z}=\heavyrulewidth, hline{2}=\lightrulewidth}
\textbf{Abbreviation} & \textbf{Full Form}
\\
1D & One-Dimensional
\\
2D & Two-Dimensional
\\
A & Assumption
\\
DD & Data Definition
\\
GD & General Definition 
\\
GS & Goal Statement
\\
IM & Instance Model
\\
PS & Physical System Description
\\
R & Requirement
\\
RefBy & Referenced by
\\
Refname & Reference Name
\\
SRS & Software Requirements Specification
\\
TM & Theoretical Model
\\
Uncert. & Typical Uncertainty
\label{Table:TAbbAcc}
\end{longtblr}
\section{Introduction}
\label{Sec:Intro}
Projectile motion is a common problem in physics. Therefore, it is useful to have a program to solve and model these types of problems. Common examples of projectile motion include ballistics problems (missiles, bullets, etc.) and the flight of balls in various sports (baseball, golf, football, etc.). The program documented here is called Projectile.

The following section provides an overview of the Software Requirements Specification (SRS) for Projectile. This section explains the purpose of this document, the scope of the requirements, the characteristics of the intended reader, and the organization of the document.

\subsection{Purpose of Document}
\label{Sec:DocPurpose}
The primary purpose of this document is to record the requirements of Projectile. Goals, assumptions, theoretical models, definitions, and other model derivation information are specified, allowing the reader to fully understand and verify the purpose and scientific basis of Projectile. With the exception of \hyperref[Sec:SysConstraints]{system constraints}, this SRS will remain abstract, describing what problem is being solved, but not how to solve it.

This document will be used as a starting point for subsequent development phases, including writing the design specification and the software verification and validation plan. The design document will show how the requirements are to be realized, including decisions on the numerical algorithms and programming environment. The verification and validation plan will show the steps that will be used to increase confidence in the software documentation and the implementation. Although the SRS fits in a series of documents that follow the so-called waterfall model, the actual development process is not constrained in any way. Even when the waterfall model is not followed, as Parnas and Clements point out \cite{parnasClements1986}, the most logical way to present the documentation is still to ``fake'' a rational design process.

\subsection{Scope of Requirements}
\label{Sec:ReqsScope}
The scope of the requirements includes the analysis of a two-dimensional (2D) projectile motion problem with constant acceleration.

\wss{We are only interested in the position of the projectile, not
its orientation \hyperref[noOrientation]{A:noOrientation}.} %this is how we leave theories out, like theories of rotation

\wss{We assume that forces are not relevant for the model so that we only need kinematic equations \hyperref[kinematicOnly]{A:kinematicOnly}.}

\subsection{Characteristics of Intended Reader}
\label{Sec:ReaderChars}
Reviewers of this documentation should have an understanding of undergraduate level 1 physics and undergraduate level 1 calculus. The users of Projectile can have a lower level of expertise, as explained in \hyperref[Sec:UserChars]{Sec:User Characteristics}.

\subsection{Organization of Document}
\label{Sec:DocOrg}
The organization of this document follows the template for an SRS for scientific computing software proposed by \cite{koothoor2013}, \cite{smithLai2005}, \cite{smithEtAl2007}, and \cite{smithKoothoor2016}. The presentation follows the standard pattern of presenting goals, theories, definitions, and assumptions. For readers that would like a more bottom up approach, they can start reading the \hyperref[Sec:IMs]{instance models} and trace back to find any additional information they require.

The \hyperref[Sec:GoalStmt]{goal statements} are refined to the theoretical models and the \hyperref[Sec:TMs]{theoretical models} to the \hyperref[Sec:IMs]{instance models}.

\section{General System Description}
\label{Sec:GenSysDesc}
This section provides general information about the system. It identifies the interfaces between the system and its environment, describes the user characteristics, and lists the system constraints.

\subsection{System Context}
\label{Sec:SysContext}
\hyperref[Figure:sysCtxDiag]{Fig:sysCtxDiag} shows the system context. A circle represents an entity external to the software, the user in this case. A rectangle represents the software system itself (Projectile). Arrows are used to show the data flow between the system and its environment.

\begin{figure}
\begin{center}
\includegraphics[width=\textwidth]{SystemContextFigure.png}
\caption{System Context}
\label{Figure:sysCtxDiag}
\end{center}
\end{figure}
The interaction between the product and the user is through an application programming interface. The responsibilities of the user and the system are as follows:

\begin{itemize}
\item{User Responsibilities}
\begin{itemize}
\item{Provide initial conditions of the physical state of the motion and the input data related to the Projectile, ensuring no errors in the data entry.}
\item{Ensure that consistent units are used for input variables.}
\item{Ensure required \hyperref[Sec:Assumps]{software assumptions} are appropriate for any particular problem input to the software.}
\end{itemize}
\item{Projectile Responsibilities}
\begin{itemize}
\item{Detect data type mismatch, such as a string of characters input instead of a floating point number.}
\item{Determine if the inputs satisfy the required physical and software constraints.}
\item{Calculate the required outputs.}
\end{itemize}
\end{itemize}

\subsection{User Characteristics}
\label{Sec:UserChars}
The end user of Projectile should have an understanding of high school physics and high school calculus.

\subsection{System Constraints}
\label{Sec:SysConstraints}
There are no system constraints.

\section{Specific System Description}
\label{Sec:SpecSystDesc}
This section first presents the problem description, which gives a high-level view of the problem to be solved. This is followed by the solution characteristics specification, which presents the assumptions, theories, and definitions that are used.

\subsection{Problem Description}
\label{Sec:ProbDesc}
A system is needed to predict whether a launched projectile hits its target.

\subsubsection{Terminology and Definitions}
\label{Sec:TermDefs}
This subsection provides a list of terms that are used in the subsequent sections and their meaning, with the purpose of reducing ambiguity and making it easier to correctly understand the requirements.

\begin{itemize}
\item{Launcher: Where the projectile is launched from and the device that does the launching.}
\item{Projectile: The object to be launched at the target.}
\item{Target: Where the projectile should be launched to.}
\item{Gravity: The force that attracts one physical body with mass to another.}
\item{Cartesian coordinate system: A coordinate system that specifies each point uniquely in a plane by a set of numerical coordinates, which are the signed distances to the point from two fixed perpendicular oriented lines, measured in the same unit of length (from \cite{cartesianWiki}).}
\item{Rectilinear: Occurring \wss{fixed typo in ``Ocurring''} in one dimension.}
\end{itemize}
\subsubsection{Physical System Description}
\label{Sec:PhysSyst}
The physical system of Projectile, as shown in \hyperref[Figure:Launch]{Fig:Launch}, includes the following elements:

\begin{itemize}
\item[PS1:]{The launcher.}
\item[PS2:]{The projectile (with initial velocity ${\symbf{v}^{\text{i}}}$ and launch angle $θ$).}
\item[PS3:]{The target.}
\end{itemize}
\begin{figure}
\begin{center}
\includegraphics[width=0.7\textwidth]{Launch.jpg}
\caption{The physical system}
\label{Figure:Launch}
\end{center}
\end{figure}
\subsubsection{Goal Statements}
\label{Sec:GoalStmt}
Given the initial velocity vector of the projectile and the geometric layout of the launcher and target, the goal statement is:

\begin{itemize}
\item[targetHit:\phantomsection\label{targetHit}]{Determine if the projectile hits the target.}
\end{itemize}
\subsection{Solution Characteristics Specification}
\label{Sec:SolCharSpec}
The instance models that govern Projectile are presented in the \hyperref[Sec:IMs]{Instance Model Section}. The information to understand the meaning of the instance models and their derivation is also presented, so that the instance models can be verified.

\subsubsection{\wss{Types}}

\wss{$\text{time} = \mathbb{R}$}
\subsubsection{\wss{Scope Assumptions}}
\label{Sec:Assumps}
This section simplifies the original problem and helps in developing the theoretical models by filling in the missing information for the physical system. The assumptions refine the scope by providing more detail.

\begin{itemize}
\item[noOrientation:\phantomsection\label{noOrientation}]{\wss{The orientation of the projectile is ignored. We only care about its translation not its rotation. (RefBy: \hyperref[TM:acceleration]{TM:acceleration} and \hyperref[TM:velocity]{TM:velocity}.)}}
\item[kinematicOnly:\phantomsection\label{kinematicOnly}]{\wss{The motion of the
projectile is modelled with only kinematic equations. Forces are not
considered.}} 
\end{itemize}

\subsubsection{\wss{Background Theory Assumptions}}

\begin{itemize}
\item[cartSyst:\phantomsection\label{cartSyst}]{A Cartesian coordinate system is used (from \hyperref[neglectCurv]{A:neglectCurv}). (RefBy: \hyperref[GD:velVec]{GD:velVec} and \hyperref[GD:posVec]{GD:posVec}.)}
\end{itemize}

\subsubsection{\wss{Helper Theory Assumptions (GD:rectVel, GD:rectPos)}}

\begin{itemize}
\item[oneD:\phantomsection\label{oneD}]{The motion of the particle is one dimensional. (RefBy: \hyperref[GD:velVec]{GD:velVec} and \hyperref[GD:posVec]{GD:posVec}.)}
\item[constAccel:\phantomsection\label{constAccel}]{The acceleration is constant (from \hyperref[accelXZero]{A:accelXZero}, \hyperref[accelYGravity]{A:accelYGravity}, \hyperref[neglectDrag]{A:neglectDrag}, and \hyperref[freeFlight]{A:freeFlight}). (RefBy: \hyperref[GD:velVec]{GD:velVec} and \hyperref[GD:posVec]{GD:posVec}.)}
\end{itemize}

\subsubsection{\wss{Theory Assumptions (GD:velVec, GD:posVec)}}

\begin{itemize}
\item[twoD:\phantomsection\label{twoD}]{The variables only depend on
two-dimensions (2D). (RefBy: \hyperref[GD:velVec]{GD:velVec} and
\hyperref[GD:posVec]{GD:posVec}.)} \wss{changed twoDMotion to just twoD, so that
it can be used for things that are 2D other than just the motion.}
\item[constAccelX:\phantomsection\label{constAccelX}]{\wss{The acceleration is constant in the $x$ direction. (RefBy: \hyperref[GD:velVec]{GD:velVec} and \hyperref[GD:posVec]{GD:posVec}.)}}
\item[constAccelY:\phantomsection\label{constAccelY}]{\wss{The acceleration is constant in the $y$ direction. (RefBy: \hyperref[GD:velVec]{GD:velVec} and \hyperref[GD:posVec]{GD:posVec}.)}}

\end{itemize}

\subsubsection{\wss{Theory Assumptions}}

\begin{itemize}
\item[twoD:\phantomsection\label{twoD}]{The projectile motion is two-dimensional (2D). (RefBy: \hyperref[GD:velVec]{GD:velVec} and \hyperref[GD:posVec]{GD:posVec}.)}
\item[yAxisGravity:\phantomsection\label{yAxisGravity}]{The direction of the $y$-axis is directed opposite to gravity. (RefBy: \hyperref[IM:calOfLandingDist]{IM:calOfLandingDist}, \hyperref[IM:calOfLandingTime]{IM:calOfLandingTime}, and \hyperref[accelYGravity]{A:accelYGravity}.)}
\item[launchOrigin:\phantomsection\label{launchOrigin}]{The launcher is coincident with the origin. (RefBy: \hyperref[IM:calOfLandingDist]{IM:calOfLandingDist} and \hyperref[IM:calOfLandingTime]{IM:calOfLandingTime}.)}
\item[targetXAxis:\phantomsection\label{targetXAxis}]{The target lies on the $x$-axis (from \hyperref[neglectCurv]{A:neglectCurv}). (RefBy: \hyperref[IM:calOfLandingTime]{IM:calOfLandingTime}.)}
\item[posXDirection:\phantomsection\label{posXDirection}]{The positive $x$-direction is from the launcher to the target. (RefBy: \hyperref[IM:offsetIM]{IM:offsetIM}, \hyperref[IM:messageIM]{IM:messageIM}, \hyperref[IM:calOfLandingDist]{IM:calOfLandingDist}, and \hyperref[IM:calOfLandingTime]{IM:calOfLandingTime}.)}
\item[accelXZero:\phantomsection\label{accelXZero}]{The acceleration in the $x$-direction is zero. (RefBy: \hyperref[IM:calOfLandingDist]{IM:calOfLandingDist} and \hyperref[constAccel]{A:constAccel}.)}
\item[accelYGravity:\phantomsection\label{accelYGravity}]{The acceleration in the $y$-direction is the acceleration due to gravity (from \hyperref[yAxisGravity]{A:yAxisGravity}). (RefBy: \hyperref[IM:calOfLandingTime]{IM:calOfLandingTime} and \hyperref[constAccel]{A:constAccel}.)}
\item[neglectDrag:\phantomsection\label{neglectDrag}]{Air drag is neglected. (RefBy: \hyperref[constAccel]{A:constAccel}.)}
\item[pointMass:\phantomsection\label{pointMass}]{The size and shape of the projectile are negligible, so that it can be modelled as a point mass. (RefBy: \hyperref[GD:rectVel]{GD:rectVel} and \hyperref[GD:rectPos]{GD:rectPos}.)}
\item[freeFlight:\phantomsection\label{freeFlight}]{The flight is free; there are no collisions during the trajectory of the projectile. (RefBy: \hyperref[constAccel]{A:constAccel}.)}
\item[neglectCurv:\phantomsection\label{neglectCurv}]{The distance is small enough that the curvature of the celestial body can be neglected. (RefBy: \hyperref[targetXAxis]{A:targetXAxis} and \hyperref[cartSyst]{A:cartSyst}.)}
\item[timeStartZero:\phantomsection\label{timeStartZero}]{Time starts at zero. (RefBy: \hyperref[GD:velVec]{GD:velVec}, \hyperref[GD:rectVel]{GD:rectVel}, \hyperref[GD:rectPos]{GD:rectPos}, \hyperref[GD:posVec]{GD:posVec}, and \hyperref[IM:calOfLandingTime]{IM:calOfLandingTime}.)}
\item[gravAccelValue:\phantomsection\label{gravAccelValue}]{The acceleration due to gravity is assumed to have the value provided in the section for \hyperref[Sec:AuxConstants]{Values of Auxiliary Constants}. (RefBy: \hyperref[IM:calOfLandingDist]{IM:calOfLandingDist} and \hyperref[IM:calOfLandingTime]{IM:calOfLandingTime}.)}
\end{itemize}

\subsubsection{\wss{Final Theory Assumptions}}

\begin{itemize}
\item[twoD:\phantomsection\label{twoD}]{The projectile motion is two-dimensional (2D). (RefBy: \hyperref[GD:velVec]{GD:velVec} and \hyperref[GD:posVec]{GD:posVec}.)}
\item[yAxisGravity:\phantomsection\label{yAxisGravity}]{The direction of the $y$-axis is directed opposite to gravity. (RefBy: \hyperref[IM:calOfLandingDist]{IM:calOfLandingDist}, \hyperref[IM:calOfLandingTime]{IM:calOfLandingTime}, and \hyperref[accelYGravity]{A:accelYGravity}.)}
\item[launchOrigin:\phantomsection\label{launchOrigin}]{The launcher is coincident with the origin. (RefBy: \hyperref[IM:calOfLandingDist]{IM:calOfLandingDist} and \hyperref[IM:calOfLandingTime]{IM:calOfLandingTime}.)}
\item[targetXAxis:\phantomsection\label{targetXAxis}]{The target lies on the $x$-axis (from \hyperref[neglectCurv]{A:neglectCurv}). (RefBy: \hyperref[IM:calOfLandingTime]{IM:calOfLandingTime}.)}
\item[posXDirection:\phantomsection\label{posXDirection}]{The positive $x$-direction is from the launcher to the target. (RefBy: \hyperref[IM:offsetIM]{IM:offsetIM}, \hyperref[IM:messageIM]{IM:messageIM}, \hyperref[IM:calOfLandingDist]{IM:calOfLandingDist}, and \hyperref[IM:calOfLandingTime]{IM:calOfLandingTime}.)}
\item[accelXZero:\phantomsection\label{accelXZero}]{The acceleration in the $x$-direction is zero. (RefBy: \hyperref[IM:calOfLandingDist]{IM:calOfLandingDist} and \hyperref[constAccel]{A:constAccel}.)}
\item[accelYGravity:\phantomsection\label{accelYGravity}]{The acceleration in the $y$-direction is the acceleration due to gravity (from \hyperref[yAxisGravity]{A:yAxisGravity}). (RefBy: \hyperref[IM:calOfLandingTime]{IM:calOfLandingTime} and \hyperref[constAccel]{A:constAccel}.)}
\item[neglectDrag:\phantomsection\label{neglectDrag}]{Air drag is neglected. (RefBy: \hyperref[constAccel]{A:constAccel}.)}
\item[pointMass:\phantomsection\label{pointMass}]{The size and shape of the projectile are negligible, so that it can be modelled as a point mass. (RefBy: \hyperref[GD:rectVel]{GD:rectVel} and \hyperref[GD:rectPos]{GD:rectPos}.)}
\item[freeFlight:\phantomsection\label{freeFlight}]{The flight is free; there are no collisions during the trajectory of the projectile. (RefBy: \hyperref[constAccel]{A:constAccel}.)}
\item[neglectCurv:\phantomsection\label{neglectCurv}]{The distance is small enough that the curvature of the celestial body can be neglected. (RefBy: \hyperref[targetXAxis]{A:targetXAxis} and \hyperref[cartSyst]{A:cartSyst}.)}
\item[timeStartZero:\phantomsection\label{timeStartZero}]{Time starts at zero. (RefBy: \hyperref[GD:velVec]{GD:velVec}, \hyperref[GD:rectVel]{GD:rectVel}, \hyperref[GD:rectPos]{GD:rectPos}, \hyperref[GD:posVec]{GD:posVec}, and \hyperref[IM:calOfLandingTime]{IM:calOfLandingTime}.)}
\item[gravAccelValue:\phantomsection\label{gravAccelValue}]{The acceleration due to gravity is assumed to have the value provided in the section for \hyperref[Sec:AuxConstants]{Values of Auxiliary Constants}. (RefBy: \hyperref[IM:calOfLandingDist]{IM:calOfLandingDist} and \hyperref[IM:calOfLandingTime]{IM:calOfLandingTime}.)}
\end{itemize}

\subsubsection{\wss{Context Theories}}

\wss{Some theories do not have to be explicitly invoked. They are part of the context for the other theories, without having to be explicitly stated or defined. The context theories for this problem are as follows:
}

\medskip
\noindent
\begin{minipage}{\textwidth}
\begin{tabular}{>{\raggedright}p{0.13\textwidth}>{\raggedright\arraybackslash}p{0.82\textwidth}}
\toprule \textbf{Refname} & \textbf{\wss{CT:realArith}}
\phantomsection 
\label{CT:realArith}
\\ \midrule
Label & Real Arithmetic
\\ \midrule
Source & \cite{}
\\ \midrule
RefBy & \hyperref[TM:acceleration]{TM:acceleration},
\hyperref[TM:velocity]{TM:velocity},
\hyperref[TM:directionCosines]{TM:directionCosines},
\hyperref[GD:rectVel]{GD:rectVel}, \hyperref[GD:rectVel]{GD:rectVel}, \hyperref[DD:speedIX]{DD:speedIX}, \hyperref[DD:speedIY]{DD:speedIY},everything (TODO: fill in)
\\ \bottomrule
\end{tabular}
\end{minipage}
~\\

\medskip
\noindent
\begin{minipage}{\textwidth}
\begin{tabular}{>{\raggedright}p{0.13\textwidth}>{\raggedright\arraybackslash}p{0.82\textwidth}}
\toprule \textbf{Refname} & \textbf{\wss{CT:trigonometry}}
\phantomsection 
\label{CT:trigonometry}
\\ \midrule
Label & Trigonometry
\\ \midrule
Source & \cite{}
\\ \midrule
RefBy & \hyperref[TM:directionCosines]{TM:directionCosines}, \hyperref[DD:speedIX]{DD:speedIX}, \hyperref[DD:speedIY]{DD:speedIY},everything (TODO: fill in)
\\ \bottomrule
\end{tabular}
\end{minipage}
~\\

\medskip
\noindent
\begin{minipage}{\textwidth}
\begin{tabular}{>{\raggedright}p{0.13\textwidth}>{\raggedright\arraybackslash}p{0.82\textwidth}}
\toprule \textbf{Refname} & \textbf{\wss{CT:vectors}}
\phantomsection 
\label{CT:vectors}
\\ \midrule
Label & Vectors
\\ \midrule
Source & \cite{}
\\ \midrule
RefBy & \hyperref[TM:acceleration]{TM:acceleration},
\hyperref[TM:velocity]{TM:velocity},
\hyperref[TM:directionCosines]{TM:directionCosines},
\hyperref[GD:rectVel]{GD:rectVel}, \hyperref[GD:rectVel]{GD:rectVel}, \hyperref[DD:speedIX]{DD:speedIX}, \hyperref[DD:speedIY]{DD:speedIY},TODO: fill in
\\ \bottomrule
\end{tabular}
\end{minipage}
~\\

\medskip
\noindent
\begin{minipage}{\textwidth}
\begin{tabular}{>{\raggedright}p{0.13\textwidth}>{\raggedright\arraybackslash}p{0.82\textwidth}}
\toprule \textbf{Refname} & \textbf{\wss{CT:CartCoordSyst}}
\phantomsection 
\label{CT:CartCoordSyst}
\\ \midrule
Label & Cartesian Coordinate System
\\ \midrule
Source & \cite{}
\\ \midrule
RefBy & \hyperref[TM:acceleration]{TM:acceleration},
\hyperref[TM:velocity]{TM:velocity},
\hyperref[TM:directionCosines]{TM:directionCosines},
\hyperref[GD:rectVel]{GD:rectVel}, \hyperref[GD:rectVel]{GD:rectVel}, \hyperref[DD:speedIX]{DD:speedIX}, \hyperref[DD:speedIY]{DD:speedIY},TODO: fill in
\\ \bottomrule
\end{tabular}
\end{minipage}
~\\

\medskip
\noindent
\begin{minipage}{\textwidth}
\begin{tabular}{>{\raggedright}p{0.13\textwidth}>{\raggedright\arraybackslash}p{0.82\textwidth}}
\toprule \textbf{Refname} & \textbf{\wss{CT:Differentiation}}
\phantomsection 
\label{CT:Differentiation}
\\ \midrule
Label & Differentiation
\\ \midrule
Source & \cite{}
\\ \midrule
RefBy & \hyperref[TM:acceleration]{TM:acceleration},
\hyperref[TM:velocity]{TM:velocity}, \hyperref[GD:rectVel]{GD:rectVel}, \hyperref[GD:rectVel]{GD:rectVel}, TODO:
fill in 
\\ \bottomrule
\end{tabular}
\end{minipage}
~\\

\medskip
\noindent
\begin{minipage}{\textwidth}
\begin{tabular}{>{\raggedright}p{0.13\textwidth}>{\raggedright\arraybackslash}p{0.82\textwidth}}
\toprule \textbf{Refname} & \textbf{\wss{CT:Integration}}
\phantomsection 
\label{CT:Integration}
\\ \midrule
Label & Integration
\\ \midrule
Source & \cite{}
\\ \midrule
RefBy & \hyperref[GD:rectVel]{GD:rectVel}, \hyperref[GD:rectVel]{GD:rectVel}, TODO: fill in
\\ \bottomrule
\end{tabular}
\end{minipage}
~\\

\subsubsection{\wss{Background Theories (BT)}}
\label{Sec:TMs}
This section focuses on the general equations and laws that Projectile is based
on.  \wss{Maybe relabel all of the background theories with the prefix BT,
instead of TM?}

\medskip
\noindent
\begin{minipage}{\textwidth}
\begin{tabular}{>{\raggedright}p{0.13\textwidth}>{\raggedright\arraybackslash}p{0.82\textwidth}}
\toprule \textbf{Refname} & \textbf{TM:acceleration}
\phantomsection 
\label{TM:acceleration}
\\ \midrule
Label & Acceleration
\\ \midrule
Equation & \begin{displaymath}
           \symbf{a}\text{(}t\text{)}=\frac{\,d\symbf{v}\text{(}t\text{)}}{\,dt}
           \end{displaymath}
\\ \midrule
Description & \begin{symbDescription}
              \item{\wss{$t: \text{time}$} is the time (${\text{s}}$)}
              \item{\wss{$\symbf{a}: \text{time} \rightarrow \mathbb{R}^3$} is the acceleration ($\frac{\text{m}}{\text{s}^{2}}$)}
              \item{\wss{$\symbf{v}: \text{time} \rightarrow \mathbb{R}^3$} is the velocity ($\frac{\text{m}}{\text{s}}$)}
              \end{symbDescription}
\\ \midrule
Notes & The velocity and acceleration of the body are expressed using a
Cartesian coordinate system (\hyperref[CT:CartCoordSyst]{CT:CartCoordSyst},
\hyperref[cartSyst]{A:cartSyst}).  That is, the coordinate system is rectangular
(orthonormal). The relationship between acceleration and velocity uses the
concepts of real arithmetic, vectors and differentiation
(\hyperref[CT:realArith]{CT:realArith}, \hyperref[CT:vectors]{CT:vectors},
\hyperref[CT:Differentiation]{CT:Differentiation}). \wss{Should it be made
explicit that this theory is 3D by having an A:threeD assumption?}
\\ \midrule
\wss{Constraints} & \wss{None}


\\ \midrule
Source & \cite{accelerationWiki}
         
\\ \midrule
RefBy & \hyperref[GD:rectVel]{GD:rectVel}
        
\\ \bottomrule
\end{tabular}
\end{minipage}
~\\
\noindent \textbf{Theories Used by TM:acceleration}

\begin{itemize}
\item \hyperref[CT:realArith]{CT:realArith}
\item \hyperref[CT:vectors]{CT:vectors}
\item \hyperref[CT:CartCoordSyst]{CT:CartCoodSyst}
\item \hyperref[CT:Differentiation]{CT:Differentiation}
\end{itemize}

\noindent \textbf{Preconditions for TM:acceleration}

\begin{itemize}
\item \hyperref[cartSyst]{A:cartSyst}
\end{itemize}

\medskip
\noindent
\begin{minipage}{\textwidth}
\begin{tabular}{>{\raggedright}p{0.13\textwidth}>{\raggedright\arraybackslash}p{0.82\textwidth}}
\toprule \textbf{Refname} & \textbf{TM:velocity}
\phantomsection 
\label{TM:velocity}
\\ \midrule
Label & Velocity
        
\\ \midrule
Equation & \begin{displaymath}
           \symbf{v}\text{(}t\text{)}=\frac{\,d\symbf{p}\text{(}t\text{)}}{\,dt}
           \end{displaymath}
\\ \midrule
Description & \begin{symbDescription}
              \item{\wss{$t: \text{time}$} is the time (${\text{s}}$)}
              \item{\wss{$\symbf{v}: \text{time} \rightarrow \mathbb{R}^3$} is the velocity ($\frac{\text{m}}{\text{s}}$)}
              \item{\wss{$\symbf{p}: \text{time} \rightarrow \mathbb{R}^3$} is the position (${\text{m}}$)}
              \end{symbDescription}
\\ \midrule
Notes & The position and velocity of the body are expressed using a Cartesian
coordinate system (\hyperref[CT:CartCoordSyst]{CT:CartCoordSyst},
\hyperref[cartSyst]{A:cartSyst}).  That is, the coordinate system is rectangular
(orthonormal).  The relationship between velocity and position uses the concepts
of real arithmetic, vectors and differentiation
(\hyperref[CT:realArith]{CT:realArith}, \hyperref[CT:vectors]{CT:vectors},
\hyperref[CT:Differentiation]{CT:Differentiation}).

\\ \midrule
\wss{Constraints} & \wss{None}

\\ \midrule
Source & \cite{velocityWiki}
         
\\ \midrule
RefBy & \hyperref[GD:rectPos]{GD:rectPos}
        
\\ \bottomrule
\end{tabular}
\end{minipage}
~\\
\noindent \textbf{Theories Used by TM:velocity}

\begin{itemize}
\item \hyperref[CT:realArith]{CT:realArith}
\item \hyperref[CT:vectors]{CT:vectors}
\item \hyperref[CT:CartCoordSyst]{CT:CartCoodSyst}
\item \hyperref[CT:Differentiation]{CT:Differentiation}
\end{itemize}

\noindent \textbf{Preconditions for TM:velocity}

\begin{itemize}
\item \hyperref[cartSyst]{A:cartSyst}
\end{itemize}

\medskip
\noindent
\begin{minipage}{\textwidth}
\begin{tabular}{>{\raggedright}p{0.13\textwidth}>{\raggedright\arraybackslash}p{0.82\textwidth}}
\toprule \textbf{Refname} & \textbf{TM:directionCosines}
\phantomsection 
\label{TM:directionCosines}
\\ \midrule
Label & \wss{Direction Cosines Representation for Vectors}
\\ \midrule   
Equation & \begin{displaymath}
           b_x = |\symbf{b}| \cos (\alpha), b_y = |\symbf{b}| \cos (\beta), b_z
           = |\symbf{b}| \cos (\gamma)
           \end{displaymath}
\\ \midrule
Description & \begin{symbDescription}
              \item \wss{$\alpha: \mathbb{R}$} is the angle between the vector and the positive $x$ axis 
              \item \wss{$\beta: \mathbb{R}$} is the angle between the vector and the positive $y$ axis 
              \item \wss{$\gamma: \mathbb{R}$} is the angle between the vector and the positive $z$ axis 
              \item \wss{$|\symbf{b}|: \mathbb{R}$} is the magnitude of the vector
              \item \wss{$b_x: \mathbb{R}$} is the $x$ component of the vector $\symbf{b}$
              \item \wss{$b_y: \mathbb{R}$} is the $y$ component of the vector $\symbf{b}$
              \item \wss{$b_z: \mathbb{R}$} is the $z$ component of the vector $\symbf{b}$
              \end{symbDescription}
\\ \midrule
Notes & \wss{The vector $\symbf{b}$ is in a three dimensional Cartesian
coordinate system (\hyperref[CT:CartCoordSyst]{CT:CartCoordSyst},
\hyperref[cartSyst]{A:cartSyst}).}  \wss{A figure showing the angles for a
sample vector would be a nice addition.}  Direction cosines use the context
theories of real arithmetic, trigonometry and vectors
(\hyperref[CT:realArith]{CT:realArith},
\hyperref[CT:trigonometry]{CT:trigonometry}, \hyperref[CT:vectors]{CT:vectors}).
\\ \midrule

\wss{Constraints} & \wss{$\cos^2(\alpha) + \cos^2(\beta) + \cos^2(\gamma) = 1$}
\\ \midrule

Source & \cite{} \href{https://byjus.com/maths/direction-cosines/} {web-page
resource}, Long book 
         
\\ \midrule
RefBy & \hyperref[GD:magAngleToCompRep]{GD:magAngleToCompRep}
        
\\ \bottomrule
\end{tabular}
\end{minipage}
~\\
\noindent \textbf{Theories Used by TM:directionCosines}

\begin{itemize}
\item \hyperref[CT:realArith]{CT:realArith}
\item \hyperref[CT:trigonometry]{CT:trigonometry}
\item \hyperref[CT:CartCoordSyst]{CT:CartCoordSyst}
\item \hyperref[CT:vectors]{CT:vectors}
\end{itemize}

\noindent \textbf{Preconditions for TM:directionCosines}

\begin{itemize}
\item \hyperref[cartSyst]{A:cartSyst}
\end{itemize}

\subsubsection{\wss{Helper Theories (GD:rectVel and GD:rectPos)}}
\label{Sec:GDs}
This section collects the laws and equations that will be used to build the instance models.
\wss{We should remove the prefix GD.  Maybe we should replace it with the prefix
TM?}
\medskip
\noindent
\begin{minipage}{\textwidth}
\begin{tabular}{>{\raggedright}p{0.13\textwidth}>{\raggedright\arraybackslash}p{0.82\textwidth}}
\toprule \textbf{Refname} & \textbf{GD:rectVel}
\phantomsection 
\label{GD:rectVel}
\\ \midrule
Label & Rectilinear (1D) velocity as a function of time for constant acceleration
        
\\ \midrule
Units & $\frac{\text{m}}{\text{s}}$ \wss{Do we need units as a separate field? user option?}
\\ \midrule
Equation & \begin{displaymath}
           v\text{(}t\text{)}={v^{\text{i}}}+{a^{c}} t
           \end{displaymath}
\\ \midrule
Description & \begin{symbDescription}
              \item{\wss{$v: \text{time} \rightarrow \mathbb{R}$} is the 1D speed ($\frac{\text{m}}{\text{s}}$)}
              \item{\wss{${v^{\text{i}}: \mathbb{R}}$} is the initial speed ($\frac{\text{m}}{\text{s}}$)}
              \item{\wss{${a^{c}: \mathbb{R}}$} is the constant acceleration ($\frac{\text{m}}{\text{s}^{2}}$)}
              \item{\wss{$t: \text{time}$} is the time (${\text{s}}$)}
              \end{symbDescription}

              \\ \midrule
\wss{Notes} & \wss{See detailed derivation below}

\\ \midrule
\wss{Constraints} & \wss{$t \geq 0$} \wss{\hyperref[timeStartZero]{A:timeStartZero}}

\\ \midrule
Source & \cite[(pg. 8)]{hibbeler2004}
         
\\ \midrule
RefBy & \hyperref[GD:velVec]{GD:velVec} and \hyperref[GD:rectPos]{GD:rectPos}
        
\\ \bottomrule
\end{tabular}
\end{minipage}

\paragraph{Detailed derivation of rectilinear velocity:}
\label{GD:rectVelDeriv}
\wss{We start from the theory of acceleration
\hyperref[TM:acceleration]{TM:acceleration} for a body in 3D Cartesian space:}

$$\symbf{a}\text{(}t\text{)}=\frac{\,d\symbf{v}\text{(}t\text{)}}{\,dt}$$

\wss{At this point we assume the body only travels in a straight line along one
dimension of the coordinate system (\hyperref[oneD]{A:oneD}):}

$${a}\text{(}t\text{)}=\frac{\,d{v}\text{(}t\text{)}}{\,dt}$$

\wss{We now assume that the acceleration is constant (does not vary with time)
(\hyperref[constAccel]{A:constAccel}) represented by ${a^{c}}$. The initial
velocity (at $t=0$, from \hyperref[timeStartZero]{A:timeStartZero}) is
represented by ${v^{\text{i}}}$. We now have:}

\begin{displaymath}
{a^{c}}=\frac{\,dv}{\,dt}
\end{displaymath}
Rearranging and integrating, we have:

\begin{displaymath}
\int_{{v^{\text{i}}}}^{v}{1}\,dv=\int_{0}^{t}{{a^{c}}}\,dt
\end{displaymath}
Performing the integration, we have the required equation:

\begin{displaymath}
v\text{(}t\text{)}={v^{\text{i}}}+{a^{c}} t
\end{displaymath}

\wss{The theory for rectilinear velocity uses the context theories from
\hyperref[TM:acceleration]{TM:acceleration} of real arithmetic, vectors,
Cartesian coodinate system, and differentiation
(\hyperref[CT:realArith]{CT:realArith}, \hyperref[CT:vectors]{CT:vectors},
\hyperref[CT:CartCoordSyst]{CT:CartCoordSyst},
\hyperref[CT:Differentiation]{CT:Differentiation}).  In addition, the context
theory of integration is used (\hyperref[CT:Integration]{CT:Integration}).}
~\\

\noindent \textbf{Theories Used by GD:rectVel}

\begin{itemize}
\item \hyperref[TM:acceleration]{TM:acceleration}
\item \hyperref[CT:realArith]{CT:realArith}
\item \hyperref[CT:vectors]{CT:vectors}
\item \hyperref[CT:CartCoordSyst]{CT:CartCoordSyst}
\item \hyperref[CT:Differentiation]{CT:Differentiation}
\item \hyperref[CT:Integration]{CT:Integration}
\end{itemize}

\noindent \textbf{\wss{Preconditions for GD:rectVel}}

\begin{itemize}
\item \hyperref[cartSyst]{A:cartSyst} (inherited from \hyperref[TM:acceleration]{TM:acceleration})
\item \hyperref[oneD]{A:oneD}
\item \hyperref[constAccel]{A:constAccel}
\item \hyperref[timeStartZero]{A:timeStartZero}
\end{itemize}

\medskip
\noindent
\begin{minipage}{\textwidth}
\begin{tabular}{>{\raggedright}p{0.13\textwidth}>{\raggedright\arraybackslash}p{0.82\textwidth}}
\toprule \textbf{Refname} & \textbf{GD:rectPos}
\phantomsection 
\label{GD:rectPos}
\\ \midrule
Label & Rectilinear (1D) position as a function of time for constant acceleration
        
\\ \midrule
Units & ${\text{m}}$
        
\\ \midrule
Equation & \begin{displaymath}
           p\text{(}t\text{)}={p^{\text{i}}}+{v^{\text{i}}} t+\frac{{a^{c}} t^{2}}{2}
           \end{displaymath}

\\ \midrule
Description & \begin{symbDescription}
              \item{\wss{$p: \text{time} \rightarrow \mathbb{R}$} is the 1D position (${\text{m}}$)}
              \item{\wss{${p^{\text{i}}}: \mathbb{R}$} is the initial position (${\text{m}}$)}
              \item{\wss{${v^{\text{i}}}: \mathbb{R}$} is the initial speed ($\frac{\text{m}}{\text{s}}$)}
              \item{\wss{$t: \mathbb{R}$} is the time (${\text{s}}$)}
              \item{\wss{${a^{c}}: \mathbb{R}$} is the constant acceleration ($\frac{\text{m}}{\text{s}^{2}}$)}
              \end{symbDescription}
\\ \midrule
\wss{Notes} & \wss{See detailed derivation below}

\\ \midrule
\wss{Constraints} & \wss{$t \geq 0$} \wss{\hyperref[timeStartZero]{A:timeStartZero}}

\\ \midrule
Source & \cite[(pg. 8)]{hibbeler2004}
         
\\ \midrule
RefBy & \hyperref[GD:posVec]{GD:posVec}
        
\\ \bottomrule
\end{tabular}
\end{minipage}

\paragraph{Detailed derivation of rectilinear position:}
\label{GD:rectPosDeriv}
\wss{We start from the kinematic equation for velocity
\hyperref[TM:velocity]{TM:velocity} for a body in 3D Cartesian space:}

$$\symbf{v}\text{(}t\text{)}=\frac{\,d\symbf{p}\text{(}t\text{)}}{\,dt}$$

\wss{At this point we assume the body only travels in a straight line along one
dimension of the coordinate system (\hyperref[oneD]{A:oneD}):}

$${v}\text{(}t\text{)}=\frac{\,d{p}\text{(}t\text{)}}{\,dt}$$

\wss{The initial position (at $t=0$, from
\hyperref[timeStartZero]{A:timeStartZero}) is represented by ${p^{\text{i}}}$.
Rearranging the above equation and integrating we have:}

\begin{displaymath}
\int_{{p^{\text{i}}}}^{p(t)}{1}\,dp=\int_{0}^{t}{v(t)}\,dt
\end{displaymath}
\wss{This equation has been changed from the previous version to show $p(t)$ and
$v(t)$}

\wss{We now assume that the acceleration is constant (does not vary with time)
(\hyperref[constAccel]{A:constAccel}) represented by ${a^{c}}$. The initial
velocity (at $t=0$, from \hyperref[timeStartZero]{A:timeStartZero}) is
represented by ${v^{\text{i}}}$. Since we satisfy the preconditions for
\hyperref[GD:rectVel]{GD:rectVel} we can replace $v(t)$ to get:}

\begin{displaymath}
\int_{{p^{\text{i}}}}^{p(t)}{1}\,dp=\int_{0}^{t}{{v^{\text{i}}}+{a^{c}} t}\,dt
\end{displaymath}
\wss{The above equation has been changed from the previous version to show
$p(t)$.}
Performing the integration, we have the required equation:

\begin{displaymath}
p\text{(}t\text{)}={p^{\text{i}}}+{v^{\text{i}}} t+\frac{{a^{c}} t^{2}}{2}
\end{displaymath}

\wss{The theory for rectilinear position uses the context theories from
\hyperref[TM:acceleration]{TM:acceleration} of real arithmetic, vectors,
Cartesian coodinate system, and differentiation
(\hyperref[CT:realArith]{CT:realArith}, \hyperref[CT:vectors]{CT:vectors},
\hyperref[CT:CartCoordSyst]{CT:CartCoordSyst},
\hyperref[CT:Differentiation]{CT:Differentiation}).  In addition, the context
theory of integration is used (\hyperref[CT:Integration]{CT:Integration}).}
~\\

\noindent \textbf{Theories Used by GD:rectPos}

\begin{itemize}
\item \hyperref[TM:velocity]{TM:velocity}
\item \hyperref[CT:realArith]{CT:realArith}
\item \hyperref[CT:vectors]{CT:vectors}
\item \hyperref[CT:CartCoordSyst]{CT:CartCoordSyst}
\item \hyperref[CT:Differentiation]{CT:Differentiation}
\item \hyperref[CT:Integration]{CT:Integration}
\end{itemize}

\noindent \textbf{Preconditions for GD:rectPos}

\begin{itemize}
\item \hyperref[cartSyst]{A:cartSyst} (inherited from \hyperref[TM:velocity]{TM:velocity})
\item \hyperref[oneD]{A:oneD}
\item \hyperref[constAccel]{A:constAccel}
\item \hyperref[timeStartZero]{A:timeStartZero}
\end{itemize}

\subsubsection{\wss{Theories (GD:velVec and GD:posVec)}}

\noindent
\begin{minipage}{\textwidth}
\begin{tabular}{>{\raggedright}p{0.13\textwidth}>{\raggedright\arraybackslash}p{0.82\textwidth}}
\toprule \textbf{Refname} & \textbf{GD:velVec}
\phantomsection 
\label{GD:velVec}
\\ \midrule
Label & Velocity vector as a function of time for 2D motion under constant acceleration
        
\\ \midrule
Units & $\frac{\text{m}}{\text{s}}$
        
\\ \midrule
Equation & \begin{displaymath}
           \symbf{v}\text{(}t\text{)}=\begin{bmatrix}
                                      {{v_{\text{x}}}^{\text{i}}}+{{a_{\text{x}}}^{\text{c}}} t\\
                                      {{v_{\text{y}}}^{\text{i}}}+{{a_{\text{y}}}^{\text{c}}} t
                                      \end{bmatrix}
           \end{displaymath}
\\ \midrule
Description & \begin{symbDescription}
              \item{\wss{$\symbf{v}: \text{time} \rightarrow \mathbb{R}^2$} is the velocity \wss{vector} ($\frac{\text{m}}{\text{s}}$)}
              \item{\wss{${{v_{\text{x}}}^{\text{i}}}: \mathbb{R}$} is the $x$-component of initial velocity ($\frac{\text{m}}{\text{s}}$)}
              \item{\wss{${{a_{\text{x}}}^{\text{c}}}: \mathbb{R}$} is the $x$-component of constant acceleration ($\frac{\text{m}}{\text{s}^{2}}$)}
              \item{\wss{$t$: \text{time}} is the time (${\text{s}}$)}
              \item{\wss{${{v_{\text{y}}}^{\text{i}}}: \mathbb{R}$} is the $y$-component of initial velocity ($\frac{\text{m}}{\text{s}}$)}
              \item{\wss{${{a_{\text{y}}}^{\text{c}}}: \mathbb{R}$} is the $y$-component of constant acceleration ($\frac{\text{m}}{\text{s}^{2}}$)}
              \end{symbDescription}
\\ \midrule

\wss{Constraints} & \wss{$t \geq 0$} \wss{\hyperref[timeStartZero]{A:timeStartZero}}
\\ \midrule

Source & --
         
\\ \midrule
RefBy & 
\\ \bottomrule
\end{tabular}
\end{minipage}

\paragraph{Detailed derivation of velocity vector:}
\label{GD:velVecDeriv}
For a two-dimensional Cartesian coordinate system (\hyperref[twoD]{A:twoD} and \hyperref[cartSyst]{A:cartSyst}), we can represent the velocity vector as $\symbf{v}\text{(}t\text{)}=\begin{bmatrix}
                                                                                                                                                                                                 {v_{\text{x}}(\wss{t})}\\
                                                                                                                                                                                                 {v_{\text{y}}(\wss{t})}
                                                                                                                                                                                                 \end{bmatrix}$ and the acceleration vector as $\symbf{a}\text{(}t\text{)}=\begin{bmatrix}
                                                                                                                                                                                                                                                                           {a_{\text{x}}(\wss{t})}\\
                                                                                                                                                                                                                                                                           {a_{\text{y}}(\wss{t})}
                                                                                                                                                                                                                                                                           \end{bmatrix}$. The acceleration is assumed to be constant in both the $x$ direction ${a_{\text{x}}}^{\text{c}}$ (\hyperref[constAccelX]{A:constAccelX}) and $y$ direction ${a_{\text{y}}}^{\text{c}}$ (\hyperref[constAccelY]{A:constAccelY}). The constant acceleration vector is represented as ${\symbf{a}^{\text{c}}}=\begin{bmatrix}
                                                                                                                                                                                                                                                                                                                                                                                                                                                           {{a_{\text{x}}}^{\text{c}}}\\
                                                                                                                                                                                                                                                                                                                                                                                                                                                           {{a_{\text{y}}}^{\text{c}}}
                                                                                                                                                                                                                                                                                                                                                                                                                                                           \end{bmatrix}$. The initial velocity (at $t=0$, from \hyperref[timeStartZero]{A:timeStartZero}) is represented by ${\symbf{v}^{\text{i}}}=\begin{bmatrix}
                                                                                                                                                                                                                                                                                                                                                                                                                                                                                                                                                                                                     {{v_{\text{x}}}^{\text{i}}}\\
                                                                                                                                                                                                                                                                                                                                                                                                                                                                                                                                                                                                     {{v_{\text{y}}}^{\text{i}}}
                                                                                                                                                                                                                                                                                                                                                                                                                                                                                                                                                                                                     \end{bmatrix}$. \wss{For each coordinate direction we satisfy the preconditions for \hyperref[GD:rectVel]{GD:rectVel} (\hyperref[cartSyst]{A:cartSyst}, \hyperref[oneD]{A:oneD}, \hyperref[constAccel]{A:constAccel}, \hyperref[timeStartZero]{A:timeStartZero}). This means we can use the one dimensional equation for each of the two coordinate directions to yield the required equation:}

\begin{displaymath}
\symbf{v}\text{(}t\text{)}=\begin{bmatrix}
                           {{v_{\text{x}}}^{\text{i}}}+{{a_{\text{x}}}^{\text{c}}} t\\
                           {{v_{\text{y}}}^{\text{i}}}+{{a_{\text{y}}}^{\text{c}}} t
                           \end{bmatrix}
\end{displaymath}

\wss{The theory for the velocity vector for rectilinear motion in 2D uses the
context theories from \hyperref[GD:rectVel]{GD:rectVel} of real arithmetic,
vectors, Cartesian coodinate system, differentiation and integration
(\hyperref[CT:realArith]{CT:realArith}, \hyperref[CT:vectors]{CT:vectors},
\hyperref[CT:CartCoordSyst]{CT:CartCoordSyst},
\hyperref[CT:Differentiation]{CT:Differentiation},
\hyperref[CT:Integration]{CT:Integration}).}
~\\

\noindent \textbf{Theories Used by GD:velVect}

\begin{itemize}
\item \hyperref[GD:rectVel]{GD:rectVel}
\item \hyperref[CT:realArith]{CT:realArith}
\item \hyperref[CT:vectors]{CT:vectors}
\item \hyperref[CT:CartCoordSyst]{CT:CartCoordSyst}
\item \hyperref[CT:Differentiation]{CT:Differentiation}
\item \hyperref[CT:Integration]{CT:Integration}
\end{itemize}

\wss{If a theory A depends another theory B and theory B uses context theory C,
does theory A also depend on context theory C, even if context theory C never
appears in the expression or derivation of theory A?}

\noindent \textbf{Preconditions for GD:velVec}
\begin{itemize}
\item \hyperref[timeStartZero]{A:timeStartZero} (inherited from \hyperref[GD:rectVel]{GD:rectVel})
\item \hyperref[cartSyst]{A:cartSyst} (inherited from
\hyperref[GD:rectVel]{GD:rectVel})
\item \hyperref[twoD]{A:twoD} (\hyperref[oneD]{A:oneD} in both the $x$ and $y$ directions)
\item \hyperref[constAccelX]{A:constAccelX} (\hyperref[constAccel]{A:constAccel}
in $x$ direction)
\item \hyperref[constAccelY]{A:constAccelY} (\hyperref[constAccel]{A:constAccel}
in $y$ direction)
\end{itemize}

\wss{The assumptions (preconditions) for GD:velVec are not a superset of the assumptions for the theories it is based on.  Specifically, \hyperref[constAccel]{A:constAccel} becomes two assumptions: \hyperref[constAccel]{A:constAccel} and \hyperref[constAccelY]{A:constAccelY}.  Moreover, the assumption \hyperref[oneD]{A:oneD} becomes \hyperref[twoD]{A:twoD}.}

\medskip
\noindent
\begin{minipage}{\textwidth}
\begin{tabular}{>{\raggedright}p{0.13\textwidth}>{\raggedright\arraybackslash}p{0.82\textwidth}}
\toprule \textbf{Refname} & \textbf{GD:posVec}
\phantomsection 
\label{GD:posVec}
\\ \midrule
Label & Position vector as a function of time for 2D motion under constant acceleration
        
\\ \midrule
Units & ${\text{m}}$
        
\\ \midrule
Equation & \begin{displaymath}
           \symbf{p}\text{(}t\text{)}=\begin{bmatrix}
                                      {{p_{\text{x}}}^{\text{i}}}+{{v_{\text{x}}}^{\text{i}}} t+\frac{{{a_{\text{x}}}^{\text{c}}} t^{2}}{2}\\
                                      {{p_{\text{y}}}^{\text{i}}}+{{v_{\text{y}}}^{\text{i}}} t+\frac{{{a_{\text{y}}}^{\text{c}}} t^{2}}{2}
                                      \end{bmatrix}
           \end{displaymath}
\\ \midrule
Description & \begin{symbDescription}
              \item{\wss{$\symbf{p}: \text{time} \rightarrow \mathbb{R}$} is the position (${\text{m}}$)}
              \item{\wss{${{p_{\text{x}}}^{\text{i}}}: \mathbb{R}$} is the $x$-component of initial position (${\text{m}}$)}
              \item{\wss{${{v_{\text{x}}}^{\text{i}}}: \mathbb{R}$} is the $x$-component of initial velocity ($\frac{\text{m}}{\text{s}}$)}
              \item{\wss{$t: \text{time}$} is the time (${\text{s}}$)}
              \item{\wss{${{a_{\text{x}}}^{\text{c}}}: \mathbb{R}$} is the $x$-component of constant acceleration ($\frac{\text{m}}{\text{s}^{2}}$)}
              \item{\wss{${{p_{\text{y}}}^{\text{i}}}: \mathbb{R}$} is the $y$-component of initial position (${\text{m}}$)}
              \item{\wss{${{v_{\text{y}}}^{\text{i}}}: \mathbb{R}$} is the $y$-component of initial velocity ($\frac{\text{m}}{\text{s}}$)}
              \item{\wss{${{a_{\text{y}}}^{\text{c}}}: \mathbb{R}$} is the $y$-component of constant acceleration ($\frac{\text{m}}{\text{s}^{2}}$)}
              \end{symbDescription}

              \\ \midrule
\wss{Constraints} & \wss{$t \geq 0$} \wss{\hyperref[timeStartZero]{A:timeStartZero}}

\\ \midrule
Source & --
         
\\ \midrule
RefBy & \hyperref[IM:calOfLandingDist]{IM:calOfLandingDist} and \hyperref[IM:calOfLandingTime]{IM:calOfLandingTime}
        
\\ \bottomrule
\end{tabular}
\end{minipage}

\paragraph{Detailed derivation of position vector:}
\label{GD:posVecDeriv}
For a two-dimensional Cartesian coordinate system (\hyperref[twoD]{A:twoD} and \hyperref[cartSyst]{A:cartSyst}), we can represent the position vector as $\symbf{p}\text{(}t\text{)}=\begin{bmatrix}
                                                                                                                                                                                                 {p_{\text{x}}\wss{(t)}}\\
                                                                                                                                                                                                 {p_{\text{y}}\wss{(t)}}
                                                                                                                                                                                                 \end{bmatrix}$, the velocity vector as $\symbf{v}\text{(}t\text{)}=\begin{bmatrix}
                                                                                                                                                                                                                                                                    {v_{\text{x}}}\\
                                                                                                                                                                                                                                                                    {v_{\text{y}}}
                                                                                                                                                                                                                                                                    \end{bmatrix}$, and the acceleration vector as $\symbf{a}\text{(}t\text{)}=\begin{bmatrix}
                                                                                                                                                                                                                                                                                                                                               {a_{\text{x}}\wss{(t)}}\\
                                                                                                                                                                                                                                                                                                                                               {a_{\text{y}}\wss{(t)}}
                                                                                                                                                                                                                                                                                                                                               \end{bmatrix}$. The acceleration is assumed to be constant (\hyperref[constAccel]{A:constAccel}) and the constant acceleration vector is represented as ${\symbf{a}^{\text{c}}}=\begin{bmatrix}
                                                                                                                                                                                                                                                                                                                                                                                                                                                                                                                               {{a_{\text{x}}}^{\text{c}}}\\
                                                                                                                                                                                                                                                                                                                                                                                                                                                                                                                               {{a_{\text{y}}}^{\text{c}}}
                                                                                                                                                                                                                                                                                                                                                                                                                                                                                                                               \end{bmatrix}$. The initial velocity (at $t=0$, from \hyperref[timeStartZero]{A:timeStartZero}) is represented by ${\symbf{v}^{\text{i}}}=\begin{bmatrix}
                                                                                                                                                                                                                                                                                                                                                                                                                                                                                                                                                                                                                                                                         {{v_{\text{x}}}^{\text{i}}}\\
                                                                                                                                                                                                                                                                                                                                                                                                                                                                                                                                                                                                                                                                         {{v_{\text{y}}}^{\text{i}}}
                                                                                                                                                                                                                                                                                                                                                                                                                                                                                                                                                                                                                                                                         \end{bmatrix}$. \wss{For each coordinate direction we satisfy the preconditions for \hyperref[GD:rectPos]{GD:rectPos} (\hyperref[cartSyst]{A:cartSyst}, \hyperref[oneD]{A:oneD}, \hyperref[constAccel]{A:constAccel}, \hyperref[timeStartZero]{A:timeStartZero}). This means we can use the one dimensional equation for each of the two coordinate directions to yield the required equation:} 

\begin{displaymath}
\symbf{p}\text{(}t\text{)}=\begin{bmatrix}
                           {{p_{\text{x}}}^{\text{i}}}+{{v_{\text{x}}}^{\text{i}}} t+\frac{{{a_{\text{x}}}^{\text{c}}} t^{2}}{2}\\
                           {{p_{\text{y}}}^{\text{i}}}+{{v_{\text{y}}}^{\text{i}}} t+\frac{{{a_{\text{y}}}^{\text{c}}} t^{2}}{2}
                           \end{bmatrix}
\end{displaymath}

\wss{The theory for the position vector for rectilinear motion in 2D uses the
context theories from \hyperref[GD:rectPos]{GD:rectPos} of real arithmetic,
vectors, Cartesian coodinate system, differentiation and integration
(\hyperref[CT:realArith]{CT:realArith}, \hyperref[CT:vectors]{CT:vectors},
\hyperref[CT:CartCoordSyst]{CT:CartCoordSyst},
\hyperref[CT:Differentiation]{CT:Differentiation},
\hyperref[CT:Integration]{CT:Integration}).}
~\\

\noindent \textbf{Theories Used by GD:posVec}

\begin{itemize}
\item \hyperref[GD:rectPos]{GD:rectPos}
\item \hyperref[CT:realArith]{CT:realArith}
\item \hyperref[CT:vectors]{CT:vectors}
\item \hyperref[CT:CartCoordSyst]{CT:CartCoordSyst}
\item \hyperref[CT:Differentiation]{CT:Differentiation}
\item \hyperref[CT:Integration]{CT:Integration}
\end{itemize}

\noindent \textbf{Preconditions for GD:posVec}
\begin{itemize}
\item \hyperref[timeStartZero]{A:timeStartZero} (inherited from \hyperref[GD:rectPos]{GD:rectPos})
\item \hyperref[cartSyst]{A:cartSyst} (inherited from
\hyperref[GD:rectPos]{GD:rectPos})
\item \hyperref[twoD]{A:twoD} (\hyperref[oneD]{A:oneD} in both the $x$ and $y$ directions)
\item \hyperref[constAccelX]{A:constAccelX} (\hyperref[constAccel]{A:constAccel}
in $x$ direction)
\item \hyperref[constAccelY]{A:constAccelY} (\hyperref[constAccel]{A:constAccel}
in $y$ direction)
\end{itemize}

\medskip
\noindent
\begin{minipage}{\textwidth}
\begin{tabular}{>{\raggedright}p{0.13\textwidth}>{\raggedright\arraybackslash}p{0.82\textwidth}}
\toprule \textbf{Refname} & \textbf{GD:magAngleToCompRep}
\phantomsection 
\label{GD:magAngleToCompRep}
\\ \midrule
Label & \wss{Conversion of Magnitude and Angle Representation of a Vector to the
Component Representation}
        
\\ \midrule
Equation & \begin{displaymath}
           b_x = |\symbf{b}| \cos (\theta), b_y = |\symbf{b}| \sin (\theta)
           \end{displaymath}
\\ \midrule
Description & \begin{symbDescription}
              \item \wss{$\theta: \mathbb{R}$} is the angle between the vector and the positive $x$ axis 
              \item \wss{$|\symbf{b}|: \mathbb{R}$} is the magnitude of the vector
              \item \wss{$b_x: \mathbb{R}$} is the $x$ component of the vector $\symbf{b}$
              \item \wss{$b_y: \mathbb{R}$} is the $y$ component of the vector $\symbf{b}$
              \end{symbDescription}
\\ \midrule
Notes & \wss{The vector $\symbf{b}$ is in a two dimensional
(\hyperref[twoD]{A:twoD}) Cartesian coordinate system
(\hyperref[cartSyst]{A:cartSyst}).} \wss{The equations can be derived from
\hyperref[TM:directionCosines]{TM:directionCosines} for a 2D sytem.  In a 2D system,
the angle $\gamma$ is not relevant because in this case $\gamma = \pi/2$, and 
$\cos(\gamma) = \cos(\pi/2) = 0$.  For the 2D case we rename the angle $\alpha$
as $\theta$.  The angle $\beta$ is related to $\theta$ by $\beta = \pi/2 -
\theta$; therefore, $\cos(\beta) = \cos(\pi/2 - \theta) = \sin(\theta)$.}  This theory uses the same context theories as \hyperref[TM:directionCosines]{TM:directionCosines}: \hyperref[CT:realArith]{CT:realArith}, \hyperref[CT:trigonometry]{CT:trigonometry}, \hyperref[CT:CartCoordSyst]{CT:CartCoodSyst} and \hyperref[CT:vectors]{CT:vectors}.

\\ \midrule
\wss{Constraints} & \wss{None}

\\ \midrule
Source & \cite{}
         
\\ \midrule
RefBy & \hyperref[DD:speedIX]{DD:speedIX}, \hyperref[DD:speedIY]{DD:speedIY}
        
\\ \bottomrule
\end{tabular}
\end{minipage}

~\\
\noindent \textbf{Theories Used by GD:magAngleToCompRep}

\begin{itemize}
\item \hyperref[TM:directionCosines]{TM:directionCosines}
\item \hyperref[CT:realArith]{CT:realArith}
\item \hyperref[CT:trigonometry]{CT:trigonometry}
\item \hyperref[CT:vectors]{CT:vectors}
\item \hyperref[CT:CartCoordSyst]{CT:CartCoordSyst}
\end{itemize}
~\\
\noindent \textbf{Preconditions for GD:magAngleToCompRep}

\begin{itemize}
\item \hyperref[cartSyst]{A:cartSyst} (inherited from \hyperref[directionCosines]{directionCosines})
\item \hyperref[twoD]{A:twoD}
\end{itemize}

\subsubsection{Data Definitions}
\label{Sec:DDs}
This section collects and defines all the data needed to build the instance models.

\medskip
\noindent
\begin{minipage}{\textwidth}
\begin{tabular}{>{\raggedright}p{0.13\textwidth}>{\raggedright\arraybackslash}p{0.82\textwidth}}
\toprule \textbf{Refname} & \textbf{DD:vecMag} \wss{REMOVE}
\phantomsection 
\label{DD:vecMag}
\\ \midrule
Label & Speed        
\\ \bottomrule
\end{tabular}
\end{minipage}

\medskip
\noindent
\begin{minipage}{\textwidth}
\begin{tabular}{>{\raggedright}p{0.13\textwidth}>{\raggedright\arraybackslash}p{0.82\textwidth}}
\toprule \textbf{Refname} & \textbf{DD:speedIX}
\phantomsection 
\label{DD:speedIX}
\\ \midrule
Label & $x$-component of initial velocity
        
\\ \midrule
Symbol & ${{v_{\text{x}}}^{\text{i}}}$
         
\\ \midrule
Units & $\frac{\text{m}}{\text{s}}$
        
\\ \midrule
Equation & \begin{displaymath}
           {{v_{\text{x}}}^{\text{i}}}={v^{\text{i}}} \cos\left(θ\right)
           \end{displaymath}
\\ \midrule
Description & \begin{symbDescription}
              \item{\wss{${{v_{\text{x}}}^{\text{i}}}: \mathbb{R}$} is the $x$-component of initial velocity ($\frac{\text{m}}{\text{s}}$)}
              \item{\wss{${v^{\text{i}}}: \mathbb{R}$} is the initial speed ($\frac{\text{m}}{\text{s}}$)}
              \item{\wss{$θ: \mathbb{R}$} is the launch angle (${\text{rad}}$)}
              \end{symbDescription}
\\ \midrule
Notes & \wss{This equation is a relabelling of the $x$ component of
\hyperref[GD:magAngleToCompRep]{GD:magAngleToCompRep} --- ${v^{\text{i}}}$ is
$|\symbf{b}|$, ${v_{\text{x}}}^{\text{i}}$ is $b_x$ and $\theta$ is $\theta$.}
$θ$ is shown in \hyperref[Figure:Launch]{Fig:Launch}.  This equation inherits the assumptions from \hyperref[GD:magAngleToCompRep]{GD:magAngleToCompRep}: \hyperref[cartSyst]{A:cartSyst} and \hyperref[twoD]{A:twoD}.
        
\\ \midrule
Source & --
         
\\ \midrule
RefBy & \hyperref[IM:calOfLandingDist]{IM:calOfLandingDist}
        
\\ \bottomrule
\end{tabular}
\end{minipage}
~\\

\noindent \textbf{Theories Used by DD:speedIX}

\begin{itemize}
\item \hyperref[GD:magAngleToCompRep]{GD:magAngleToCompRep}
\item \hyperref[TM:directionCosines]{TM:directionCosines}
\item \hyperref[CT:realArith]{CT:realArith}
\item \hyperref[CT:trigonometry]{CT:trigonometry}
\item \hyperref[CT:CartCoordSyst]{CT:CartCoordSyst}
\item \hyperref[CT:vectors]{CT:vectors}
\end{itemize}

\noindent \textbf{Preconditions for DD:speedIX}

\begin{itemize}
\item \hyperref[cartSyst]{A:cartSyst} (inherited from \hyperref[GD:magAngleToCompRep]{GD:magAngleToCompRep})
\item \hyperref[twoD]{A:twoD} (inherited from \hyperref[GD:magAngleToCompRep]{GD:magAngleToCompRep})
\end{itemize}

\medskip
\noindent
\begin{minipage}{\textwidth}
\begin{tabular}{>{\raggedright}p{0.13\textwidth}>{\raggedright\arraybackslash}p{0.82\textwidth}}
\toprule \textbf{Refname} & \textbf{DD:speedIY}
\phantomsection 
\label{DD:speedIY}
\\ \midrule
Label & $y$-component of initial velocity
        
\\ \midrule
Symbol & ${{v_{\text{y}}}^{\text{i}}}$
         
\\ \midrule
Units & $\frac{\text{m}}{\text{s}}$
        
\\ \midrule
Equation & \begin{displaymath}
           {{v_{\text{y}}}^{\text{i}}}={v^{\text{i}}} \sin\left(θ\right)
           \end{displaymath}
\\ \midrule
Description & \begin{symbDescription}
              \item{\wss{${{v_{\text{y}}}^{\text{i}}}: \mathbb{R}$} is the $y$-component of initial velocity ($\frac{\text{m}}{\text{s}}$)}
              \item{\wss{${v^{\text{i}}}: \mathbb{R}$} is the initial speed ($\frac{\text{m}}{\text{s}}$)}
              \item{\wss{$θ: \mathbb{R}$}is the launch angle (${\text{rad}}$)}
              \end{symbDescription}
\\ \midrule
Notes & \wss{This equation is a relabelling of the $x$ component of
\hyperref[GD:magAngleToCompRep]{GD:magAngleToCompRep} --- ${v^{\text{i}}}$ is
$|\symbf{b}|$, ${v_{\text{y}}}^{\text{i}}$ is $b_y$ and $\theta$ is $\theta$.}
$θ$ is shown in \hyperref[Figure:Launch]{Fig:Launch}.   This equation inherits
the assumptions from \hyperref[GD:magAngleToCompRep]{GD:magAngleToCompRep}:
\hyperref[cartSyst]{A:cartSyst} and \hyperref[twoD]{A:twoD}.
        
\\ \midrule
Source & --
         
\\ \midrule
RefBy & \hyperref[IM:calOfLandingTime]{IM:calOfLandingTime}
        
\\ \bottomrule
\end{tabular}
\end{minipage}
~\\

\noindent \textbf{Theories Used by DD:speedIY}

\begin{itemize}
\item \hyperref[GD:magAngleToCompRep]{GD:magAngleToCompRep}
\item \hyperref[TM:directionCosines]{TM:directionCosines}
\item \hyperref[CT:realArith]{CT:realArith}
\item \hyperref[CT:vectors]{CT:vectors}
\item \hyperref[CT:CartCoordSyst]{CT:CartCoordSyst}
\end{itemize}

\noindent \textbf{Preconditions for DD:speedIY}

\begin{itemize}
\item \hyperref[cartSyst]{A:cartSyst} (inherited from \hyperref[GD:magAngleToCompRep]{GD:magAngleToCompRep})
\item \hyperref[twoD]{A:twoD} (inherited from \hyperref[GD:magAngleToCompRep]{GD:magAngleToCompRep})
\end{itemize}

\subsubsection{Instance Models}
\label{Sec:IMs}
This section transforms the problem defined in the \hyperref[Sec:ProbDesc]{problem description} into one which is expressed in mathematical terms. It uses concrete symbols defined in the \hyperref[Sec:DDs]{data definitions} to replace the abstract symbols in the models identified in \hyperref[Sec:TMs]{theoretical models} and \hyperref[Sec:GDs]{general definitions}.

\medskip
\noindent
\begin{minipage}{\textwidth}
\begin{tabular}{>{\raggedright}p{0.13\textwidth}>{\raggedright\arraybackslash}p{0.82\textwidth}}
\toprule \textbf{Refname} & \textbf{IM:calOfLandingTime}
\phantomsection 
\label{IM:calOfLandingTime}
\\ \midrule
Label & Calculation of landing time
        
\\ \midrule
Input & ${v_{\text{launch}}}$ \wss{:$\mathbb{R}$}, $θ$ \wss{:$\mathbb{R}$}
        
\\ \midrule
Output & ${t_{\text{flight}}}$ \wss{:$\mathbb{R}$}
         
\\ \midrule
Input Constraints & \begin{displaymath}
                    {v_{\text{launch}}}\gt{}0
                    \end{displaymath}
                    \begin{displaymath}
                    0\lt{}θ\lt{}\frac{π}{2}
                    \end{displaymath}
\\ \midrule
Output Constraints & \begin{displaymath}
                     {t_{\text{flight}}}\gt{}0
                     \end{displaymath}
\\ \midrule
Equation & \begin{displaymath}
           {t_{\text{flight}}}=\frac{2 {v_{\text{launch}}} \sin\left(θ\right)}{g}
           \end{displaymath}
\\ \midrule
Description & \begin{symbDescription}
              \item{${t_{\text{flight}}}$ is the flight duration (${\text{s}}$)}
              \item{${v_{\text{launch}}}$ is the launch speed ($\frac{\text{m}}{\text{s}}$)}
              \item{$θ$ is the launch angle (${\text{rad}}$)}
              \item{$g$ is the magnitude of gravitational acceleration ($\frac{\text{m}}{\text{s}^{2}}$)}
              \end{symbDescription}
\\ \midrule
Notes & The constraint $0\lt{}θ\lt{}\frac{π}{2}$ is from \hyperref[posXDirection]{A:posXDirection} and \hyperref[yAxisGravity]{A:yAxisGravity}, and is shown in \hyperref[Figure:Launch]{Fig:Launch}.
        
        $g$ is defined in \hyperref[gravAccelValue]{A:gravAccelValue}.
        
        The constraint ${t_{\text{flight}}}\gt{}0$ is from \hyperref[timeStartZero]{A:timeStartZero}.
        
\\ \midrule
Source & --
         
\\ \midrule
RefBy & \hyperref[IM:calOfLandingDist]{IM:calOfLandingDist}, \hyperref[outputValues]{FR:Output-Values}, and \hyperref[calcValues]{FR:Calculate-Values}
        
\\ \bottomrule
\end{tabular}
\end{minipage}

\paragraph{Detailed derivation of flight duration:}
\label{IM:calOfLandingTimeDeriv}
We know that ${{p_{\text{y}}}^{\text{i}}}=0$ (\hyperref[launchOrigin]{A:launchOrigin}) and ${{a_{\text{y}}}^{\text{c}}}=-g$ (\hyperref[accelYGravity]{A:accelYGravity}). Substituting these values into the y-direction of \hyperref[GD:posVec]{GD:posVec} gives us:

\begin{displaymath}
{p_{\text{y}}}={{v_{\text{y}}}^{\text{i}}} t-\frac{g t^{2}}{2}
\end{displaymath}
To find the time that the projectile lands, we want to find the $t$ value (${t_{\text{flight}}}$) where ${p_{\text{y}}}=0$ (since the target is on the $x$-axis from \hyperref[targetXAxis]{A:targetXAxis}). From the equation above we get:

\begin{displaymath}
{{v_{\text{y}}}^{\text{i}}} {t_{\text{flight}}}-\frac{g {t_{\text{flight}}}^{2}}{2}=0
\end{displaymath}
Dividing by ${t_{\text{flight}}}$ (with the constraint ${t_{\text{flight}}}\gt{}0$) gives us:

\begin{displaymath}
{{v_{\text{y}}}^{\text{i}}}-\frac{g {t_{\text{flight}}}}{2}=0
\end{displaymath}
Solving for ${t_{\text{flight}}}$ gives us:

\begin{displaymath}
{t_{\text{flight}}}=\frac{2 {{v_{\text{y}}}^{\text{i}}}}{g}
\end{displaymath}
From \hyperref[DD:speedIY]{DD:speedIY} (with ${v^{\text{i}}}={v_{\text{launch}}}$) we can replace ${{v_{\text{y}}}^{\text{i}}}$:

\begin{displaymath}
{t_{\text{flight}}}=\frac{2 {v_{\text{launch}}} \sin\left(θ\right)}{g}
\end{displaymath}
\noindent \textbf{Preconditions for IM:calOfLandingTime}
\begin{itemize}
\item 
\end{itemize}

\medskip
\noindent
\begin{minipage}{\textwidth}
\begin{tabular}{>{\raggedright}p{0.13\textwidth}>{\raggedright\arraybackslash}p{0.82\textwidth}}
\toprule \textbf{Refname} & \textbf{IM:calOfLandingDist}
\phantomsection 
\label{IM:calOfLandingDist}
\\ \midrule
Label & Calculation of landing position
        
\\ \midrule
Input & ${v_{\text{launch}}}$, $θ$
        
\\ \midrule
Output & ${p_{\text{land}}}$
         
\\ \midrule
Input Constraints & \begin{displaymath}
                    {v_{\text{launch}}}\gt{}0
                    \end{displaymath}
                    \begin{displaymath}
                    0\lt{}θ\lt{}\frac{π}{2}
                    \end{displaymath}
\\ \midrule
Output Constraints & \begin{displaymath}
                     {p_{\text{land}}}\gt{}0
                     \end{displaymath}
\\ \midrule
Equation & \begin{displaymath}
           {p_{\text{land}}}=\frac{2 {v_{\text{launch}}}^{2} \sin\left(θ\right) \cos\left(θ\right)}{g}
           \end{displaymath}
\\ \midrule
Description & \begin{symbDescription}
              \item{${p_{\text{land}}}$ is the landing position (${\text{m}}$)}
              \item{${v_{\text{launch}}}$ is the launch speed ($\frac{\text{m}}{\text{s}}$)}
              \item{$θ$ is the launch angle (${\text{rad}}$)}
              \item{$g$ is the magnitude of gravitational acceleration ($\frac{\text{m}}{\text{s}^{2}}$)}
              \end{symbDescription}
\\ \midrule
Notes & The constraint $0\lt{}θ\lt{}\frac{π}{2}$ is from \hyperref[posXDirection]{A:posXDirection} and \hyperref[yAxisGravity]{A:yAxisGravity}, and is shown in \hyperref[Figure:Launch]{Fig:Launch}.
        
        $g$ is defined in \hyperref[gravAccelValue]{A:gravAccelValue}.
        
        The constraint ${p_{\text{land}}}\gt{}0$ is from \hyperref[posXDirection]{A:posXDirection}.
        
\\ \midrule
Source & --
         
\\ \midrule
RefBy & \hyperref[IM:offsetIM]{IM:offsetIM} and \hyperref[calcValues]{FR:Calculate-Values}
        
\\ \bottomrule
\end{tabular}
\end{minipage}

\paragraph{Detailed derivation of landing position:}
\label{IM:calOfLandingDistDeriv}
We know that ${{p_{\text{x}}}^{\text{i}}}=0$ (\hyperref[launchOrigin]{A:launchOrigin}) and ${{a_{\text{x}}}^{\text{c}}}=0$ (\hyperref[accelXZero]{A:accelXZero}). Substituting these values into the x-direction of \hyperref[GD:posVec]{GD:posVec} gives us:

\begin{displaymath}
{p_{\text{x}}}={{v_{\text{x}}}^{\text{i}}} t
\end{displaymath}
To find the landing position, we want to find the ${p_{\text{x}}}$ value (${p_{\text{land}}}$) at flight duration (from \hyperref[IM:calOfLandingTime]{IM:calOfLandingTime}):

\begin{displaymath}
{p_{\text{land}}}=\frac{{{v_{\text{x}}}^{\text{i}}}\cdot{}2 {v_{\text{launch}}} \sin\left(θ\right)}{g}
\end{displaymath}
From \hyperref[DD:speedIX]{DD:speedIX} (with ${v^{\text{i}}}={v_{\text{launch}}}$) we can replace ${{v_{\text{x}}}^{\text{i}}}$:

\begin{displaymath}
{p_{\text{land}}}=\frac{{v_{\text{launch}}} \cos\left(θ\right)\cdot{}2 {v_{\text{launch}}} \sin\left(θ\right)}{g}
\end{displaymath}
Rearranging this gives us the required equation:

\begin{displaymath}
{p_{\text{land}}}=\frac{2 {v_{\text{launch}}}^{2} \sin\left(θ\right) \cos\left(θ\right)}{g}
\end{displaymath}

\noindent \textbf{Preconditions for IM:calOfLandingDistDeriv}
\begin{itemize}
\item 
\end{itemize}

\medskip
\noindent
\begin{minipage}{\textwidth}
\begin{tabular}{>{\raggedright}p{0.13\textwidth}>{\raggedright\arraybackslash}p{0.82\textwidth}}
\toprule \textbf{Refname} & \textbf{IM:offsetIM}
\phantomsection 
\label{IM:offsetIM}
\\ \midrule
Label & Offset
        
\\ \midrule
Input & ${p_{\text{land}}}$, ${p_{\text{target}}}$
        
\\ \midrule
Output & ${d_{\text{offset}}}$
         
\\ \midrule
Input Constraints & \begin{displaymath}
                    {p_{\text{land}}}\gt{}0
                    \end{displaymath}
                    \begin{displaymath}
                    {p_{\text{target}}}\gt{}0
                    \end{displaymath}
\\ \midrule
Output Constraints & 
\\ \midrule
Equation & \begin{displaymath}
           {d_{\text{offset}}}={p_{\text{land}}}-{p_{\text{target}}}
           \end{displaymath}
\\ \midrule
Description & \begin{symbDescription}
              \item{${d_{\text{offset}}}$ is the distance between the target position and the landing position (${\text{m}}$)}
              \item{${p_{\text{land}}}$ is the landing position (${\text{m}}$)}
              \item{${p_{\text{target}}}$ is the target position (${\text{m}}$)}
              \end{symbDescription}
\\ \midrule
Notes & ${p_{\text{land}}}$ is from \hyperref[IM:calOfLandingDist]{IM:calOfLandingDist}.
        
        The constraints ${p_{\text{land}}}\gt{}0$ and ${p_{\text{target}}}\gt{}0$ are from \hyperref[posXDirection]{A:posXDirection}.
        
\\ \midrule
Source & --
         
\\ \midrule
RefBy & \hyperref[IM:messageIM]{IM:messageIM}, \hyperref[outputValues]{FR:Output-Values}, and \hyperref[calcValues]{FR:Calculate-Values}
        
\\ \bottomrule
\end{tabular}
\end{minipage}

\noindent \textbf{Preconditions for IM:offsetIM}
\begin{itemize}
\item 
\end{itemize}

\medskip
\noindent
\begin{minipage}{\textwidth}
\begin{tabular}{>{\raggedright}p{0.13\textwidth}>{\raggedright\arraybackslash}p{0.82\textwidth}}
\toprule \textbf{Refname} & \textbf{IM:messageIM}
\phantomsection 
\label{IM:messageIM}
\\ \midrule
Label & Output message
        
\\ \midrule
Input & ${d_{\text{offset}}}$, ${p_{\text{target}}}$
        
\\ \midrule
Output & $s$
         
\\ \midrule
Input Constraints & \begin{displaymath}
                    {d_{\text{offset}}}\gt{}-{p_{\text{target}}}
                    \end{displaymath}
                    \begin{displaymath}
                    {p_{\text{target}}}\gt{}0
                    \end{displaymath}
\\ \midrule
Output Constraints & 
\\ \midrule
Equation & \begin{displaymath}
           s=\begin{cases}
             \text{``The target was hit.''}, & |\frac{{d_{\text{offset}}}}{{p_{\text{target}}}}|\lt{}ε\\
             \text{``The projectile fell short.''}, & {d_{\text{offset}}}\lt{}0\\
             \text{``The projectile went long.''}, & {d_{\text{offset}}}\gt{}0
             \end{cases}
           \end{displaymath}
\\ \midrule
Description & \begin{symbDescription}
              \item{$s$ is the output message as a string (Unitless)}
              \item{${d_{\text{offset}}}$ is the distance between the target position and the landing position (${\text{m}}$)}
              \item{${p_{\text{target}}}$ is the target position (${\text{m}}$)}
              \item{$ε$ is the hit tolerance (Unitless)}
              \end{symbDescription}
\\ \midrule
Notes & ${d_{\text{offset}}}$ is from \hyperref[IM:offsetIM]{IM:offsetIM}.
        
        The constraint ${p_{\text{target}}}\gt{}0$ is from \hyperref[posXDirection]{A:posXDirection}.
        
        The constraint ${d_{\text{offset}}}\gt{}-{p_{\text{target}}}$ is from the fact that ${p_{\text{land}}}\gt{}0$, from \hyperref[posXDirection]{A:posXDirection}.
        
        $ε$ is defined in \hyperref[Sec:AuxConstants]{Sec:Values of Auxiliary Constants}.
        
\\ \midrule
Source & --
         
\\ \midrule
RefBy & \hyperref[outputValues]{FR:Output-Values} and \hyperref[calcValues]{FR:Calculate-Values}
        
\\ \bottomrule
\end{tabular}
\end{minipage}

\noindent \textbf{Preconditions for IM:messageIM}
\begin{itemize}
\item 
\end{itemize}

\subsubsection{Data Constraints}
\label{Sec:DataConstraints}
The \hyperref[Table:InDataConstraints]{Data Constraints Table} shows the data constraints on the input variables. The column for physical constraints gives the physical limitations on the range of values that can be taken by the variable. The uncertainty column provides an estimate of the confidence with which the physical quantities can be measured. This information would be part of the input if one were performing an uncertainty quantification exercise. The constraints are conservative to give the user of the model the flexibility to experiment with unusual situations. The column of typical values is intended to provide a feel for a common scenario.

\begin{longtblr}
[caption={Input Data Constraints}]
{colspec={l l l l}, rowhead=1, hline{1,Z}=\heavyrulewidth, hline{2}=\lightrulewidth}
\textbf{Var} & \textbf{Physical Constraints} & \textbf{Typical Value} & \textbf{Uncert.}
\\
${p_{\text{target}}}$ & ${p_{\text{target}}}\gt{}0$ & $1000$ ${\text{m}}$ & 10$\%$
\\
${v_{\text{launch}}}$ & ${v_{\text{launch}}}\gt{}0$ & $100$ $\frac{\text{m}}{\text{s}}$ & 10$\%$
\\
$θ$ & $0\lt{}θ\lt{}\frac{π}{2}$ & $\frac{π}{4}$ ${\text{rad}}$ & 10$\%$
\label{Table:InDataConstraints}
\end{longtblr}
\subsubsection{Properties of a Correct Solution}
\label{Sec:CorSolProps}
The \hyperref[Table:OutDataConstraints]{Data Constraints Table} shows the data constraints on the output variables. The column for physical constraints gives the physical limitations on the range of values that can be taken by the variable.

\begin{longtblr}
[caption={Output Data Constraints}]
{colspec={l l}, rowhead=1, hline{1,Z}=\heavyrulewidth, hline{2}=\lightrulewidth}
\textbf{Var} & \textbf{Physical Constraints}
\\
${p_{\text{land}}}$ & ${p_{\text{land}}}\gt{}0$
\\
${d_{\text{offset}}}$ & ${d_{\text{offset}}}\gt{}-{p_{\text{target}}}$
\\
${t_{\text{flight}}}$ & ${t_{\text{flight}}}\gt{}0$
\label{Table:OutDataConstraints}
\end{longtblr}
\section{Requirements}
\label{Sec:Requirements}
This section provides the functional requirements, the tasks and behaviours that the software is expected to complete, and the non-functional requirements, the qualities that the software is expected to exhibit.

\subsection{Functional Requirements}
\label{Sec:FRs}
This section provides the functional requirements, the tasks and behaviours that the software is expected to complete.

\begin{itemize}
\item[Input-Values:\phantomsection\label{inputValues}]{Input the values from \hyperref[Table:ReqInputs]{Tab:ReqInputs}.}
\item[Verify-Input-Values:\phantomsection\label{verifyInVals}]{Check the entered input values to ensure that they do not exceed the \hyperref[Sec:DataConstraints]{data constraints}. If any of the input values are out of bounds, an error message is displayed and the calculations stop.}
\item[Calculate-Values:\phantomsection\label{calcValues}]{Calculate the following values: ${t_{\text{flight}}}$ (from \hyperref[IM:calOfLandingTime]{IM:calOfLandingTime}), ${p_{\text{land}}}$ (from \hyperref[IM:calOfLandingDist]{IM:calOfLandingDist}), ${d_{\text{offset}}}$ (from \hyperref[IM:offsetIM]{IM:offsetIM}), and $s$ (from \hyperref[IM:messageIM]{IM:messageIM}).}
\item[Output-Values:\phantomsection\label{outputValues}]{Output ${t_{\text{flight}}}$ (from \hyperref[IM:calOfLandingTime]{IM:calOfLandingTime}), $s$ (from \hyperref[IM:messageIM]{IM:messageIM}), and ${d_{\text{offset}}}$ (from \hyperref[IM:offsetIM]{IM:offsetIM}).}
\end{itemize}
\begin{longtblr}
[caption={Required Inputs following \hyperref[inputValues]{FR:Input-Values}}]
{colspec={l l l}, rowhead=1, hline{1,Z}=\heavyrulewidth, hline{2}=\lightrulewidth}
\textbf{Symbol} & \textbf{Description} & \textbf{Units}
\\
${p_{\text{target}}}$ & Target position & ${\text{m}}$
\\
${v_{\text{launch}}}$ & Launch speed & $\frac{\text{m}}{\text{s}}$
\\
$θ$ & Launch angle & ${\text{rad}}$
\label{Table:ReqInputs}
\end{longtblr}
\subsection{Non-Functional Requirements}
\label{Sec:NFRs}
This section provides the non-functional requirements, the qualities that the software is expected to exhibit.

\begin{itemize}
\item[Correct:\phantomsection\label{correct}]{The outputs of the code have the properties described in \hyperref[Sec:CorSolProps]{Properties of a Correct Solution}.}
\item[Verifiable:\phantomsection\label{verifiable}]{The code is tested with complete verification and validation plan.}
\item[Understandable:\phantomsection\label{understandable}]{The code is modularized with complete module guide and module interface specification.}
\item[Reusable:\phantomsection\label{reusable}]{The code is modularized.}
\item[Maintainable:\phantomsection\label{maintainable}]{The traceability between requirements, assumptions, theoretical models, general definitions, data definitions, instance models, likely changes, unlikely changes, and modules is completely recorded in traceability matrices in the SRS and module guide.}
\item[Portable:\phantomsection\label{portable}]{The code is able to be run in different environments.}
\end{itemize}
\subsection{\wss{Rationale}}
\wss{Capture the rationale for the scope assumptions and final theory
assumptions.  The rationale could vary between problems.  For instance, for
projectile motion the rationale could be that it is being used for teaching
purposes.  If the theories are used to solve an actual science or engineering
problem, the rationale would need more justification.}
\wss{Neglecting rotation could be justified by assuming a point mass? \hyperref[pointMass]{A:pointMass})}.
\wss{Should requirements be added related to guaranteeing assumptions and
constraints?  (As is done after a hazard analysis.)  Requirements could be added
to check the input constraints, like $x > 0$. Requirements could be added to
check neglecting curvature.  Would need the radius of the planet, or are we
assuming it's Earth?}

\section{Traceability Matrices and Graphs}
\label{Sec:TraceMatrices}
The purpose of the traceability matrices is to provide easy references on what has to be additionally modified if a certain component is changed. Every time a component is changed, the items in the column of that component that are marked with an ``X'' should be modified as well. \hyperref[Table:TraceMatAvsA]{Tab:TraceMatAvsA} shows the dependencies of the assumptions on each other. \hyperref[Table:TraceMatAvsAll]{Tab:TraceMatAvsAll} shows the dependencies of the data definitions, theoretical models, general definitions, instance models, requirements, likely changes, and unlikely changes on the assumptions. \hyperref[Table:TraceMatRefvsRef]{Tab:TraceMatRefvsRef} shows the dependencies of the data definitions, theoretical models, general definitions, and instance models on each other. \hyperref[Table:TraceMatAllvsR]{Tab:TraceMatAllvsR} shows the dependencies of the requirements and goal statements on the data definitions, theoretical models, general definitions, and instance models.

\begin{longtblr}
[caption={Traceability Matrix Showing the Connections Between Assumptions and Other Assumptions}]
{colspec={l l l l l l l l l l l l l l l l}, rowhead=1, hline{1,Z}=\heavyrulewidth, hline{2}=\lightrulewidth}
\textbf{} & \textbf{\hyperref[twoD]{A:twoD}} & \textbf{\hyperref[cartSyst]{A:cartSyst}} & \textbf{\hyperref[yAxisGravity]{A:yAxisGravity}} & \textbf{\hyperref[launchOrigin]{A:launchOrigin}} & \textbf{\hyperref[targetXAxis]{A:targetXAxis}} & \textbf{\hyperref[posXDirection]{A:posXDirection}} & \textbf{\hyperref[constAccel]{A:constAccel}} & \textbf{\hyperref[accelXZero]{A:accelXZero}} & \textbf{\hyperref[accelYGravity]{A:accelYGravity}} & \textbf{\hyperref[neglectDrag]{A:neglectDrag}} & \textbf{\hyperref[pointMass]{A:pointMass}} & \textbf{\hyperref[freeFlight]{A:freeFlight}} & \textbf{\hyperref[neglectCurv]{A:neglectCurv}} & \textbf{\hyperref[timeStartZero]{A:timeStartZero}} & \textbf{\hyperref[gravAccelValue]{A:gravAccelValue}}
\\
\hyperref[twoD]{A:twoD} &  &  &  &  &  &  &  &  &  &  &  &  &  &  & 
\\
\hyperref[cartSyst]{A:cartSyst} &  &  &  &  &  &  &  &  &  &  &  &  & X &  & 
\\
\hyperref[yAxisGravity]{A:yAxisGravity} &  &  &  &  &  &  &  &  &  &  &  &  &  &  & 
\\
\hyperref[launchOrigin]{A:launchOrigin} &  &  &  &  &  &  &  &  &  &  &  &  &  &  & 
\\
\hyperref[targetXAxis]{A:targetXAxis} &  &  &  &  &  &  &  &  &  &  &  &  & X &  & 
\\
\hyperref[posXDirection]{A:posXDirection} &  &  &  &  &  &  &  &  &  &  &  &  &  &  & 
\\
\hyperref[constAccel]{A:constAccel} &  &  &  &  &  &  &  & X & X & X &  & X &  &  & 
\\
\hyperref[accelXZero]{A:accelXZero} &  &  &  &  &  &  &  &  &  &  &  &  &  &  & 
\\
\hyperref[accelYGravity]{A:accelYGravity} &  &  & X &  &  &  &  &  &  &  &  &  &  &  & 
\\
\hyperref[neglectDrag]{A:neglectDrag} &  &  &  &  &  &  &  &  &  &  &  &  &  &  & 
\\
\hyperref[pointMass]{A:pointMass} &  &  &  &  &  &  &  &  &  &  &  &  &  &  & 
\\
\hyperref[freeFlight]{A:freeFlight} &  &  &  &  &  &  &  &  &  &  &  &  &  &  & 
\\
\hyperref[neglectCurv]{A:neglectCurv} &  &  &  &  &  &  &  &  &  &  &  &  &  &  & 
\\
\hyperref[timeStartZero]{A:timeStartZero} &  &  &  &  &  &  &  &  &  &  &  &  &  &  & 
\\
\hyperref[gravAccelValue]{A:gravAccelValue} &  &  &  &  &  &  &  &  &  &  &  &  &  &  & 
\label{Table:TraceMatAvsA}
\end{longtblr}
\begin{longtblr}
[caption={Traceability Matrix Showing the Connections Between Assumptions and Other Items}]
{colspec={l l l l l l l l l l l l l l l l}, rowhead=1, hline{1,Z}=\heavyrulewidth, hline{2}=\lightrulewidth}
\textbf{} & \textbf{\hyperref[twoD]{A:twoD}} & \textbf{\hyperref[cartSyst]{A:cartSyst}} & \textbf{\hyperref[yAxisGravity]{A:yAxisGravity}} & \textbf{\hyperref[launchOrigin]{A:launchOrigin}} & \textbf{\hyperref[targetXAxis]{A:targetXAxis}} & \textbf{\hyperref[posXDirection]{A:posXDirection}} & \textbf{\hyperref[constAccel]{A:constAccel}} & \textbf{\hyperref[accelXZero]{A:accelXZero}} & \textbf{\hyperref[accelYGravity]{A:accelYGravity}} & \textbf{\hyperref[neglectDrag]{A:neglectDrag}} & \textbf{\hyperref[pointMass]{A:pointMass}} & \textbf{\hyperref[freeFlight]{A:freeFlight}} & \textbf{\hyperref[neglectCurv]{A:neglectCurv}} & \textbf{\hyperref[timeStartZero]{A:timeStartZero}} & \textbf{\hyperref[gravAccelValue]{A:gravAccelValue}}
\\
\hyperref[DD:vecMag]{DD:vecMag} &  &  &  &  &  &  &  &  &  &  &  &  &  &  & 
\\
\hyperref[DD:speedIX]{DD:speedIX} &  &  &  &  &  &  &  &  &  &  &  &  &  &  & 
\\
\hyperref[DD:speedIY]{DD:speedIY} &  &  &  &  &  &  &  &  &  &  &  &  &  &  & 
\\
\hyperref[TM:acceleration]{TM:acceleration} &  &  &  &  &  &  &  &  &  &  &  &  &  &  & 
\\
\hyperref[TM:velocity]{TM:velocity} &  &  &  &  &  &  &  &  &  &  &  &  &  &  & 
\\
\hyperref[GD:rectVel]{GD:rectVel} &  &  &  &  &  &  &  &  &  &  & X &  &  & X & 
\\
\hyperref[GD:rectPos]{GD:rectPos} &  &  &  &  &  &  &  &  &  &  & X &  &  & X & 
\\
\hyperref[GD:velVec]{GD:velVec} & X & X &  &  &  &  & X &  &  &  &  &  &  & X & 
\\
\hyperref[GD:posVec]{GD:posVec} & X & X &  &  &  &  & X &  &  &  &  &  &  & X & 
\\
\hyperref[IM:calOfLandingTime]{IM:calOfLandingTime} &  &  & X & X & X & X &  &  & X &  &  &  &  & X & X
\\
\hyperref[IM:calOfLandingDist]{IM:calOfLandingDist} &  &  & X & X &  & X &  & X &  &  &  &  &  &  & X
\\
\hyperref[IM:offsetIM]{IM:offsetIM} &  &  &  &  &  & X &  &  &  &  &  &  &  &  & 
\\
\hyperref[IM:messageIM]{IM:messageIM} &  &  &  &  &  & X &  &  &  &  &  &  &  &  & 
\\
\hyperref[inputValues]{FR:Input-Values} &  &  &  &  &  &  &  &  &  &  &  &  &  &  & 
\\
\hyperref[verifyInVals]{FR:Verify-Input-Values} &  &  &  &  &  &  &  &  &  &  &  &  &  &  & 
\\
\hyperref[calcValues]{FR:Calculate-Values} &  &  &  &  &  &  &  &  &  &  &  &  &  &  & 
\\
\hyperref[outputValues]{FR:Output-Values} &  &  &  &  &  &  &  &  &  &  &  &  &  &  & 
\\
\hyperref[correct]{NFR:Correct} &  &  &  &  &  &  &  &  &  &  &  &  &  &  & 
\\
\hyperref[verifiable]{NFR:Verifiable} &  &  &  &  &  &  &  &  &  &  &  &  &  &  & 
\\
\hyperref[understandable]{NFR:Understandable} &  &  &  &  &  &  &  &  &  &  &  &  &  &  & 
\\
\hyperref[reusable]{NFR:Reusable} &  &  &  &  &  &  &  &  &  &  &  &  &  &  & 
\\
\hyperref[maintainable]{NFR:Maintainable} &  &  &  &  &  &  &  &  &  &  &  &  &  &  & 
\\
\hyperref[portable]{NFR:Portable} &  &  &  &  &  &  &  &  &  &  &  &  &  &  & 
\label{Table:TraceMatAvsAll}
\end{longtblr}
\begin{longtblr}
[caption={Traceability Matrix Showing the Connections Between Items and Other Sections}]
{colspec={l l l l l l l l l l l l l l}, rowhead=1, hline{1,Z}=\heavyrulewidth, hline{2}=\lightrulewidth}
\textbf{} & \textbf{\hyperref[DD:vecMag]{DD:vecMag}} & \textbf{\hyperref[DD:speedIX]{DD:speedIX}} & \textbf{\hyperref[DD:speedIY]{DD:speedIY}} & \textbf{\hyperref[TM:acceleration]{TM:acceleration}} & \textbf{\hyperref[TM:velocity]{TM:velocity}} & \textbf{\hyperref[GD:rectVel]{GD:rectVel}} & \textbf{\hyperref[GD:rectPos]{GD:rectPos}} & \textbf{\hyperref[GD:velVec]{GD:velVec}} & \textbf{\hyperref[GD:posVec]{GD:posVec}} & \textbf{\hyperref[IM:calOfLandingTime]{IM:calOfLandingTime}} & \textbf{\hyperref[IM:calOfLandingDist]{IM:calOfLandingDist}} & \textbf{\hyperref[IM:offsetIM]{IM:offsetIM}} & \textbf{\hyperref[IM:messageIM]{IM:messageIM}}
\\
\hyperref[DD:vecMag]{DD:vecMag} &  &  &  &  &  &  &  &  &  &  &  &  & 
\\
\hyperref[DD:speedIX]{DD:speedIX} & X &  &  &  &  &  &  &  &  &  &  &  & 
\\
\hyperref[DD:speedIY]{DD:speedIY} & X &  &  &  &  &  &  &  &  &  &  &  & 
\\
\hyperref[TM:acceleration]{TM:acceleration} &  &  &  &  &  &  &  &  &  &  &  &  & 
\\
\hyperref[TM:velocity]{TM:velocity} &  &  &  &  &  &  &  &  &  &  &  &  & 
\\
\hyperref[GD:rectVel]{GD:rectVel} &  &  &  & X &  &  &  &  &  &  &  &  & 
\\
\hyperref[GD:rectPos]{GD:rectPos} &  &  &  &  & X & X &  &  &  &  &  &  & 
\\
\hyperref[GD:velVec]{GD:velVec} &  &  &  &  &  & X &  &  &  &  &  &  & 
\\
\hyperref[GD:posVec]{GD:posVec} &  &  &  &  &  &  & X &  &  &  &  &  & 
\\
\hyperref[IM:calOfLandingTime]{IM:calOfLandingTime} &  &  & X &  &  &  &  &  & X &  &  &  & 
\\
\hyperref[IM:calOfLandingDist]{IM:calOfLandingDist} &  & X &  &  &  &  &  &  & X & X &  &  & 
\\
\hyperref[IM:offsetIM]{IM:offsetIM} &  &  &  &  &  &  &  &  &  &  & X &  & 
\\
\hyperref[IM:messageIM]{IM:messageIM} &  &  &  &  &  &  &  &  &  &  &  & X & 
\label{Table:TraceMatRefvsRef}
\end{longtblr}
\begin{longtblr}
[caption={Traceability Matrix Showing the Connections Between Requirements, Goal Statements and Other Items}]
{colspec={l l l l l l l l l l l l l l l l l l l l l l l l}, rowhead=1, hline{1,Z}=\heavyrulewidth, hline{2}=\lightrulewidth}
\textbf{} & \textbf{\hyperref[DD:vecMag]{DD:vecMag}} & \textbf{\hyperref[DD:speedIX]{DD:speedIX}} & \textbf{\hyperref[DD:speedIY]{DD:speedIY}} & \textbf{\hyperref[TM:acceleration]{TM:acceleration}} & \textbf{\hyperref[TM:velocity]{TM:velocity}} & \textbf{\hyperref[GD:rectVel]{GD:rectVel}} & \textbf{\hyperref[GD:rectPos]{GD:rectPos}} & \textbf{\hyperref[GD:velVec]{GD:velVec}} & \textbf{\hyperref[GD:posVec]{GD:posVec}} & \textbf{\hyperref[IM:calOfLandingTime]{IM:calOfLandingTime}} & \textbf{\hyperref[IM:calOfLandingDist]{IM:calOfLandingDist}} & \textbf{\hyperref[IM:offsetIM]{IM:offsetIM}} & \textbf{\hyperref[IM:messageIM]{IM:messageIM}} & \textbf{\hyperref[inputValues]{FR:Input-Values}} & \textbf{\hyperref[verifyInVals]{FR:Verify-Input-Values}} & \textbf{\hyperref[calcValues]{FR:Calculate-Values}} & \textbf{\hyperref[outputValues]{FR:Output-Values}} & \textbf{\hyperref[correct]{NFR:Correct}} & \textbf{\hyperref[verifiable]{NFR:Verifiable}} & \textbf{\hyperref[understandable]{NFR:Understandable}} & \textbf{\hyperref[reusable]{NFR:Reusable}} & \textbf{\hyperref[maintainable]{NFR:Maintainable}} & \textbf{\hyperref[portable]{NFR:Portable}}
\\
\hyperref[targetHit]{GS:targetHit} &  &  &  &  &  &  &  &  &  &  &  &  &  &  &  &  &  &  &  &  &  &  & 
\\
\hyperref[inputValues]{FR:Input-Values} &  &  &  &  &  &  &  &  &  &  &  &  &  &  &  &  &  &  &  &  &  &  & 
\\
\hyperref[verifyInVals]{FR:Verify-Input-Values} &  &  &  &  &  &  &  &  &  &  &  &  &  &  &  &  &  &  &  &  &  &  & 
\\
\hyperref[calcValues]{FR:Calculate-Values} &  &  &  &  &  &  &  &  &  & X & X & X & X &  &  &  &  &  &  &  &  &  & 
\\
\hyperref[outputValues]{FR:Output-Values} &  &  &  &  &  &  &  &  &  & X &  & X & X &  &  &  &  &  &  &  &  &  & 
\\
\hyperref[correct]{NFR:Correct} &  &  &  &  &  &  &  &  &  &  &  &  &  &  &  &  &  &  &  &  &  &  & 
\\
\hyperref[verifiable]{NFR:Verifiable} &  &  &  &  &  &  &  &  &  &  &  &  &  &  &  &  &  &  &  &  &  &  & 
\\
\hyperref[understandable]{NFR:Understandable} &  &  &  &  &  &  &  &  &  &  &  &  &  &  &  &  &  &  &  &  &  &  & 
\\
\hyperref[reusable]{NFR:Reusable} &  &  &  &  &  &  &  &  &  &  &  &  &  &  &  &  &  &  &  &  &  &  & 
\\
\hyperref[maintainable]{NFR:Maintainable} &  &  &  &  &  &  &  &  &  &  &  &  &  &  &  &  &  &  &  &  &  &  & 
\\
\hyperref[portable]{NFR:Portable} &  &  &  &  &  &  &  &  &  &  &  &  &  &  &  &  &  &  &  &  &  &  & 
\label{Table:TraceMatAllvsR}
\end{longtblr}
The purpose of the traceability graphs is also to provide easy references on what has to be additionally modified if a certain component is changed. The arrows in the graphs represent dependencies. The component at the tail of an arrow is depended on by the component at the head of that arrow. Therefore, if a component is changed, the components that it points to should also be changed. \hyperref[Figure:TraceGraphAvsA]{Fig:TraceGraphAvsA} shows the dependencies of assumptions on each other. \hyperref[Figure:TraceGraphAvsAll]{Fig:TraceGraphAvsAll} shows the dependencies of data definitions, theoretical models, general definitions, instance models, requirements, likely changes, and unlikely changes on the assumptions. \hyperref[Figure:TraceGraphRefvsRef]{Fig:TraceGraphRefvsRef} shows the dependencies of data definitions, theoretical models, general definitions, and instance models on each other. \hyperref[Figure:TraceGraphAllvsR]{Fig:TraceGraphAllvsR} shows the dependencies of requirements and goal statements on the data definitions, theoretical models, general definitions, and instance models. \hyperref[Figure:TraceGraphAllvsAll]{Fig:TraceGraphAllvsAll} shows the dependencies of dependencies of assumptions, models, definitions, requirements, goals, and changes with each other. 

\begin{figure}
\begin{center}
\includesvg[width=\textwidth, inkscapelatex = false]{projectile/avsa}
\caption{TraceGraphAvsA}
\label{Figure:TraceGraphAvsA}
\end{center}
\end{figure}
\begin{figure}
\begin{center}
\includesvg[width=\textwidth, inkscapelatex = false]{projectile/avsall}
\caption{TraceGraphAvsAll}
\label{Figure:TraceGraphAvsAll}
\end{center}
\end{figure}
\begin{figure}
\begin{center}
\includesvg[width=\textwidth, inkscapelatex = false]{projectile/refvsref}
\caption{TraceGraphRefvsRef}
\label{Figure:TraceGraphRefvsRef}
\end{center}
\end{figure}
\begin{figure}
\begin{center}
\includesvg[width=\textwidth, inkscapelatex = false]{projectile/allvsr}
\caption{TraceGraphAllvsR}
\label{Figure:TraceGraphAllvsR}
\end{center}
\end{figure}
\begin{figure}
\begin{center}
\includesvg[width=\textwidth, inkscapelatex = false]{projectile/allvsall}
\caption{TraceGraphAllvsAll}
\label{Figure:TraceGraphAllvsAll}
\end{center}
\end{figure}
For convenience, the following graphs can be found at the links below:

\begin{itemize}
\item{\hyperref{projectile/avsa.svg}{}{}{TraceGraphAvsA}}
\item{\hyperref{projectile/avsall.svg}{}{}{TraceGraphAvsAll}}
\item{\hyperref{projectile/refvsref.svg}{}{}{TraceGraphRefvsRef}}
\item{\hyperref{projectile/allvsr.svg}{}{}{TraceGraphAllvsR}}
\item{\hyperref{projectile/allvsall.svg}{}{}{TraceGraphAllvsAll}}
\end{itemize}
\section{Values of Auxiliary Constants}
\label{Sec:AuxConstants}
This section contains the standard values that are used for calculations in Projectile.

\begin{longtblr}
[caption={Auxiliary Constants}]
{colspec={l l l l}, rowhead=1, hline{1,Z}=\heavyrulewidth, hline{2}=\lightrulewidth}
\textbf{Symbol} & \textbf{Description} & \textbf{Value} & \textbf{Unit}
\\
$g$ & magnitude of gravitational acceleration & $9.8$ & $\frac{\text{m}}{\text{s}^{2}}$
\\
$ε$ & hit tolerance & $2.0\%$ & --
\\
$π$ & ratio of circumference to diameter for any circle & $3.14159265$ & --
\label{Table:TAuxConsts}
\end{longtblr}
\section{References}
\label{Sec:References}
\begin{filecontents*}{bibfile.bib}
@book{hibbeler2004,
author={Hibbeler, R. C.},
title={Engineering Mechanics: Dynamics},
publisher={Pearson Prentice Hall},
year={2004}}
@mastersthesis{koothoor2013,
author={Koothoor, Nirmitha},
title={A Document Driven Approach to Certifying Scientific Computing Software},
school={McMaster University},
year={2013},
address={Hamilton, ON, Canada}}
@article{parnasClements1986,
author={Parnas, David L. and Clements, P. C.},
title={A rational design process: How and why to fake it},
journal={IEEE Transactions on Software Engineering},
year={1986},
month=feb,
volume={12},
number={2},
pages={251--257},
address={Washington, USA}}
@article{smithKoothoor2016,
author={Smith, W. Spencer and Koothoor, Nirmitha},
title={A Document-Driven Method for Certifying Scientific Computing Software for Use in Nuclear Safety Analysis},
journal={ Nuclear Engineering and Technology},
year={2016},
month=apr,
volume={48},
number={2},
pages={404--418},
howpublished={\url{http://www.sciencedirect.com/science/article/pii/S1738573315002582}}}
@inproceedings{smithLai2005,
author={Smith, W. Spencer and Lai, Lei},
title={A new requirements template for scientific computing},
booktitle={Proceedings of the First International Workshop on Situational Requirements Engineering Processes - Methods, Techniques and Tools to Support Situation-Specific Requirements Engineering Processes, SREP'05},
year={2005},
editor={Agerfalk, PJ and Kraiem, N. and Ralyte, J.},
address={Paris, France},
pages={107--121},
note={In conjunction with 13th IEEE International Requirements Engineering Conference,}}
@article{smithEtAl2007,
author={Smith, W. Spencer and Lai, Lei and Khedri, Ridha},
title={Requirements Analysis for Engineering Computation: A Systematic Approach for Improving Software Reliability},
journal={Reliable Computing, Special Issue on Reliable Engineering Computation},
year={2007},
month=feb,
volume={13},
number={1},
pages={83--107},
howpublished={\url{https://doi.org/10.1007/s11155-006-9020-7}}}
@misc{accelerationWiki,
author={Wikipedia Contributors},
title={Acceleration},
howpublished={\url{https://en.wikipedia.org/wiki/Acceleration}},
month=jun,
year={2019}}
@misc{cartesianWiki,
author={Wikipedia Contributors},
title={Cartesian coordinate system},
howpublished={\url{https://en.wikipedia.org/wiki/Cartesian\_coordinate\_system}},
month=jun,
year={2019}}
@misc{velocityWiki,
author={Wikipedia Contributors},
title={Velocity},
howpublished={\url{https://en.wikipedia.org/wiki/Velocity}},
month=jun,
year={2019}}
\end{filecontents*}
\nocite{*}
\bibstyle{ieeetr}
\printbibliography[heading=none]
\end{document}

TODO

- make assumptions mathematical
- rotation of the Earth assumption
- type for position in 1D, 2D and 3D?
- draw fault tree diagram?
- draw relationship between parts in a graph
- rules - every assumption is invoked and listed, inherited assumptions maintained, or justified or renamed
- table of symbols for final theories?
- data definition - body becomes projectile?