\documentclass{article}
\renewcommand{\baselinestretch}{1.5} 

\usepackage{amsmath, mathtools}
\usepackage{amsfonts}
\usepackage{amssymb}
\usepackage{fullpage}

\title{Questions on Refining Theories for Projectile}
\author{Spencer Smith}

\begin{document}

\maketitle

\section{Context Theories}

\noindent CT:realArith

\noindent CT:functions

\noindent CT:nDimSpace(n)

\noindent CT:trigonometry

\noindent CT:vectors

\noindent CT:CartCoordSyst

\noindent CT:Differentiation

\noindent CT:Integration

\section{Proposed Sketch of New Approach}

Does the following informal idea of what we want to do make sense?  If not, how
do we change it?  Once we have a reasonable sequence of theory refinements, how
do we write it down in a rigorous/formal way?

\begin{itemize}
\item P = a theory of a position function in 1D space.  Given a time, return the
position.  Build using the context theories of CT:nDimSpace(1), CT:functions.
\item V = refine P by differentiating with respect to time. Uses context
theories from P and adds CT:Differentiation.
\item A = refine V by differentiating with respect to time.  Uses context theories of V.
\item GD:rectVel = refine A by integration using A:timeStartZero and
A:constAccel. $v(t) = v^i + a^c t$. Adds CT:Integration.
\item GD:rectPos = refine V by integration using A:timeStartZero and A:constAccel. $p(t) = p^i + v^i t + a^c t^2/2$.  Adds context theory CT:realArith, CT:Integration.
\item GD:velVec = refine GD:rectVel using CT:nDimSpace(2), CT:CartCoordSyst,
CT:vectors and the independence of two coordinate directions. The result is a 2D
vector using the equation from GD:rectVel twice with a different initial
velocity and constant acceleration in each direction.
\item PT:posVecInitMagAndAngle = refine GD:velVec using the angle and magnitude
representation of the initial velocity vector, rather than the component-wise
representation.  Uses CT:trigonometry.
\item PT:velVecPlanetaryGrav = refine PT:posVecInitMagAndAngle using an
acceleration of 0 in the $x$ direction and an acceleration of $-g$ in the $y$
direction.
\end{itemize}

\end{document}