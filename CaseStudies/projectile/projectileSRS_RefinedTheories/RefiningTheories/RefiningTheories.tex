\documentclass{article}
\renewcommand{\baselinestretch}{1.5} 

\usepackage{amsmath, mathtools}
\usepackage{amsfonts}
\usepackage{amssymb}
\usepackage{fullpage}

\title{Refining Theories for Projectile}
\author{Spencer Smith}

\begin{document}

\maketitle

\section{Context Theories}

\noindent CT:realArith

\noindent CT:functions

\noindent CT:nDimSpace(n)

\noindent CT:trigonometry

\noindent CT:vectors

\noindent CT:CartCoordSyst

\noindent CT:Differentiation

\noindent CT:Integration

\section{Assumptions}

The fields for the assumptions are the text description, the relevant
mathematical relation and the rationale (intention?).\\

\noindent A:oneD = (``The motion of the body is one dimensional.'', $v_2(t) =
v_3(t) = 0$, ``The body can be modelled as moving in a straight line.'')

\noindent A:constAccel = (``The acceleration is constant'', $\frac{d a}{dt} =
0$, ``The body undergoes constant acceleration, like when a body is in free fall
with no external force acting on it, or a charged particle in a constant
electric field.'')

\noindent A:timeStartZero = (``Time starts at zero.'', $t = 0$, ``The time that
the modelling starts is an arbitrary decision, so the choice is made to start at
zero to simplify the equations'')

\section{Theories and Theory Refinement}

\subsection{Theory for TM:acceleration}

\noindent TM:acceleration = ($a(t) = \frac{d}{dt} v(t)\\$, \{CT:realArith,
CT:vectors, CT:CartCoodSyst, CT:Differentiation\}, \{MD:cartSyst, A:threeD\})

\noindent TM:acceleration = ($
    \left [ 
    \begin{array}{c}
    a_1(t)\\
    a_2(t)\\
    a_3(t) 
    \end{array} 
    \right ] =
    \frac{d}{dt}
    \left [ 
    \begin{array}{c}
    v_1(t)\\
    v_2(t)\\
    v_3(t) 
    \end{array} 
    \right ]
    $, \{CT:realArith, CT:vectors, CT:CartCoodSyst, CT:Differentiation\}, \{MD:cartSyst, A:threeD\})
    
\subsection{Refinement for GD:rectVel}

TM:acceleration = ($
    \left [ 
    \begin{array}{c}
    a_1(t)\\
    a_2(t)\\
    a_3(t) 
    \end{array} 
    \right ] =
    \frac{d}{dt}
    \left [ 
    \begin{array}{c}
    v_1(t)\\
    v_2(t)\\
    v_3(t) 
    \end{array} 
    \right ]
    $, \{CT:realArith, ...\}, \{MD:cartSyst, A:threeD\})

\noindent Replace A:threeD by A:oneD.  $v_2(t) = v_3(t) = 0$ therefore $a_2(t) = a_3(t) = 0$.

TM:acceleration' = ($a_1(t) = \frac{d}{dt} v_1(t)$, \{CT:realArith,
...\}, \{MD:cartSyst, A:oneD\})

\noindent Apply the assumption A:constAccel.

TM:acceleration'' = ($a_1 = \frac{d}{dt} v_1(t)$, \{CT:realArith,
...\}, \{MD:cartSyst, A:oneD, A:constAccel\})

\noindent Relabel $a_1$ as $a^c$, the constant acceleration.

TM:acceleration''' = ($a^c = \frac{d}{dt} v_1(t)$, \{CT:realArith,
...\}, \{MD:cartSyst, A:oneD, A:constAccel\})

\noindent Relabel $v_1(t)$ as $v(t)$.

TM:acceleration'''' = ($a^c = \frac{d}{dt} v(t)$, \{CT:realArith,
...\}, \{MD:cartSyst, A:oneD, A:constAccel\})

\noindent Assume that at $t = 0$ (A:timeStartZero) the velocity $v(0)$ is $v^i$
and integrate using CT:integration.

GD:rectVel = ($v(t) = v^i + a^c t$, \{CT:realArith, ...,
CT:integration\}, \{MD:cartSyst, A:oneD, A:constAccel, A:timeStartZero\})

\section{Projectile Project}

\noindent Project = (StrategyP, \{CT, BT, HT, GT, PT, FT, RT\})\\

\noindent CT = (\{CT:realArith, CT:trigonometry, CT:vectors, CT:CartCoordSyst,
CT:Differentiation, CT:Integration\})\\

\noindent BT = (StrategyBT, \{TM:acceleration, TM:velocity,
TM:directionCosines\})\\

\noindent HT = (StrategyHT, \{GD:rectVel, GD:rectPos\})\\

\noindent GT = (StrategyGT, \{GD:velVec, GD:posVec, GD:magAngleToCompRep\})\\

\noindent PT = (StrategyPT, \{PT:coordSyst, DD:speedIX, DD:speedIY,
PT:velVecInitMagAndAngle, PT:posVecInitMagAndAngle, PT:posVecInitPos,
PT:velVecPlanetaryGrav, PT:posVecPlanetaryGrav\})\\

\noindent FT = (StrategyFT, \{IM:calOfLandingTime, IM:calOfLandingDist,
IM:offsetIM, IM:messageIM\})\\

\noindent RT = (StrategyRT, \{RT:lngDstErr\})\\

\end{document}