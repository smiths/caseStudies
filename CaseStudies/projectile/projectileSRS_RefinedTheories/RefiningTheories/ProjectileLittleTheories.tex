\documentclass{article}
\renewcommand{\baselinestretch}{1.5} 

\usepackage{amsmath, mathtools}
\usepackage{amsfonts}
\usepackage{amssymb}

\usepackage{fullpage}

\usepackage{hyperref}

\title{Little Theories for Projectile}
\author{Spencer Smith}

\begin{document}

\maketitle

The inspiration for the following ``little theories'' version of projectile
motion is Farmer's little theories formalization in simple type theory.  An
example of this can be found in the paper
``\href{https://imps.mcmaster.ca/doc/monoids.pdf} {Monoid Theory in Alonzo}'' by
Farmer and Zvigelsky (2023).

The intention of this document is to start a discussion of the best way to
envision projectile motion using the little theories approach.  This is an
informal document where the notation is not rigorous.  For instance, given that
the author doesn't understand exactly how one theory is transported to another,
this issue is glossed over by adding a field called ``Transport'' to informally
communicate this notion.

\section{1D Kinematic Theory}

\noindent \textbf{Name:} Kin1D

\noindent \textbf{Extends:} Theory of real numbers with functions and differentiation

\noindent \textbf{Constants:} $p_{\mathbb{R} \rightarrow \mathbb{R}}$ (position)

\noindent \textbf{Axioms:}

\begin{enumerate}
    \item $\forall c: \mathbb{R} \; . \; (\forall \epsilon: \mathbb{R} |
    \epsilon > 0 \; . \; (\exists \delta : \mathbb{R} | \delta > 0 \; . \;
    (\forall x: \mathbb{R} \; . \; | x - c | < \delta \rightarrow |p \; x - p \;
    c | < \epsilon)))$ ($p$ is continuous over all of $\mathbb{R}$)
    \item $\forall x: \mathbb{R} \; . \; \frac{dp} {dt} \; \text{exists}$ (first derivative of $p$ exists over all of $\mathbb{R}$)
    \item $\forall x: \mathbb{R} \; . \; \frac{d^2 p} {d t^2} \; \text{exists}$ (second derivative of $p$ exists over all of $\mathbb{R}$)
\end{enumerate}

\noindent \textbf{Definitions and theorems:}

Def1: $v_{\mathbb{R} \rightarrow \mathbb{R}} = \frac{dp}{dt}$ (velocity)

Def2: $a_{\mathbb{R} \rightarrow \mathbb{R}} = \frac{dv}{dt}$ (acceleration)

\section{1D Kinematic Constant Acceleration Theory}

\noindent \textbf{Name:} Kin1DConstAccel

\noindent \textbf{Transport:} Kin1D

\noindent \textbf{Constants:} $t_{\mathbb{R}}$ (time), ${a^c}_{\mathbb{R}}$
(constant accel.), ${p^i}_{\mathbb{R}}$ (initial pos.), ${v^i}_{\mathbb{R}}$ (initial velo.)

\noindent \textbf{Axioms:}

\begin{enumerate}
    \item $t \geq 0$ (A:timeStartZero)
    \item $p \; 0 = p^i$ (initial position)
    \item $v \; 0 = v^i$ (initial velocity)
    \item $\forall t: \mathbb{R} | t \geq 0 \; . \; a \; t = a^c$ (A:constAccel)
\end{enumerate}

\noindent \textbf{Definitions and theorems:}

Thm1: $v \; t = v^i + a^c t$

Thm2: $p \; t = p^i + v^i t + \frac{a^c t^2}{2}$


\section{nD Kinematic Constant Acceleration Theory}

\noindent \textbf{Name:} KinnDConstAccel (\textit{This section is not complete;
it is currently a mix of nD and 2D.})

\noindent \textbf{Transport:} Kin1DConstAccel, $n$ Dimensional Euclidean space.  Somehow $p$, $v$ and $a$
have to be mapped to both $x$ and $y$ components of each.

\noindent \textbf{Constants:} ${a_x^c}_{\mathbb{R}}$ (constant accel.\ $x$
direction), ${a_y^c}_{\mathbb{R}}$ (constant accel.\ $y$ direction),
${p_x^i}_{\mathbb{R}}$ (initial pos.\ $x$ direction), ${p_y^i}_{\mathbb{R}}$
(initial pos.\ $y$ direction),${v_x^i}_{\mathbb{R}}$ (initial velo.\ $x$
direction), ${v_y^i}_{\mathbb{R}}$ (initial velo.\ $y$ direction) 

\noindent \textbf{Axioms:}

\begin{enumerate}
    \item $\forall k: \mathbb{N} | 0 \leq k \leq (n-1) \; . \; p_k \; 0 = p_k^i$ (initial position vector)
    \item $\forall k: \mathbb{N} | 0 \leq k \leq (n-1) \; . \; v_k \; 0 = v_k^i$ (initial velocity vector)
    \item $\forall k: \mathbb{N} | 0 \leq k \leq (n-1) \; . \; (\forall t:
    \mathbb{R} | t \geq 0 \; . \; a_k \; t = a_k^c)$ (constant acceleration vector)
\end{enumerate}

\noindent \textbf{Definitions and theorems:}

Thm1: \begin{displaymath}
    \mathbf{v}\text{(}t\text{)}=\begin{bmatrix}
        v_\text{x}\\
        v_\text{y}
        \end{bmatrix} =\begin{bmatrix}
                               {{v_{\text{x}}}^{\text{i}}}+{{a_{\text{x}}}^{\text{c}}} t\\
                               {{v_{\text{y}}}^{\text{i}}}+{{a_{\text{y}}}^{\text{c}}} t
                               \end{bmatrix}
    \end{displaymath}

Thm2: \begin{displaymath}
    \mathbf{p}\text{(}t\text{)}=\begin{bmatrix}
        p_\text{x}\\
        p_\text{y}
        \end{bmatrix} = \begin{bmatrix}
                               {{p_{\text{x}}}^{\text{i}}}+{{v_{\text{x}}}^{\text{i}}} t+\frac{{{a_{\text{x}}}^{\text{c}}} t^{2}}{2}\\
                               {{p_{\text{y}}}^{\text{i}}}+{{v_{\text{y}}}^{\text{i}}} t+\frac{{{a_{\text{y}}}^{\text{c}}} t^{2}}{2}
                               \end{bmatrix}
    \end{displaymath}

\section{Next Steps}

\begin{enumerate}
    \item Projectile motion specific via $p_x = 0$, $p_y = 0$, $v_x = v^{\text{launch}} \cos \theta$, $v_y = v^{\text{launch}} \sin \theta$, $a_x^c = 0$, $a_y^c = -g$
    \item Add constraint axioms $v^{\text{launch}} > 0$, $0 < \theta < \frac{\pi} {2}$
    \item Add theorem for calculation of landing time
    \item Add theorem for calculation of landing distance
\end{enumerate}

\section{Questions}

\begin{itemize}
    \item How should units be handled?  For instance, position has units of
    length (m), velocity has units of length per unit time (m/s) and
    acceleration has units of length$^2$ per unit time (m$^2$/2).
    \item How to capture context information?  The typical values for
    projectiles used in games are much different than the values for projectiles
    in ballistics.
    \item How to document that context-specific rationale information?  For
    instance, the assumption that the Earth is flat leads to only a tiny error
    for sports-related projectiles.
\end{itemize}

\end{document}