\documentclass[12pt, titlepage]{article}

\usepackage{tabularx}
\usepackage{longtable}
\usepackage{comment}
\usepackage{amsmath}
\usepackage{amssymb}
\usepackage{booktabs}
\usepackage{xr}
\usepackage{siunitx}
\usepackage{caption}
\usepackage{graphicx}
\usepackage{adjustbox}
\usepackage{hyperref}
\usepackage[numbib,nottoc]{tocbibind}
\hypersetup{
    colorlinks,
    citecolor=green,
    filecolor=black,
    linkcolor=red,
    urlcolor=blue
}
\usepackage[round]{natbib}
\usepackage{enumitem}

\usepackage{xr}
\externaldocument[SRS-]{../SRS/glassbr_srs}
\newcommand{\rref}[1]{R\ref{#1}}

\newcounter{testnum} %Assumption Number
\newcommand{\tcthetestnum}{TC\thetestnum}
\newcommand{\tcref}[1]{TC\ref{#1}}

\input{../Comments}

\newcommand{\progname}{GlassBR}

\begin{document}

\title{Glass Breakage Analysis: System Verification and Validation Plan} 
\author{Vajiheh Motamer}
\date{\today}
	
\maketitle

\pagenumbering{roman}

\section{Revision History}

\begin{tabularx}{\textwidth}{p{3cm}p{2cm}X}
\toprule {\bf Date} & {\bf Version} & {\bf Notes}\\
\midrule
10/26/18 & 1.0 & Initial Draft\\

\bottomrule
\end{tabularx}

~\newpage

\section{Symbols, Abbreviations and Acronyms}
The symbols, abbreviations, and acronyms used in this document include those 
defined in the table below, as well as any defined in the tables found in 
Section \ref{SRS-sec_abbrev} of the Software Requirements Specification (SRS) 
document.
\newline

\renewcommand{\arraystretch}{1.2}
\begin{tabular}{l l} 
  \toprule		
  \textbf{symbol} & \textbf{description}\\
  \midrule 
    TC & Test Case\\
  VnV & Verification and Validation\\
  \bottomrule
\end{tabular}\\

\newpage

\tableofcontents

\listoftables

\newpage

\pagenumbering{arabic}

\noindent This document presents the system verification and validation plan 
for the software. General information regarding the system under test and the 
objectives of the verification and validation activities is provided in Section 
\ref{sec_GenInfo}. Overviews of verification plans for the SRS, design, and 
implementation are given in Section \ref{sec_Plan}, along with a summary of the 
validation plan for the software. Section \ref{sec_System} details specific  
system test cases for verifying the requirements outlined in Section 
\ref{SRS-sec_Reqs} of the SRS. A summary of planned static verification 
activities can be found in Section \ref{sec_Static} of this document.

\section{General Information} \label{sec_GenInfo}

\subsection{Summary}

\noindent The software being tested is the Glass Breakage Analysis Program 
(\progname{}). Based on user-defined glass type and blast properties,
\progname{} interprets the inputs to give out the outputs which
predict whether the glass slab can withstand the blast under the given conditions. The blast under
consideration is a type of  blast load. Software is helpful to efficiently and correctly
predict the blast risk involved with the glass slab using an intuitive interface.

\subsection{Objectives}

\noindent The purpose of the verification and validation activities is to 
confirm that \progname{} exhibits desired software qualities. The primary 
objective is to build confidence in the correctness of the software. The tests 
described in this document cannot definitively prove correctness, but they can 
build confidence by verifying that the software is correct for the cases 
covered by tests. Other important qualities to be verified are the 
portability and reusability of the software.

\subsection{References}

\noindent Extensive information about the purpose and requirements of 
\progname{} can be found in the SRS document. This System VnV Plan is 
complemented by the System VnV Report, where the results of the tests planned 
in this document are discussed. For more details on test cases specific to the 
implementation of \progname{}, consult the Unit VnV Plan document. The latest 
documentation for \progname{} can be found on GitHub, at \newline 
\href{https://github.com/smiths/caseStudies/tree/master/CaseStudies/glass}{https://github.com/smiths/caseStudies/tree/master/CaseStudies/glass}.

\section{Plan} \label{sec_Plan}
	
\subsection{Verification and Validation Team}

\noindent Responsible member for the verification and validation 
of \progname{} is Vajiheh Motamer

\subsection{SRS Verification Plan}

\noindent The SRS for the project will be reviewed by Dr. Smith and some of the class mates in CAS741.
So, feedback will be provided. Some SRS feedback for this project have been
provided and addressed using github issue tracker. Besides, during preparing of all steps SRS will be reviewed and if would be required, the changes would be made.

\subsection{Design Verification Plan}

\noindent The design of \progname{} s  documents will be verified by getting feedback from Dr.
Spencer Smith and my CAS 741 classmates.The Module Guide and Module Interface Specification will
contain information about the software design. Feedback is expected to be
provided by reviewers via github issue tracker.

\subsection{Implementation Verification Plan} \label{sec_ImpPlan}

The implementation of the\progname{} program will be verified  by performing statically code review with Dr. Spencer Smith and the classmates in CAS741 and
dynamically by executing the test cases detailed in this plan and the unit
VnV plan using Pyton testing frameworks.


\subsection{Software Validation Plan}

There is no validation plan for \progname{}.

\section{System Test Description} \label{sec_System}

System testing for the \progname{} ensures that the correct inputs produce
the correct outputs. The test cases in this section are derived from the
instance models and the requirements detailed in the tool’s SRS. These values were 
taken from the' Test Documentation' for this project . Individual 
test cases will reference the table as input but specify new values for any 
input parameter that should have a different value than specified by the table.



	
\subsection{Tests for Functional Requirements}

\subsubsection{User Input Tests}
		
\paragraph{Valid User Input}

~\newline \noindent The following set of test cases is intended to cover 
different forms of valid user input. 



\begin{enumerate}[label=TC\arabic*:,ref={\arabic*}]
%BeginTestCase1
\item [TC\refstepcounter{testnum}\thetestnum: \label{TC_defultInput}] 
Tst\_Pb\_DefaultValues

Control: Automatic
					
Initial State: New session
					
Input: As described in Table~\ref{defaultInputTBL}.
					
Output: \\
$\hspace*{2cm} P_b= (1.301524590203718e-04) < P_{b_{\text{tol}}}$ ,\\
$\hspace*{2cm} Demand (q) = (3.258285992018616e+00) ,\\
$\hspace*{2cm} $Capacity (LR)=(6.843002155788037e+00) > q$, \\
$\hspace*{2cm}$\text{$\text{is\_safePb} = \text{True}$ and $\text{is\_safeLR} = \text{True}$, The glass is considered safe.}


How test will be performed: Unit testing using PyUnit.

Test Case Derivation:  \ref{SRS-IM_prob} and \ref{SRS-T_Pb} in SRS.


\begin{table}[!h]
\centering

\renewcommand{\arraystretch}{1.2}
\begin{tabular}{ | p{3cm} | p{3cm}| p{3cm} | }  
\toprule
\textbf{Input} & \textbf{Value} & \textbf{Unit}\\
\midrule 
		$\text{a}$ &1600 & \text{m} \\
		$\text{b}$ &1500 & \text{m}\\
		$\text{g}$ &\text{HS} & \text{-}\\
		$P_{b_{\text{tol}}}$ &0.008& \text{-}\\
		$\text{SD}_x$ & 0 &  \si{\meter}\\
		$\text{SD}_y$ &1.5 & \si{\metre}\\
		$\text{SD}_z$ & 11.0 &\si{\metre}\\
		$\text{t}$ &10.0 & \text{mm}\\
		$\text{TNT}$ &1.0 & \text{-}\\
$w$ &10.0	& \si{\kilo\gram}\\
		\bottomrule
\end{tabular}
\caption{Inputs for Tst\_Pb\_DefaultValues} 
\label{defaultInputTBL}
\end{table}

%EndTestCase1
\newpage
%BeginTestCase2				
\item [TC\refstepcounter{testnum}\thetestnum: \label{TC_SmallDimensionInput}] 
Tst\_ Pb\_ SmallDimensionValues

Control: Automatic
					
Initial State: New session
					
Input: As described in Table~\ref{SmallDimensionTBL}.
					
Output: \\
$\hspace*{2cm} P_b= (1.824662149424894e-03) < P_{b_{\text{tol}}}$,\\
$\hspace*{2cm} Demand (q) = (3.658003449421614e+00) ,\\
$\hspace*{2cm} $Capacity (LR)=(4.916016610996773e+00) > q$, \\
$\hspace*{2cm}$\text{$\text{is\_safePb} = \text{True}$ and $\text{is\_safeLR} = \text{True}$, The glass is considered safe.}



How test will be performed: Unit testing using PyUnit.

Test Case Derivation:  \ref{SRS-IM_prob} and \ref{SRS-T_Pb} in SRS.



\begin{table}[!h]
\centering

\renewcommand{\arraystretch}{1.2}
\begin{tabular}{ | p{3cm} | p{3cm}| p{3cm} | }  
\toprule
\textbf{Input} & \textbf{Value} & \textbf{Unit}\\
\midrule 
		$\text{a}$ &1200 & \text{m} \\
		$\text{b}$ &1000 & \text{m}\\
		$\text{g}$ &\text{AN} & \text{-}\\
		$P_{b_{\text{tol}}}$ &0.010& \text{-}\\
		$\text{SD}_x$ & 0 &  \si{\meter}\\
		$\text{SD}_y$ &2.0& \si{\metre}\\
		$\text{SD}_z$ & 10.0 &\si{\metre}\\
		$\text{t}$ &8.0 & \text{mm}\\
		$\text{TNT}$ &1.0 & \text{-}\\
$w$ &10.0	& \si{\kilo\gram}\\
		\bottomrule
\end{tabular}
\caption{Inputs for Tst\_ Pb\_ SmallDimensionValues} 
\label{SmallDimensionTBL}
\end{table}


%EndTestCase2
\newpage
%BeginTestCase3

\item [TC\refstepcounter{testnum}\thetestnum: \label{TC_LargeDimensionInput}] 
Tst\_Pb\_LargeDimensionValues

Control: Automatic
					
Initial State: New session
					
Input: As described in Table~\ref{LargeDimensionTBL}.
					
Output: \\
$\hspace*{2cm} P_b= (3.459068155453604e-04) < P_{b_{\text{tol}}}$,\\
$\hspace*{2cm} Demand (q) = (5.777809021268771e+00) ,\\
$\hspace*{2cm} $Capacity (LR)=(1.092974208994522e+01) > q$ ,\\
$\hspace*{2cm}$\text{$\text{is\_safePb} = \text{True}$ and $\text{is\_safeLR} = \text{True}$, The glass is considered safe.}




How test will be performed: Unit testing using PyUnit.

Test Case Derivation:  \ref{SRS-IM_prob} and \ref{SRS-T_Pb} in SRS.


\begin{table}[!h]
\centering

\renewcommand{\arraystretch}{1.2}
\begin{tabular}{ | p{3cm} | p{3cm}| p{3cm} | }  
\toprule
\textbf{Input} & \textbf{Value} & \textbf{Unit}\\
\midrule 
		$\text{a}$ &1600 & \text{m} \\
		$\text{b}$ &1500 & \text{m}\\
		$\text{g}$ &\text{HS} & \text{-}\\
		$P_{b_{\text{tol}}}$ &0.010& \text{-}\\
		$\text{SD}_x$ & 0 &  \si{\meter}\\
		$\text{SD}_y$ &1.5& \si{\metre}\\
		$\text{SD}_z$ & 11.0 &\si{\metre}\\
		$\text{t}$ &10.0 & \text{mm}\\
		$\text{TNT}$ &1.0 & \text{-}\\
$w$ &10.0	& \si{\kilo\gram}\\
		\bottomrule
\end{tabular}
\caption{Inputs for Tst\_ Pb\_ LargeDimensionValues} 
\label{LargeDimensionTBL}
\end{table}
%EndTestCase3
\newpage
%BeginTestCase4
\item [TC\refstepcounter{testnum}\thetestnum: \label{TC_LowPbTol}] 
Tst\_ Pb\_ LowPbTol

Control: Automatic
					
Initial State: New session
					
Input: As described in Table~\ref{LowPbTolTBL}.
					
Output: \\
$\hspace*{2cm} P_b= (1.301524590203718e-04) > P_{b_{\text{tol}}}$,\\
$\hspace*{2cm} Demand (q) = (3.258285992018616e+00) ,\\
$\hspace*{2cm} $Capacity (LR)=(3.124424950223241e+00) \ngtr q$ ,\\
$\hspace*{2cm}$\text{$\text{is\_safePb} = \text{False}$ and $\text{is\_safeLR} = \text{False}$, The glass is NOT considered safe.}



How test will be performed: Unit testing using PyUnit.

Test Case Derivation:  \ref{SRS-IM_prob} and \ref{SRS-T_Pb} in SRS.


\begin{table}[!h]
\centering

\renewcommand{\arraystretch}{1.2}
\begin{tabular}{ | p{3cm} | p{3cm}| p{3cm} | }  
\toprule
\textbf{Input} & \textbf{Value} & \textbf{Unit}\\
\midrule 
		$\text{a}$ &1600 & \text{m} \\
		$\text{b}$ &1500 & \text{m}\\
		$\text{g}$ &\text{HS} & \text{-}\\
		$P_{b_{\text{tol}}}$ &0.008& \text{-}\\
		$\text{SD}_x$ & 0 &  \si{\meter}\\
		$\text{SD}_y$ &2.5& \si{\metre}\\
		$\text{SD}_z$ & 6.0 &\si{\metre}\\
		$\text{t}$ &10.0 & \text{mm}\\
		$\text{TNT}$ &1.0 & \text{-}\\
$w$ &10.0	& \si{\kilo\gram}\\
		\bottomrule
\end{tabular}
\caption{Inputs for Tst\_ Pb\_ LowPbTol} 
\label{LowPbTolTBL}
\end{table}

%EndTestCase4
\newpage
%BeginTestCase5
\item [TC\refstepcounter{testnum}\thetestnum: \label{TC_DiffSDInput}] 
Tst\_ Pb\_ DiffSDValues

Control: Automatic
					
Initial State: New session
					
Input: As described in Table~\ref{DiffSDTBL}.
					
Output: \\
$\hspace*{2cm} P_b= (1.185574651484522e-02) >  P_{b_{\text{tol}}}$, \\
$\hspace*{2cm} Demand (q) = (7.377747177423622e+00) ,\\
$\hspace*{2cm} $Capacity (LR)=(6.843002155788037e+00) \ngtr q$ ,\\
$\hspace*{2cm}$\text{$\text{is\_safePb} = \text{False}$ and $\text{is\_safeLR} = \text{False}$, The glass is NOT considered safe.}


How test will be performed: Unit testing using PyUnit.

Test Case Derivation:  \ref{SRS-IM_prob} and \ref{SRS-T_Pb} in SRS.


\begin{table}[!h]
\centering

\renewcommand{\arraystretch}{1.2}
\begin{tabular}{ | p{3cm} | p{3cm}| p{3cm} | }  
\toprule
\textbf{Input} & \textbf{Value} & \textbf{Unit}\\
\midrule 
		$\text{a}$ &1600 & \text{m} \\
		$\text{b}$ &1500 & \text{m}\\
		$\text{g}$ &\text{HS} & \text{-}\\
		$P_{b_{\text{tol}}}$ &0.008& \text{-}\\
		$\text{SD}_x$ & 0.0 &  \si{\meter}\\
		$\text{SD}_y$ &2.5& \si{\metre}\\
		$\text{SD}_z$ & 6.0 &\si{\metre}\\
		$\text{t}$ &0.008 & \text{mm}\\
		$\text{TNT}$ &1.0 & \text{-}\\
$w$ &10.0	& \si{\kilo\gram}\\
		\bottomrule
\end{tabular}
\caption{Inputs for Tst\_ Pb\_ DiffSDValues} 
\label{DiffSDTBL}
\end{table}

%EndTestCase5
\newpage
%BeginTestCase6

\item [TC\refstepcounter{testnum}\thetestnum: \label{TC_HighChgWght}] 
Tst\_ Pb\_ HighChgWght

Control: Automatic
					
Initial State: New session
					
Input: As described in Table~\ref{HighChgWghtTBL}.
					
Output: \\
$\hspace*{2cm} P_b= (2.497577817262034e-01) >  P_{b_{\text{tol}}}$, \\
$\hspace*{2cm} Demand (q) = (1.428204355548630e+01) ,\\
$\hspace*{2cm} $Capacity (LR)=(6.843002155788037e+00) \ngtr q$, \\
$\hspace*{2cm}$\text{$\text{is\_safePb} = \text{False}$ and $\text{is\_safeLR} = \text{False}$, The glass is NOT considered safe.}


How test will be performed: Unit testing using PyUnit.

Test Case Derivation:  \ref{SRS-IM_prob} and \ref{SRS-T_Pb} in SRS.



\begin{table}[!h]
\centering

\renewcommand{\arraystretch}{1.2}
\begin{tabular}{ | p{3cm} | p{3cm}| p{3cm} | }  
\toprule
\textbf{Input} & \textbf{Value} & \textbf{Unit}\\
\midrule 
		$\text{a}$ &1600 & \text{m} \\
		$\text{b}$ &1500 & \text{m}\\
		$\text{g}$ &\text{HS} & \text{-}\\
		$P_{b_{\text{tol}}}$ &0.008& \text{-}\\
		$\text{SD}_x$ & 0 &  \si{\meter}\\
		$\text{SD}_y$ &1.5& \si{\metre}\\
		$\text{SD}_z$ & 11.0 &\si{\metre}\\
		$\text{t}$ &10.0 & \text{mm}\\
		$\text{TNT}$ &1.0 & \text{-}\\
$w$ &60.0	& \si{\kilo\gram}\\
		\bottomrule
\end{tabular}
\caption{Inputs for Tst\_ Pb\_ HighChgWght} 
\label{HighChgWghtTBL}
\end{table}

%EndTestCase6
\newpage
%BeginTestCase7

\item [TC\refstepcounter{testnum}\thetestnum: \label{TC_LowThicknessInput}] 
Tst\_ Pb\_ LowThickness

Control: Automatic
					
Initial State: New session
					
Input: As described in Table~\ref{LowThicknessTBL}.
					
Output: \\
$\hspace*{2cm} P_b= (2.528418262282350e-01) >  P_{b_{\text{tol}}}$, \\
$\hspace*{2cm} Demand (q) = (3.258285992018616e+00) ,\\
$\hspace*{2cm} $Capacity (LR)=(1.716982174845693e+00) \ngtr q$, \\
$\hspace*{2cm}$\text{$\text{is\_safePb} = \text{False}$ and $\text{is\_safeLR} = \text{False}$, The glass is NOT considered safe.}


How test will be performed: Unit testing using PyUnit.

Test Case Derivation:  \ref{SRS-IM_prob} and \ref{SRS-T_Pb} in SRS.



\begin{table}[!h]
\centering

\renewcommand{\arraystretch}{1.2}
\begin{tabular}{ | p{3cm} | p{3cm}| p{3cm} | }  
\toprule
\textbf{Input} & \textbf{Value} & \textbf{Unit}\\
\midrule 
		$\text{a}$ &1600 & \text{m} \\
		$\text{b}$ &1500 & \text{m}\\
		$\text{g}$ &\text{HS} & \text{-}\\
		$P_{b_{\text{tol}}}$ &0.008& \text{-}\\
		$\text{SD}_x$ & 0 &  \si{\meter}\\
		$\text{SD}_y$ &1.5& \si{\metre}\\
		$\text{SD}_z$ & 11.0 &\si{\metre}\\
		$\text{t}$ &3.0 & \text{mm}\\
		$\text{TNT}$ &1.0 & \text{-}\\
$w$ &10.0	& \si{\kilo\gram}\\
		\bottomrule
\end{tabular}
\caption{Inputs for Tst\_ Pb\_ LowThickness} 
\label{LowThicknessTBL}
\end{table}

\end{enumerate}
%EndTestCase7
\paragraph{Invalid User Input}

~\newline \noindent The test cases described in Table~\ref{testCheckConstraints} 
are intended to cover all invalid input possibilities. Invalid input is input 
that defies the data constraints described in Section 
\ref{SRS-sec_DataConstraints} of the SRS. These test cases are identical to 
each other with the exception of their input. The input for each is specified 
in Table~\ref{testCheckConstraints}. For each test case, the inputs which have not been specified 
in this table, have been specified in Table~\ref{defaultInputTBL}. Besides,The expected output has been specified for each test case and the control method 
for these test cases is automatic. The initial state for each is a new session. 
The tests will be performed as automated tests on the PyUnit.

%Table : testCheckConstraints
\begin{table}[h!]
\centering
\caption{TestCheckConstraints}
\label{testCheckConstraints}
\begin{adjustbox}{max width=\textwidth}
\begin{tabular}{*{4}{|c|}}
\hline
\textbf{Test Case} & \textbf{Test Name} & \textbf{Significant Input} & \textbf{Expected Output} \\
\hline
\hline
TC\refstepcounter{testnum}\thetestnum \label{TC_checkAPositiveTest}  & checkAPositiveTest  & a = -1600 & InputError: a and b must be greater than 0 
\\
TC\refstepcounter{testnum}\thetestnum \label{TC_checkBPositiveTest} & checkBPositiveTest &  b = -1500 & InputError: a and b must be greater than 0 
\\
TC\refstepcounter{testnum}\thetestnum \label{TC_checkSmallAspectRTest} & checkSmallAspectRTest & b = 2000 & (a/b=0.8\textless1); InputError: a/b must be between 1 and 5 
\\
TC\refstepcounter{testnum}\thetestnum \label{TC_checkLargeAspectRTest} & checkLargeAspectRTest & b = 200 & (a/b=8\textgreater5); InputError: a/b must be between 1 and 5 
\\
TC\refstepcounter{testnum}\thetestnum \label{TC_checkValidThicknessTest} & checkValidThicknessTest & t = 7 & InputError: t must be in {[}2.5,2.7,3.0,4.0,5.0,6.0,8.0, 10.0,12.0,16.0,19.0,22.0{]}  
\\
TC\refstepcounter{testnum}\thetestnum \label{TC_checkLowerConstrOnWTest} & checkLowerConstrOnWTest  & w = 3 & InputError: wtnt must be between 4.5 and 910 
\\
TC\refstepcounter{testnum}\thetestnum \label{TC_ checkUpperConstrOnWTest} & checkUpperConstrOnWTest & w = 1000 & InputError: wtnt must be between 4.5 and 910 
\\
TC\refstepcounter{testnum}\thetestnum \label{TC_checkTNTPositiveTest} & checkTNTPositiveTest & tnt = -2 & InputError: TNT must be greater than 0 
\\
TC\refstepcounter{testnum}\thetestnum \label{TC_checkLowerConstrOnSDTest } & checkLowerConstrOnSDTest & sdx = 0; sdy = 1.0; sdz = 2.0 & InputError: SD must be between 6 and 130 
\\
TC\refstepcounter{testnum}\thetestnum \label{TC_checkUpperConstrOnSDTest} & checkUpperConstrOnSDTest &  sdx = 0; sdy = 200; sdz = 100 & InputError: SD must be between 6 and 130 
\\
TC\refstepcounter{testnum}\thetestnum \label{TC_ incorrectA0Test } & incorrectA0Test & a = 0 & InputError: a and b must be greater than 0 
\\
TC\refstepcounter{testnum}\thetestnum \label{TC_ incorrectB0Test} & incorrectB0Test  & b = 0 & InputError: a and b must be greater than 0 
\\
TC\refstepcounter{testnum}\thetestnum \label{TC_ incorrectTNT0Test} & incorrectTNT0Test & tnt = 0 & InputError: TNT must be greater than 0 
\\
TC\refstepcounter{testnum}\thetestnum \label{TC_incorrectAspectREqLwrBndTest } & incorrectAspectREqLwrBndTest & a = 1500; b = 1500 & (a/b = 1);  "Encountered an unexpected exception" 
\\
TC\refstepcounter{testnum}\thetestnum \label{TC_ incorrectAspectREqUpprBndTest} & incorrectAspectREqUpprBndTest & a = 7500; b = 1500 & (a/b = 5); "Encountered an unexpected exception" 
\\
TC\refstepcounter{testnum}\thetestnum \label{TC_ incorrectWEqLwrBndTest } & incorrectWEqLwrBndTest & w = 4.5 & "Encountered an unexpected exception" 
\\
TC\refstepcounter{testnum}\thetestnum \label{TC_ incorrectWEqUpprBndTest} & incorrectWEqUpprBndTest & w = 910 & "Encountered an unexpected exception" 
\\
TC\refstepcounter{testnum}\thetestnum \label{TC_ incorrectSDEqLwrBndTest} & incorrectSDEqLwrBndTest & sdx = 0; sdy = 6; sdz = 0 & "Encountered an unexpected exception" 
\\
TC\refstepcounter{testnum}\thetestnum \label{TC_ incorrectWEqUpprBndTest } & incorrectWEqUpprBndTest  & sdx = 130; sdy = 0; sdz = 0 & "Encountered an unexpected exception" 
\\
\hline
\end{tabular}
\end{adjustbox}
\end{table}


\subsubsection{Output Tests}

\paragraph{Test Derived Values}
~\newline \noindent The following set of test cases are intended to 
verify initial inputs have been correctly converted into derived quantities. These test cases follow term definitions and equations from Data Definitions in Section \ref{SRS-sec_datadef} of the SRS.. For each test case, one input  column references to the specified input table. Besides,The expected output has been specified for each test case and the control method for these test cases is automatic. The initial state for each is a new session. The tests will be performed as automated tests on the PyUnit.


%Table 3: testDerivedValues
\begin{table}[h!]
\centering
\caption{TestDerivedValues}
\label{testDerivedValues}
\begin{adjustbox}{max width=\textwidth}
\begin{tabular}{*{9}{|c|}}
\hline
\textbf{Test Case} & \textbf{Test Name} &  \textbf{Input Table} & \textbf{AR Expected}  & \textbf{SD Expected} & \textbf{LDF Expected} &\textbf{wTNT Expected}& \textbf{h Expected} & \textbf{GTF Expected}\\
\hline
\hline
TC\refstepcounter{testnum}\thetestnum \label{TstDrvdValsHSGlTy}  &  Table \ref{defaultInputTBL} & 1.0666666666666667 & 11.10180165558726 & 0.2696493494752911 & 10.0 & 9.02 & 2
\\ 
TC\refstepcounter{testnum}\thetestnum \label{TstDrvdValsANGlTy} & Table \ref{SmallDimensionTBL}  & 1.2 & 10.198039027185569 & 0.2696493494752911 & 10.0 & 7.42 & 1
\\
TC\refstepcounter{testnum}\thetestnum \label{TstDrvdValsFTGlTy} & Table \ref{LargeDimensionTBL} & 1.25 & 9.093404203047394 & 0.2696493494752911 & 15.0 & 9.02 & 4
\\  
\hline             

\end{tabular}
\end{adjustbox}
\end{table}



\subsection{Tests for Nonfunctional Requirements} \label{sec_NFRTests}

\subsubsection{Portability test}

\begin{enumerate}[label=TC\arabic*:,ref={\arabic*}]
	
\item [TC\refstepcounter{testnum}\thetestnum: \label{TC_Portability}] 
test-portability

Type: Manual
					
Initial State:There is a completed implementation of \progname{}
					
Input/Condition: -
					
Output/Result:-

					
How test will be performed: Running \progname{} on Mac, Windows and
Linux operating systems
					
\end{enumerate}

\subsubsection{Reusability test}

\begin{enumerate}[label=TC\arabic*:,ref={\arabic*}]
	
\item [TC\refstepcounter{testnum}\thetestnum: \label{TC_Reusability}] 
test-reusability

Type: Manual

Initial State: There is a completed implementation of \progname{}
	
Input/Condition: -

Output/Result: An alternative version of \progname{} that uses the input code. 
All modules in the alternative version should be identical to the existing 
\progname{} modules, with the exception of the Input Module.

How test will be performed:  If only the Input Module is changed from the 
existing version, then the test passes and confidence in the reusability of 
\progname{}'s modules is increased. If other modules need to be changed, the 
test fails and the other modules must be modified to be completely reusable.
	
\end{enumerate}

\subsection{Traceability Between Test Cases and Requirements}

\noindent The purpose of the traceability matrix shown in 
Table~\ref{Table:T_trace} is to provide easy 
references on which requirements are verified by which test cases, and which 
test cases need to be updated if a requirement changes.  If a requirement is 
changed, the items in the column of that requirement that are marked
with an ``X'' may have to be modified as well. 

\begin{table}[!h]
	\centering
	\begin{tabular}{|c|c|c|c|c|c|c|}
		\hline
		& \rref{SRS-Input}& \rref{SRS-KnownValues}& \rref{SRS-Verify}& \rref{SRS-R_OutputInput}& 
		\rref{SRS-R_ Comparison}& \rref{SRS-R_Output}\\
		\hline
		\tcref{TC_defultInput} - \tcref{TC_LowThicknessInput}                 
		& X& & & & &  \\ \hline
		\tcref{TC_defultInput} - \tcref{TC_LowThicknessInput}                 
		& &X & & & &  \\ \hline
		\tcref{TC_checkAPositiveTest} - \tcref{TC_ incorrectWEqUpprBndTest} 
		& & & X& & & \\ \hline
		\tcref{TC_defultInput} - \tcref{TC_LowThicknessInput}                             
		& & & &X& &   \\ \hline
		\tcref{TC_defultInput} - \tcref{TC_LowThicknessInput}                              
		& & & & &X &   \\ \hline
		\tcref{TstDrvdValsHSGlTy} - \tcref{TstDrvdValsFTGlTy}                                                
		& & & & & &X  \\ \hline
		
		\hline
	\end{tabular}
	\caption{Traceability Matrix Showing the Connections Between Requirements 
	and Test Cases}
	\label{Table:T_trace}
\end{table}

\section{Static Verification Techniques} \label{sec_Static}

Static verification of the \progname{} library implementation will performed using
code review with Dr. Spencer Smith and my CAS 741 classmates.
				
\newpage

\bibliographystyle {plainnat}
%\bibliography {../../../refs/References}

%\newpage

%\section{Appendix}
%
%This is where you can place additional information.
%
%\subsection{Symbolic Parameters}
%
%The definition of the test cases will call for SYMBOLIC\_CONSTANTS.
%Their values are defined in this section for easy maintenance.
%
%\subsection{Usability Survey Questions?}
%
%\wss{This is a section that would be appropriate for some projects.}

\end{document}