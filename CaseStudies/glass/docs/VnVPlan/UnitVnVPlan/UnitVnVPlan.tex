\documentclass[12pt]{article}
\usepackage{multirow}
\usepackage{geometry}
\usepackage{caption}
\usepackage{bm}
\usepackage{amsmath}
\usepackage{amsfonts}
\usepackage{amssymb}
\usepackage{graphicx}
\usepackage{colortbl}
\usepackage{adjustbox}
\usepackage{xr}
\usepackage{hyperref}
\usepackage[all]{hypcap} 
\usepackage{longtable}
\usepackage{xfrac}
\usepackage{tabularx}
\usepackage{float}
\usepackage{siunitx}
\usepackage{booktabs}
\usepackage[toc, page]{appendix}
\usepackage{url}
\usepackage[usenames,dvipsnames]{xcolor}
\usepackage{array}
\usepackage{tabu}
\usepackage[numbib,nottoc]{tocbibind}
%\usepackage{refcheck}

\hypersetup{
	colorlinks=true,       % false: boxed links; true: coloured links
	linkcolor=red,          % colour of internal links (change box colour with
	% %linkbordercolor)
	citecolor=blue,        % colour of links to bibliography
	filecolor=magenta,      % colour of file links
	urlcolor=cyan           % colour of external links
}

\input{../../Comments}
\externaldocument[SRS-]{../../SRS/glassbr_srs}

\externaldocument[MG-]{../../Design/MG/MG}
\externaldocument[MIS-]{../../Design/MIS/glassbr_mis}
\externaldocument[SVnV-]{../SystVnVPlan/SystVnVPlan}


\newcommand{\colZwidth}{1.0\textwidth}
\newcommand{\blt}{- } %used for bullets in a list
\newcommand{\colAwidth}{0.13\textwidth}
\newcommand{\colBwidth}{0.82\textwidth}
\newcommand{\colCwidth}{0.1\textwidth}
\newcommand{\colDwidth}{0.05\textwidth}
\newcommand{\colEwidth}{0.8\textwidth}
\newcommand{\colFwidth}{0.17\textwidth}
\newcommand{\colGwidth}{0.5\textwidth}
\newcommand{\colHwidth}{0.28\textwidth}
\newcounter{defnum} %Definition Number
\newcommand{\dthedefnum}{GD\thedefnum}
\newcommand{\dref}[1]{GD\ref{#1}}
\newcounter{datadefnum} %Datadefinition Number
\newcommand{\ddthedatadefnum}{DD\thedatadefnum}
\newcommand{\ddref}[1]{DD\ref{#1}}
\newcounter{theorynum} %Theory Number
\newcommand{\tthetheorynum}{T\thetheorynum}
\newcommand{\tref}[1]{T\ref{#1}}
\newcounter{tablenum} %Table Number
\newcommand{\tbthetablenum}{T\thetablenum}
\newcommand{\tbref}[1]{TB\ref{#1}}
\newcounter{assumpnum} %Assumption Number
\newcommand{\atheassumpnum}{P\theassumpnum}
\newcommand{\aref}[1]{A\ref{#1}}
\newcounter{physsysnum} %Physical System Description Number
\newcommand{\pthephyssysnum}{P\thephyssysnum}
\newcommand{\psref}[1]{PS\ref{#1}}
\newcounter{goalnum} %Goal Number
\newcommand{\gthegoalnum}{P\thegoalnum}
\newcommand{\gsref}[1]{GS\ref{#1}}
\newcounter{instnum} %Instance Number
\newcommand{\itheinstnum}{IM\theinstnum}
\newcommand{\iref}[1]{IM\ref{#1}}
\newcounter{reqnum} %Requirement Number
\newcommand{\rthereqnum}{P\thereqnum}
\newcommand{\rref}[1]{R\ref{#1}}
\newcounter{lafnum}
\newcommand{\lthelafnum}{L\thelafnum}
\newcommand{\lref}[1]{L\laf{#1}}
\newcounter{uahnum}
\newcommand{\utheuahnum}{U\theuahnum}
\newcommand{\uref}[1]{U\uaf{#1}}
\newcounter{perfnum}%Appendix
\newcommand{\ptheperfnum}{AP\theperfnum}
\newcommand{\perf}[1]{AP\perf{#1}}
\newcounter{oaenum}
\newcommand{\otheoaenum}{O\theoaenum}
\newcommand{\oae}[1]{P\oae{#1}}
\newcounter{masnum}
\newcommand{\mthemasnum}{M\themasnum}
\newcommand{\mas}[1]{P\mas{#1}}
\newcounter{secunum}
\newcommand{\sthesecunum}{S\thesecunum}
\newcommand{\secu}[1]{S\secu{#1}}
\newcounter{culnum}
\newcommand{\ctheculnum}{C\theculnum}
\newcommand{\cul}[1]{P\cul{#1}}
\newcounter{apnum}
\newcommand{\atheapnum}{L\theapnum}
\newcommand{\apref}[1]{AP\ap{#1}}

\newcounter{lcnum} %Likely change number
\newcommand{\lthelcnum}{LC\thelcnum}
\newcommand{\lcref}[1]{LC\ref{#1}}

\newcounter{ucnum} %Unlikely change number
\newcommand{\utheucnum}{UC\theucnum}
\newcommand{\ucref}[1]{UC\ref{#1}}

\newcommand{\tclad}{T_\text{CL}}
\newcommand{\degree}{\ensuremath{^\circ}}
\newcommand{\progname}{GlassBR}
\newcommand{\euler}{e}
\newcommand{\complex}{i}

\newcolumntype{P}[1]{>{\centering\arraybackslash}p{#1}}

%\oddsidemargin 0mm
%\evensidemargin 0mm
%\textwidth 160mm
%\textheight 200mm
%\usepackage{fullpage}
\newgeometry{margin=2cm}

\begin{filecontents}{../../../refs/References.bib}
\end{filecontents}


\begin{document}

\title{Project Title: Unit Verification and Validation Plan for \progname{}} 
\author{Vajiheh Motamer}
\date{\today}
	
\maketitle

\pagenumbering{roman}

\section{Revision History}

\begin{tabularx}{\textwidth}{p{3cm}p{2cm}X}
\toprule {\bf Date} & {\bf Version} & {\bf Notes}\\
\midrule
27/11/18 & 1.0 & initial UnitVnVPlan based on new template\\

\bottomrule
\end{tabularx}

~\newpage

\tableofcontents

\listoftables

\wss{Do not include if not relevant}

\listoffigures

\wss{Do not include if not relevant}

\newpage

\section{Symbols, Abbreviations and Acronyms}

\renewcommand{\arraystretch}{1.2}
\begin{tabular}{l l} 
  \toprule		
  \textbf{symbol} & \textbf{description}\\
  \midrule 
  T & Test\\
  \bottomrule
\end{tabular}\\

\wss{symbols, abbreviations or acronyms -- you can reference the SRS, MG or MIS
  tables if needed}

\newpage

\pagenumbering{arabic}

This document ... \wss{provide an introductory blurb and roadmap of the
  unit V\&V plan}

\section{General Information}

\subsection{Purpose}

\wss{Identify software that is being unit tested (verified).}

\subsection{Scope}
All introduced modules in MG are inside of the scope to UnitVnVPlan



\section{Plan}
	
\subsection{Verification and Validation Team}
Responsible member for the verification and validation of GlassBR is Vajiheh Motamer.


\subsection{Automated Testing and Verification Tools}
According to that all test files are going to develop based on Python. The following tools have been considered:
\begin{itemize}
	\item Unit Testing Tools :
	\begin{itemize}
		\item pytest:  no API. Automatic collection of tests; simple asserts; strong support for test fixture/state management via setup/teardown hooks; strong debugging support via customized traceback. In additional, it is considered as tests runner which Selectivly run tests; Stop on first failure 
		\item unittest : Strong support for test organization and reuse via test suites 
		 \end{itemize}
	 \end{itemize}


\subsection{Non-Testing Based Verification}
This section for \progname is not applicable.

\section{Unit Test Description}

\wss{Reference your MIS and explain your overall philosophy for test case
  selection.}

\subsection{Tests for Functional Requirements}

\wss{Most of the verification will be through automated unit testing.  If
  appropriate specific modules can be verified by a non-testing based
  technique.  That can also be documented in this section.}

\subsubsection{Control Module}
With reference to Section \ref{MIS-Main} from MIS, this section determines if the main program produces the correct output.

\begin{enumerate}

\item{TstMain\_1}

Type: Automotic
					
Initial State: New Session
					
Input: defaultInput.txt, output.txt\\ from the following path: https://github.com/smiths/caseStudies/tree/master
	/CaseStudies/glass/src/Python/Test/Inputfiles

					
Output: assertEqual(GenOutput.txt, output.txt)
\aida{Should I use sOut (Exported Constant in MIS) instead of GenOutput.txt? Besides, Should I use "GenOutput.txt, output.txt" for expected output instead of assert?}

Test Case Derivation: \iref{SRS-IM_prob} and \tref{SRS-T_Pb} in SRS.

How test will be performed: Unit testing using PyUnit.
					
\item{TstMain\_2\\}

Type: Automotic

Initial State: New Session

Input: TestInput1.txt, output1.txt\\ https://github.com/smiths/caseStudies/tree/master
/CaseStudies/glass/src/Python/Test/Inputfiles


Output: assertEqual(GenOutput1.txt, output1.txt)
\aida{Should I consider only a test case which consists of a number of inputfile and outputfile instead of seperate test cases with same body and different input and output? }

Test Case Derivation: \iref{SRS-IM_prob} and \tref{SRS-T_Pb} in SRS.

How test will be performed: Unit testing using PyUnit.


    
\end{enumerate}

\begin{enumerate}

\subsubsection{Input Module}

\item{TstCheckConstraints} \\
Invalid input is input that defies the data constraints described
in Section~\ref{SRS-sec_datadef} of the SRS. This test case has considered to test verify params routin from MIS \\
Type: Automotic

Initial State: New Session

Input: Table~\ref{SVnV-testCheckConstraints}  and Table~\ref{SVnV-defaultInputTBL} from SystVnVPlan


Output: assertEqual(("Specified Error in the Table~\ref{SVnV-testCheckConstraints} "), "Generated Error by GlassBR")
\aida{What should I consider instead of "generated error by glassbr"?}

Test Case Derivation: R1 and R2 from SRS.

How test will be performed: Unit testing using PyUnit.

\item{testInputFormat} \\
The following set of test cases is intended to ensure data is being read in from the input file correctly and it has considered to test of load\_params(s) routin from MIS

Type: Automotic

Initial State: New Session

Input: defaultInput.txt,testInput1.txt,testInput2.txt \\ https://github.com/smiths/caseStudies/tree/master
/CaseStudies/glass/src/Python/Test/Inputfiles


Output: assertEqual(Params as maual( for example for length 1200), readed param from the input file)
\aida{Do you have  better suggestion for the format of arguments of assertEqual?}

Test Case Derivation: R1 and R2 from SRS.

How test will be performed: Unit testing using PyUnit.

\item{testDerivedValues} \\
The following set of test cases is intended to ensure value from the derived quantities has been calculated correctly.

Type: Automotic

Initial State: New Session

Input: defaultInput.txt,testInput1.txt,testInput2.txt \\ https://github.com/smiths/caseStudies/tree/master
/CaseStudies/glass/src/Python/Test/Inputfiles


Output: assertEqual(Params as manual( for example for sdExpected 11.10180165558726), readed sd from the input file)
\aida{Should I have a sperate table for input and outputs similar to SystVnVPlan??}

Test Case Derivation: R2 from SRS.

How test will be performed: Unit testing using PyUnit.




\end{enumerate}
\begin{enumerate}
	
	\subsubsection{Input Module}
	
	\item{TstCheckConstraints} \\
	Invalid input is input that defies the data constraints described
	in Section~\ref{SRS-sec_datadef} of the SRS. This test case has considered to test verify params routin from MIS \\
	Type: Automotic
	
	Initial State: New Session
	
	Input: Table~\ref{SVnV-testCheckConstraints}  and Table~\ref{SVnV-defaultInputTBL} from SystVnVPlan
	
	
	Output: assertEqual(("Specified Error in the Table~\ref{SVnV-testCheckConstraints} "), "Generated Error by GlassBR")
	\aida{What should I consider instead of "generated error by glassbr"?}
	
	Test Case Derivation: R1 and R2 from SRS.
	
	How test will be performed: Unit testing using PyUnit.
	
	\item{testInputFormat} \\
	The following set of test cases is intended to ensure data is being read in from the input file correctly and it has considered to test of load\_params(s) routin from MIS
	
	Type: Automotic
	
	Initial State: New Session
	
	Input: defaultInput.txt,testInput1.txt,testInput2.txt \\ https://github.com/smiths/caseStudies/tree/master
	/CaseStudies/glass/src/Python/Test/Inputfiles
	
	
	Output: assertEqual(Params as maual( for example for length 1200), readed param from the input file)
	\aida{Do you have  better suggestion for the format of arguments of assertEqual?}
	
	Test Case Derivation: R1 and R2 from SRS.
	
	How test will be performed: Unit testing using PyUnit.
	
	\item{testDerivedValues} \\
	The following set of test cases is intended to ensure value from the derived quantities has been calculated correctly.
	
	Type: Automotic
	
	Initial State: New Session
	
	Input: defaultInput.txt,testInput1.txt,testInput2.txt \\ https://github.com/smiths/caseStudies/tree/master
	/CaseStudies/glass/src/Python/Test/Inputfiles
	
	
	Output: assertEqual(Params as manual( for example for sdExpected 11.10180165558726), readed sd from the input file)
	\aida{Should I have a sperate table for input and outputs similar to SystVnVPlan??}
	
	Test Case Derivation: R2 from SRS.
	
	How test will be performed: Unit testing using PyUnit.
	
\aida{I think this test case covers GlassTypeADT Module and ThicknessADT Module and Constants Module. Do you agree?}	
	
	
\end{enumerate}
\begin{enumerate}
	
	\subsubsection{Calc Module}
	
	\item{testCalculations} \\
	These set of tests are same to  TC1 to TC7 in SystVnVPlan.

	
	\aida{Is that enough to reference to SystVnVplan?}	
	
	
\end{enumerate}

\subsection{Tests for Nonfunctional Requirements}

\wss{If there is a module that needs to be independently assessed for
  performance, those test cases can go here.  In some projects, planning for
  nonfunctional tests of units will not be that relevant.}

\wss{These tests may involve collecting performance data from previously
  mentioned functional tests.}

\subsubsection{Module ?}
		
\begin{enumerate}

\item{test-id1\\}

Type: \wss{Functional, Dynamic, Manual, Automatic, Static etc. Most will
  be automatic}
					
Initial State: 
					
Input/Condition: 
					
Output/Result: 
					
How test will be performed: 
					
\item{test-id2\\}

Type: Functional, Dynamic, Manual, Static etc.
					
Initial State: 
					
Input: 
					
Output: 
					
How test will be performed: 

\end{enumerate}

\subsubsection{Module ?}

...

\subsection{Traceability Between Test Cases and Modules}

\wss{Provide evidence that all of the modules have been considered.}

\bibliographystyle{plainnat}

\bibliography{SRS}

\newpage

\section{Appendix}

\wss{This is where you can place additional information, as appropriate}

\subsection{Symbolic Parameters}

\wss{The definition of the test cases may call for SYMBOLIC\_CONSTANTS.
Their values are defined in this section for easy maintenance.}

\end{document}