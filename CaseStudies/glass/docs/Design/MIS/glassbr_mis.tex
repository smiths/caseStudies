\documentclass[12pt, titlepage]{article}

\usepackage{amsmath, mathtools}

\usepackage[round]{natbib}
\usepackage{amsfonts}
\usepackage{amssymb}
\usepackage{graphicx}
\usepackage{colortbl}
\usepackage{xr}
\usepackage{hyperref}
\usepackage{longtable}
\usepackage{xfrac}
\usepackage{tabularx}
\usepackage{float}
\usepackage{siunitx}
\usepackage{booktabs}
\usepackage{multirow}
\usepackage[section]{placeins}
\usepackage{caption}
\usepackage{fullpage}

\hypersetup{
bookmarks=true,     % show bookmarks bar?
colorlinks=true,       % false: boxed links; true: colored links
linkcolor=red,          % color of internal links (change box color with linkbordercolor)
citecolor=blue,      % color of links to bibliography
filecolor=magenta,  % color of file links
urlcolor=cyan          % color of external links
}

\usepackage{array}

%% Comments
\newif\ifcomments\commentstrue

\ifcomments
\newcommand{\authornote}[3]{\textcolor{#1}{[#3 ---#2]}}
\newcommand{\todo}[1]{\textcolor{red}{[TODO: #1]}}
\else
\newcommand{\authornote}[3]{}
\newcommand{\todo}[1]{}
\fi

\newcommand{\wss}[1]{\authornote{blue}{SS}{#1}}

\newcommand{\progname}[1]{Glass-BR}

\externaldocument[SRS-]{../../SRS/glassbr_srs}
\externaldocument[MG-]{../MG/glassbr_mg}

\begin{document}

\title{Module Interface Specification for \progname} 
\author{Spencer Smith and Jingwei Huang}
\date{\today}

\maketitle

\tableofcontents

\newpage

\section{Introduction}

The following document details the Module Interface Specifications for the
implemented modules in a program \progname{}. It is intended to ease navigation
through the program for design and maintenance purposes.  Complementary
documents include the \href{../SRS/glassbr_srs.pdf}{System Requirement
  Specifications} (SRS) and \href{../MG/glassbr_mg.pdf}{Module Guide} (MG).  The
full documentation and implementation can be found at
\url{https://github.com/smiths/caseStudies/tree/master/CaseStudies/glass}.

\section{Notation}

The structure of the MIS for modules comes from \citet{HoffmanAndStrooper1995},
with the addition that template modules have been adapted from
\cite{GhezziEtAl2003}.  The mathematical notation comes from Chapter 3 of
\citet{HoffmanAndStrooper1995}.  For instance, the symbol := is used for a
multiple assignment statement and conditional rules follow the form $(c_1
\Rightarrow r_1 | c_2 \Rightarrow r_2 | ... | c_n \Rightarrow r_n )$.

The following table summarizes the primitive data types used by \progname. 

\begin{center}
\renewcommand{\arraystretch}{1.2}
\noindent 
\begin{tabular}{l l p{7.5cm}} 
\toprule 
\textbf{Data Type} & \textbf{Notation} & \textbf{Description}\\ 
\midrule
character & char & a single symbol or digit\\
integer & $\mathbb{Z}$ & a number without a fractional component in (-$\infty$, $\infty$) \\
natural number & $\mathbb{N}$ & a number without a fractional component in [1, $\infty$) \\
real & $\mathbb{R}$ & any number in (-$\infty$, $\infty$)\\
\bottomrule
\end{tabular} 
\end{center}

\noindent
The specification of \progname \ uses some derived data types: sequences, strings, and
tuples. Sequences are lists filled with elements of the same data type. Strings
are sequences of characters. Tuples contain a list of values, potentially of
different types. In addition, \progname \ uses functions, which
are defined by the data types of their inputs and outputs. Local functions are
described by giving their type signature followed by their specification.

\section{Module Hierarchy} 

To view the Module Hierarchy, see section 3 of the \href{../MG/glassbr_mg.pdf}{MG}.

\newpage

%%%%%%%%%%%%%%%%%%%%%%%%%%%%%%%%%%%%%%%%%%%

\section{MIS of InputT Module} \label{InputT}

The secrets of this module are the data structure for input parameters, how the
values are input and how the values are verified.  The load and verify secrets
are isolated to their own access programs.

\subsection{Module}

Param

\subsection{Uses}

SpecParam (Section~\ref{SpecParam})

\subsection{Syntax}

\begin{tabular}{p{3cm} p{1cm} p{1cm} >{\raggedright\arraybackslash}p{9cm}}
\toprule
\textbf{Name} & \textbf{In} & \textbf{Out} & \textbf{Exceptions} \\
\midrule
load\_params & string & - &  FileError \\
verify\_params & - & - & ValueError \\
$a$ & - & $\mathbb{R}$\\
$b$ & - & $\mathbb{R}$\\
$g$ & - & GlassTypeT\\
$P_{b_\text{tol}}$ & - & $\mathbb{R}$\\
$\text{SD}_x$ & - & $\mathbb{R}$\\
$\text{SD}_y$ & - & $\mathbb{R}$\\
$\text{SD}_z$ & - & $\mathbb{R}$\\
$t$ & - & ThicknessT\\
$w$ & - & $\mathbb{R}$\\
$t_d$ & - & $\mathbb{R}$\\
LDF & - & $\mathbb{R}$\\
LSF & - & $\mathbb{R}$\\
\bottomrule
\end{tabular}

\subsection{Semantics}

\subsubsection{Environment Variables}

inputFile: sequence of string \#\textit{f[i] is the ith string in the text file f}\\ 

\subsubsection{State Variables}

\renewcommand{\arraystretch}{1.2}
\begin{longtable*}[l]{l} 
\# From R1\\
$a$: $\mathbb{R}$ \\
$b$: $\mathbb{R}$ \\
$g$: glassType \\
$P_{b_\text{tol}}$: $\mathbb{R}$ \\
$\text{SD}_x$ : $\mathbb{R}$ \\
$\text{SD}_y$ : $\mathbb{R}$ \\
$\text{SD}_z$ : $\mathbb{R}$ \\
$t$: $\mathbb{R}$ \\
$w$: $\mathbb{R}$ \\
~\\
\# From R2\\
$t_d$: $\mathbb{R}$ \\
LDF: $\mathbb{R}$ \\
LSF: $\mathbb{R}$ \\
\end{longtable*}

\subsubsection{Assumptions}

\begin{itemize}

\item load\_params will be called before the values of any state variables will be accessed.

\item The file contains the string equivalents of the numeric values for
each input parameter in order, each on a new line. The order is the same as in
the table in R1 of the SRS. Any comments in the input file should be denoted
with a '\#' symbol.

\end{itemize}

\subsubsection{Access Routine Semantics}

\noindent Param.$a$:
\begin{itemize}
\item output: \textit{out} := $a$
\item exception: none
\end{itemize}

\noindent Param.$b$:
\begin{itemize}
\item output: \textit{out} := $b$
\item exception: none
\end{itemize}

...
~\newline

\noindent Param.LSF:
\begin{itemize}
\item output: \textit{out} := LSF
\item exception: none
\end{itemize}

\noindent load\_params($s$):
\begin{itemize}
\item transition: The filename $s$ is first associated with the file f.  {inputFile} is used to
  modify the state variables using the following procedural specification:
\begin{enumerate}
\item Read data sequentially from inputFile to populate the state variables from
  R1 ($L$ to $\mathit{ConsTol}$).
\item Calculate the derived quantities (all other state variables) as follows:
\begin{itemize}
\item $V_{\text{tank}} := \pi \times L \times (\frac{D}{2}) ^ 2$
\item $m_W := \rho_w (V_t - V_p)$
\item $m_P := \rho_p V_p$
\item $\tau_W := \frac{m_w C_w}{A_c h_c}$
\item $\eta := \frac{h_p A_p}{h_c A_c}$
\item $\tau_P^S := \frac{m_p C_{ps}}{h_p A_p}$
\item $\tau_P^L := \frac{m_p C_{pl}}{h_p A_p}$
\item $E_{P\text{melt}}^{\text{init}} := C_{ps} m_p (T_{\text{melt}} - T_{\text{init}})$
\item $E_{P\text{melt}}^{\text{all}} := H_f m_p$
\item $m_W^{\text{noPCM}} := \rho_w  V_t$
\item $\tau_W^{\text{noPCM}} := \frac{m_W^{\text{noPCM}} C_w}{h_c A_c}$
\end{itemize}
\item verify\_params()
\end{enumerate}

\item exception: exc := a file name $s$ cannot be found OR the format of
  inputFile is incorrect $\Rightarrow$  FileError
\end{itemize}

\noindent verify\_params():
\begin{itemize}
\item out: \textit{out} := none
\item exception: exc := 
\end{itemize}
\noindent \begin{longtable*}[l]{l l} 
$\neg (L > 0)$ & $\Rightarrow$ badLength\\
$\neg (L_{\text{min}} \leq L \leq L_{\text{max}})$ & $\Rightarrow$ warnLength\\
$\neg (D > 0)$ & $\Rightarrow$ badDiam\\
$\neg ({\frac{D}{L}}_\text{min} \leq \frac{D}{L} \leq {\frac{D}{L}}_\text{max})$ & $\Rightarrow$ warnDiam\\
$\neg (V_P > 0)$ & $\Rightarrow$ badPCMVolume\\
$ \neg (V_P \geq \text{minfract} \cdot V_{\text{tank}}(D, L)) $ & $\Rightarrow$ warnPCMVol\\
$\neg (V_P < V_{\text{tank}}(D, L))$ & $\Rightarrow$ badPCMAndTankVol\\
$\neg (A_P > 0)$ & $\Rightarrow$ badPCMArea\\
$\neg (V_P \leq A_P \leq \frac{2}{h_\text{min}} V_P)$ & $\Rightarrow$ warnVolArea\\
$\neg (\rho_P > 0)$  & $\Rightarrow$ badPCMDensity\\
$\neg (\rho_P^{\text{min}} < \rho_P < \rho_P^{\text{max}})$ & $\Rightarrow$ warnPCMDensity\\
%   $T_\text{melt}^{P}$ 	& $0 < T_\text{melt}^{P} < T_C$ (+) & &  44.2 
% 	\si[per-mode=symbol] {\celsius} & 10\%
%   \\
%   $C_P^S$ & $C_P^S > 0$ & $C_{P\text{min}}^S < C_P^S < C_{P\text{max}}^S$ & 1760 
% 	\si[per-mode=symbol] {\joule\per\kilo\gram\per\celsius} & 10\%
%   \\
%   $C_P^L$ & $C_P^L > 0$ & $C_{P\text{min}}^L < C_P^S < C_{P\text{max}}^L$ & 2270 
% 	\si[per-mode=symbol] {\joule\per\kilo\gram\per\celsius} & 10\%
%   \\
%   $H_f$ & $H_f > 0$ & $H_f^{\text{min}} < H_f < H_f^{\text{max}}$ & 211600 
% 	\si[per-mode=symbol] {\joule\per\kilo\gram} & 10\%
%   \\
%   $A_C$ & $A_C > 0$ (*) & $A_C \leq A_C^{\text{max}}$ & 0.12 \si[per-mode=symbol] {\square\metre} & 10\%
%   \\
%   $T_C$ & $0 < T_C < 100$ (+)	& & 50 \si[per-mode=symbol] {\celsius} & 10\%
%   \\
%   $\rho_W$ & $\rho_W > 0$ & $\rho_W^{\text{min}} < \rho_W \leq \rho_W^{\text{max}}$ 
% 	& 1000 \si[per-mode=symbol] {\kilo\gram\per\cubic\metre} & 10\%
%   \\
%   $C_W$ & $C_W > 0$ & $C_W^{\text{min}} < C_W < C_W^{\text{max}}$ & 4186 
% 	\si[per-mode=symbol] {\joule\per\kilo\gram\per\celsius} & 10\%
%   \\
%   $h_C$ & $h_C > 0$ & $h_C^{\text{min}} \leq h_C \leq h_C^{\text{max}}$ 
% 	& 1000 \si[per-mode=symbol] {\watt\per\square\metre\per\celsius} & 10\%
%   \\
%   $h_P$ & $h_P > 0$ & $h_P^{\text{min}} \leq h_P \leq h_P^{\text{max}}$ 
% 	& 1000 \si[per-mode=symbol] {\watt\per\square\metre\per\celsius} & 10\%
%   \\
%   $T_\text{init}$ & $0 < T_\text{init} < T_\text{melt} $ (+) & & 40 \si[per-mode=symbol] {\celsius} & 10\%
%   \\
%   $t_\text{final}$ & $t_\text{final} > 0$ & $t_\text{final} < t_{\text{final}}^{\text{max}}$ (**) 
% 		& 50000 \si[per-mode=symbol] {\second} & 10\%
%   \\
\end{longtable*}

etc.  See Appendix (Section~\ref{Appendix}) for the complete list of exceptions and
associated error messages.

\subsection{Considerations}

The value of each state variable can be accessed through its name (getter).  An
access program is available for each state variable.  There are no setters for
the state variables, since the values will be set and checked by load params and
not changed for the life of the program.

\newpage

% \section{MIS of Input Verification Module} \label{VerifyInput}

% \subsection{Module}

% verify\_params

% \subsection{Uses}

% Param (Section~\ref{Parameters})

% \subsection{Syntax}

% \subsubsection{Exported Access Programs}

% \begin{center}
% \begin{tabular}{p{3cm} p{1cm} p{1cm} p{9cm}}
% \hline
% \textbf{Name} & \textbf{In} & \textbf{Out} & \textbf{Exceptions} \\
% \hline
% verify\_valid & - & - & badLength, badDiam, badPCMVolume, badPCMAndTankVol,
%                         badPCMArea, badPCMDensity, badMeltTemp,
%                         badCoilAndInitTemp, badCoilTemp, badPCMHeatCapSolid,
%                         badPCMHeatCapLiquid, badHeatFusion, badCoilArea,
%                         badWaterDensity, badWaterHeatCap, badCoilCoeff,
%                         badPCMCoeff, badInitTemp, badFinalTime,
%                         badInitAndMeltTemp \\
% \hline
% verify\_recommend & - & - & - \\
% \hline
% \end{tabular}
% \end{center}

% \subsection{Semantics}

% \subsubsection{Environment Variables}

% $win$: 2D array of pixels displayed on the screen.

% \subsubsection{Assumptions}

% All of the fields Param have been assigned values before any of the access
% routines for this module are called.

% \subsubsection{Access Routine Semantics}

% verify\_valid(): 
% \begin{itemize}
% \item transition: none
% \item exceptions: exc := (\\
% Param.getL() $\leq 0 \Rightarrow$ badLength $|$\\
% Param.get\_diam() $\leq 0 \Rightarrow$ badDiam $|$\\
% Params.get\_Vp() $\leq 0 \Rightarrow$ badPCMVolume $|$\\
% Params.getVp() $\geq$ Params.Vt $\Rightarrow$ badPCMAndTankVol $|$\\
% Params.getAp() $\leq 0 \Rightarrow$ badPCMArea $|$\\
% Params.get\_rho\_p() $\leq 0 \Rightarrow$ badPCMDensity $|$\\
% Params.getTmelt() $\leq 0 \Rightarrow$ badMeltTemp $|$\\
% Params.getTmelt() $\geq$ Params.getTc() $\Rightarrow$ badMeltTemp $|$\\
% Params.getTc() $\leq$ Params.getTinit() $\Rightarrow$ badCoilAndInitTemp $|$\\
% Params.getTc() $\geq 100 \lor$ Params.getTc() $\leq 0 \Rightarrow$ badCoilTemp $|$\\
% Params.getC\_ps() $\leq 0 \Rightarrow$ badPCMHeatCapSolid $|$\\
% Params.getC\_pl() $\leq 0 \Rightarrow$ badPCMHeatCapLiquid $|$\\
% Params.getHf() $\leq 0 \Rightarrow$ badHeatFusion $|$\\
% Params.getAc() $\leq 0 \Rightarrow$ badCoilArea $|$\\
% Params.get\_rho()\_w $\leq 0 \Rightarrow$ badWaterDensity $|$\\
% Params.getC\_w() $\leq 0 \Rightarrow$ badWaterHeatCap $|$\\
% Params.get\_hc() $\leq 0 \Rightarrow$ badCoilCoeff $|$\\
% Params.get\_hp() $\leq 0 \Rightarrow$ badPCMCoeff $|$\\
% Params.getTinit() $\leq 0 \lor$ Params.getTinit() $\geq 100 \Rightarrow$
% badInitTemp $|$\\
% Params.get\_tfinal() $\leq 0 \Rightarrow$ badFinalTime $|$\\
% Params.getTinit() $\geq$ Params.getTmelt() $\Rightarrow$ badInitAndMeltTemp)  
% \end{itemize}

% verify\_recommend():
% \begin{itemize}
% \item transition: none
% \item exceptions: exc := (\\
% Params.getL() $< 0.1 \lor$ Params.getL() $> 50 \Rightarrow$ warnLength $|$\\
% Params.getdiam() / Params.getL() $< 0.002 \lor$ Params.getdiam() / Params.getL() $> 200
% \Rightarrow$ warnDiam $|$\\
% Params.getVp() $<$ Params.getVt() $\times 10 ^ -6 \Rightarrow$ warnPCMVol $|$\\
% Params.getVp() $>$ Params.getAp() $\lor$ Params.getAp $> (2/0.001) \times$ Params.getVp()
% $\Rightarrow$ warnVolArea $|$\\
% (Params.get\_rho\_p() $\leq 500) \lor ($ Params.get\_rho\_p() $\geq 20000) \Rightarrow$
% warnPCMDensity $|$ ... )\\
% \# \textit{Need to continue for the rest of the example - tabular form?}
% \# \textit{Should add a module (Configuration Module) to store symbolic constants}
%  % Params.getC\_ps \leq 100 \lor Params.getC\_ps \geq 4000 \Rightarrow$
%  % warnPCMHeatCapSolid $|$\\
%  % Params.getC\_pl \leq 100 \lor Params.getC\_pl \geq 5000 \Rightarrow$
%  % warnPCMHeatCapLiquid $|$\\
%  % Params.getAc > \pi \times (Params.getdiam / 2) ^ 2 \Rightarrow$ warnCoilArea
%  % $|$\\
%  % Params.getrho\_w \leq 950 \lor Params.getrho\_w > 1000 \Rightarrow$
%  % warnWaterDensity $|$\\
%  % Params.getC\_w \leq 4170 \lor Params.getC\_w \geq 4210 \Rightarrow$
%  % warnWaterHeatCap $|$\\
%  % Params.gethc \leq 10 \lor Params.gethc \geq 10000 \Rightarrow$ warnCoilCoeff $|$\\
%  % Params.gethp \leq 10 \lor Params.gethp \geq 10000 \Rightarrow$ warnPCMCoeff $|$\\
%  % Params.gettfinal \leq 0 \lor Params.gettfinal \geq 86400 \Rightarrow$ warnFinalTime)
% \end{itemize}

% \subsection{Considerations}

% See Appendix (Section~\ref{Appendix}) for the complete list of exceptions and
% associated error messages.

\newpage

%%%%%%%%%%%%%%%%%%%%%%%%%%%%%%%%%%%%%%%%%%%

\section{MIS of LoadASTM Module} \label{LoadASTM}

\newpage

%%%%%%%%%%%%%%%%%%%%%%%%%%%%%%%%%%%%%%%%%%%

\section{MIS of Output Module} \label{Output}

\newpage

%%%%%%%%%%%%%%%%%%%%%%%%%%%%%%%%%%%%%%%%%%%

\section{MIS of Calc Module} \label{Calc}

\newpage

%%%%%%%%%%%%%%%%%%%%%%%%%%%%%%%%%%%%%%%%%%%

\section{MIS of Control Module} \label{Main}

\subsection{Module}

main

\subsection{Uses}

Param (Section~\ref{Parameters}), Temperature
(Section~\ref{Temperature}), Solver
(Section~\ref{ODE}), Energy (Section~\ref{Energy}), verify\_output (Section~\ref{VerifyOutput}), plot
(Section~\ref{Plot}), output (Section~\ref{Output})

\subsection{Syntax}

\subsubsection{Exported Access Programs}

\begin{center}
\begin{tabular}{p{2cm} p{4cm} p{4cm} p{2cm}}
\hline
\textbf{Name} & \textbf{In} & \textbf{Out} & \textbf{Exceptions} \\
\hline
main & - & - & - \\
\hline
\end{tabular}
\end{center}

\subsection{Semantics}

\subsubsection{State Variables}

None

\subsubsection{Access Routine Semantics}

\noindent main():
\begin{itemize}
\item transition: Modify the state of Param module and the environment variables
  for the Plot and Output modules by following these steps\\
\end{itemize}

\noindent Get (filenameIn: string) and (filenameOut: string) from user\\

\noindent load\_params(filenameIn)\\

\newpage

%%%%%%%%%%%%%%%%%%%%%%%%%%%%%%%%%%%%%%%%%%%

\section{MIS of ConstantsAndTypes Module} \label{Calc}

\subsection*{Module}

ConstantsAndTypes

\subsection* {Uses}

N/A

\subsection* {Syntax}

\subsubsection* {Exported Constants}

\renewcommand{\arraystretch}{1.2}
\begin{longtable*}[l]{l} 
\# From Table 8 in SRS\\
$m$ := 7\\
$k$ := $\left(2.86\right)10^{-53}$\\
$E$ := $\left(7.17\right)10^{7}$\\
${t_{d}}$ := $3$\\
LDF := $\left(\frac{{t_{d}}}{60}\right)^{\frac{m}{16}}$\\
LSF := $1$\\
${d_{\text{max}}}$ := $5.0$\\
${d_{\text{min}}}$ := $0.1$\\
${\text{AR}_{\text{max}}}$ := $5.0$\\
${w_{\text{max}}}$ := $910.0$\\
${w_{\text{min}}}$ := $4.5$\\
${\text{SD}_{\text{min}}}$ := $6.0$\\
${\text{SD}_{\text{max}}}$ := $130.0$\\
\end{longtable*}

\subsubsection* {Exported Types}

GlassTypeT = \{AN, FT, HS\}\\ 
ThicknessT = \{2.5, 3.0, 5.0, 8.0, 12.0, 19.0, 2.7, 4.0, 6.0, 10.0, 16.0, 22.0\}

\subsubsection* {Exported Access Programs}

None

\subsection* {Semantics}

\subsubsection* {State Variables}

None

\subsubsection* {State Invariant}

None

\newpage


%%%%%%%%%%%%%%%%%%%%%%%%%%%%%%%%%%%%%%%%%%%

\section{MIS of FunctT Module} \label{LoadASTM}

\newpage

%%%%%%%%%%%%%%%%%%%%%%%%%%%%%%%%%%%%%%%%%%%

\section{MIS of ContoursT Module} \label{LoadASTM}

\newpage

%%%%%%%%%%%%%%%%%%%%%%%%%%%%%%%%%%%%%%%%%%%

\section{MIS of SeqServices Module} \label{LoadASTM}

\newpage

%%%%%%%%%%%%%%%%%%%%%%%%%%%%%%%%%%%%%%%%%%%

\end{document}
