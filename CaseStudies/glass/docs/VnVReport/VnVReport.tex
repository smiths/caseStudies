\documentclass[12pt, titlepage]{article}

\usepackage{booktabs}
\usepackage{tabularx}
\usepackage{graphicx}
\usepackage{longtable}
\usepackage{comment}
\usepackage{siunitx}
\usepackage{afterpage}
\usepackage{pdflscape}
\usepackage{hyperref}
\hypersetup{
	colorlinks,
	citecolor=black,
	filecolor=black,
	linkcolor=red,
	urlcolor=blue
}
\usepackage[round]{natbib}
\usepackage{xr}

\input{../Comments}

\newcommand{\progname}{GlassBR}

\externaldocument[SRS-]{../SRS/glassbr_srs}
\newcommand{\rref}[1]{R\ref{#1}}
\newcommand{\nfrref}[1]{NFR\ref{#1}}

\externaldocument[MG-]{../Design/MG/glassbr_mg}
\newcommand{\mref}[1]{M\ref{#1}}

\externaldocument[SVnV-]{../VnVPlan/SystVnVPlan/SystVnVPlan}
\newcommand{\tcref}[1]{TC\ref{#1}}

\externaldocument[UVnV-]{../VnVPlan/UnitVnVPlan/UnitVnVPlan}
\newcommand{\utcref}[1]{TC\ref{#1}}


\begin{document}
	
	\title{Test Report: Glass Breakage analysis Program (\progname{})} 
	\author{Vajiheh Motamer}
	\date{\today}
	
	\maketitle
	
	\pagenumbering{roman}
	
	\section{Revision History}
	
	\begin{tabularx}{\textwidth}{p{3cm}p{2cm}X}
		\toprule {\bf Date} & {\bf Version} & {\bf Notes}\\
		\midrule
		12/20/18 & 1.0 & Initial draft based on initial SystemVnVPlan\\
		12/25/18 & 1.1 & Make changes based on SystemVnVPlan changes\\
		\bottomrule
	\end{tabularx}
	
	~\newpage
	
	\section{Symbols, Abbreviations and Acronyms}
	
	The symbols, abbreviations, and acronyms used in this document include those 
	defined in the table below, as well as any defined in the tables found in 
	Section \ref{SRS-sec_abbrev} of the Software Requirements Specification (SRS) 
	document
	\newline
	
	\renewcommand{\arraystretch}{1.2}
	\begin{tabular}{l l} 
		\toprule		
		\textbf{symbol} & \textbf{description}\\
		\midrule
		MIS & Module Interface Specification\\
		MG & Module Guide\\
		TC & Test Case\\
		VnV & Verification and Validation\\
		\bottomrule
	\end{tabular}\\
	
	\newpage
	
	\tableofcontents
	
	\listoftables %if appropriate
	
	\listoffigures %if appropriate
	
	\newpage
	
	\pagenumbering{arabic}
	
	This document outlines the results of testing for \progname{}. 
	Section~\ref{sec_FuncReqEval} reports on the Test Cases (TCs) for functional 
	requirements and Section~\ref{sec_NonFuncReqEval} reports on the tests for 
	non-functional requirements, all of which are described in the System 
	Verification and Validation (VnV) Plan for this project. 
	Section~\ref{sec_Comparison} compares this implementation of \progname{} to the 
	original implementation. Section~\ref{sec_UnitTests} reports on the results of 
	the unit tests, which are described in the Unit VnV Plan for this project. 
	Section~\ref{sec_Changes} comments on changes to the project that that came as 
	a result of the testing. Section~\ref{sec_AutoTests} describes how the tests 
	were implemented with an automated testing framework. 
	Sections~\ref{sec_TraceReq} and \ref{sec_TraceMod} show the traceability 
	between test cases and requirements and modules. Supporting documents and other 
	resources, such as the VnVPlans and original implementation, can be found on 
	\href{https://github.com/smiths/caseStudies/tree/master/CaseStudies/glass}{the 
		GitHub repository for this project}.
	
	\section{Functional Requirements Evaluation} \label{sec_FuncReqEval}
	
	The System VnV Plan described \tcref{SVnV-TC_defultInput} - 
	\tcref{SVnV-TC_LowThicknessInput} for testing the requirement for 
	verifying different forms of valid user inputs. Besides, these test cases verified \progname{}
	 for the special values that glass is safe or not. All of these tests passed. 
	
	~\newline \noindent The System VnV Plan described \tcref{SVnV-TC_checkAPositiveTest} - 
	\tcref{SVnV-TC_incorrectWEqUpprBndTest} for testing the requirement for 
	verifying that inputs meet physical constraints. The test cases \tcref{SVnV-TC_incorrectAspectREqLwrBndTest} and \tcref{SVnV-TC_incorrectAspectREqUpprBndTest} and \tcref{SVnV-TC_incorrectWEqLwrBndTest} and \tcref{SVnV-TC_incorrectWEqUpprBndTest} passed.
	However, \tcref{SVnV-TC_incorrectSDEqLwrBndTest} and \tcref{SVnV-TC_incorrectSDEqUpprBndTest} failed at the first time, then they passed. I think it would be better equation should be removed from SRS ``W'' and ``SD'' bands.
	
	~\newline \noindent The final functional requirement tests in the System VnV 
	Plan, \tcref{SVnV-TstDrvdValsHSGlTy} - \tcref{SVnV-TstDrvdValsFTGlTy}, were for 
	verifying the requirements for verifying and delivering output. These tests all 
	passed.	
	
	The System V\&V Plan can be found at: 
	\url{https://github.com/smiths/caseStudies/tree/master/CaseStudies/glass/docs/VnVPlan/SystVnVPlan/SystVnVPlan.pdf}.
	
	\section{Nonfunctional Requirements Evaluation} \label{sec_NonFuncReqEval}
	
	
	\subsection{Reusability}
	
	\tcref{SVnV-TC_Reusability} in the System VnV Plan is a test for gaining 
	confidence in the reusability of \progname{}. The alternative and new implementation(units has been considered as meter)  was 
	written based off of the old implementation which the units have been considered as ``mm'', which 
	accepted by \progname . The tests 
	passed, showing that all of the other modules were reusable in this alternative 
	version of \progname{}.
	
	\subsection{Understandability}
	Since reusability are related to understandability, the 
	success of the tests for those requirements gives confidence in the 
	understandability of \progname{}.

	\subsection{Correctness}

The success of the system tests for the calculations of \progname{}, 
specifically \tcref{SVnV-TC_defultInput} - \tcref{SVnV-TC_incorrectSDEqUpprBndTest}, gives 
confidence in the correctness of \progname{}.

\section{Comparison to Existing Implementation}	 \label{ComparisonImplem}

This section is not applicable for \progname{}.
	
\section{Unit Testing}  \label{sec_UnitTests}
The Unit VnV Plan described \tcref{UVnV-TC_Length} - 
\tcref{UVnV-getCEXPContoursTest}, all of which were implemented in the unit testing 
framework and passed, with the exception of 
\tcref{UVnV-FuncTevalDTest}, which failed on the first run, 

\section{Changes Due to Testing} \label{sec_Changes}



The test cases for missing inputs in the System V\&V Plan initially failed 
because there was a bug in the condition that checked for missing inputs and 
was producing an incorrect output. Upon modifying the condition, the correct 
output was produced and the tests have passed.

The failure of \tcref{UVnV-FuncTevalDTest} revealed that the 
implementation had a bug in the implementation. This prompted a 
change to the FuncADT module to properly handle this case. 

\section{Automated Testing} \label{AutomatedTesting}

Every test, with the exception of the tests 
for non-functional requirements, was automated so that the tests could be 
executed with a simple command.

\section{Trace to Requirements} \label{sec_TraceReq}

Tables~\ref{Table:TraceTestReq} and \ref{Table:TraceTestNFReq} were taken 
directly from the System VnV Plan document for this project. It shows that the 
tests effectively cover every Requirement (R) and Non-Functional Requirement 
(NFR) from the SRS. Unless otherwise specified, the test cases referenced in 
the table refer to tests described in the System VnV Plan.

\begin{table}[!h]
	\centering
	\begin{tabular}{|c|c|c|c|c|c|c|}
		\hline
		& \rref{SRS-Input}& \rref{SRS-KnownValues}& \rref{SRS-Verify_IN}& \rref{SRS-R_OutputInput}& 
		\rref{SRS-R_ Comparison}& \rref{SRS-R_Output}\\
		\hline
		\tcref{SVnV-TC_defultInput} - \tcref{SVnV-TC_LowThicknessInput}                 
		& X& & & & &  \\ \hline
		\tcref{SVnV-TC_defultInput} - \tcref{SVnV-TC_LowThicknessInput}                 
		& &X & & & &  \\ \hline
		\tcref{SVnV-TC_checkAPositiveTest} - \tcref{SVnV-TC_incorrectWEqUpprBndTest} 
		& & & X& & & \\ \hline
		\tcref{SVnV-TC_defultInput} - \tcref{SVnV-TC_LowThicknessInput}                             
		& & & &X& &   \\ \hline
		\tcref{SVnV-TC_defultInput} - \tcref{SVnV-TC_LowThicknessInput}                              
		& & & & &X &   \\ \hline
		\tcref{SVnV-TstDrvdValsHSGlTy} - \tcref{SVnV-TstDrvdValsFTGlTy}                                                
		& & & & & &X  \\ \hline
		
		\hline
	\end{tabular}
\caption{Traceability Matrix Showing the Connections Between Requirements 
and Test Cases}
\label{Table:TraceTestReq}
\end{table}


\end{document}