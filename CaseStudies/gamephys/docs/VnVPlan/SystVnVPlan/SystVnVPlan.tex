\documentclass[12pt, titlepage]{article}

\usepackage{booktabs}
\usepackage{tabularx}
\usepackage{hyperref}
\usepackage{bm}

\usepackage{amsmath, mathtools}

\usepackage[justification=centering]{caption}

\usepackage{amsfonts}

\usepackage{amssymb}

\usepackage{commath}

\usepackage{graphicx}

\usepackage{pdflscape}

\usepackage{colortbl}

\usepackage{xr}

\usepackage{hyperref}

\usepackage{longtable}

\usepackage{xfrac}

\usepackage{tabularx}

\usepackage{float}

\usepackage[per-mode=reciprocal]{siunitx}

\usepackage{booktabs}
\hypersetup{
    colorlinks,
    citecolor=black,
    filecolor=black,
    linkcolor=red,
    urlcolor=blue
}
\usepackage[round]{natbib}
\newcommand{\colZwidth}{1.0\textwidth}

\newcommand{\blt}{- } %used for bullets in a list

\newcommand{\colAwidth}{0.13\textwidth}

\newcommand{\colBwidth}{0.82\textwidth}

\newcommand{\colCwidth}{0.1\textwidth}

\newcommand{\colDwidth}{0.05\textwidth}

\newcommand{\colEwidth}{0.8\textwidth}

\newcommand{\colFwidth}{0.17\textwidth}

\newcommand{\colGwidth}{0.5\textwidth}

\newcommand{\colHwidth}{0.28\textwidth}

\newcommand{\tref}[1]{T\ref{#1}}

\newcounter{tablenum} %Table Number

\newcommand{\tbthetablenum}{T\thetablenum}

\newcommand{\rref}[1]{R\ref{#1}}

\newcommand{\lthelcnum}{LC\thelcnum}

\newcommand{\lcref}[1]{LC\ref{#1}}

\newcommand{\dv}{\mathrm{d}\mathbf{v}}

\newcommand{\dx}{\mathrm{d}\mathbf{x}}

\newcommand{\dr}{\mathrm{d}\mathbf{r}}

\newcommand{\dpos}{\mathrm{d}\mathbf{p}}

\newcommand{\dt}{\mathrm{d}t}

\newcommand{\utheucnum}{UC\theucnum}

\newcommand{\ucref}[1]{UC\ref{#1}}

\input{../../Comments}

\newcommand{\progname}{Tamias2D}

\begin{document}

\title{\progname: System Verification and Validation Plan} 
\author{Oluwaseun Owojaiye}
\date{\today}
	
\maketitle

\pagenumbering{roman}

\section{Revision History}

\begin{tabularx}{\textwidth}{p{3cm}p{2cm}X}
\toprule {\bf Date} & {\bf Version} & {\bf Notes}\\
\midrule
2018-10-15 & 1.0 & Initial draft\\
2018-11-01 & 1.1 & Updates based on document review and issue tracker\\
2018-12-22 & 1.2 & Updated testcases\\
\bottomrule
\end{tabularx}

~\newpage

\section{Symbols, Abbreviations and Acronyms}
\renewcommand{\arraystretch}{1.2}


\subsection{Table of Symbols} \label{TblOfSym}
\begin{tabularx}{\textwidth}{p{3cm}p{2cm}X}
	\toprule 
	\textbf{Symbol} & \textbf{Unit} & \textbf{Description} \\
	\midrule
	$\mathbf{a}$ & \si{\metre\per\second\tothe{2}} & Acceleration \\
	$\alpha$ & \si{\radian\per\second\tothe{2}} & Angular acceleration \\
	$C_\text{R}$ & unitless & Coefficient of restitution \\
	$\mathbf{F}$ & \si{\newton} & Force \\
	 $g$ & \si{\metre\per\second\tothe{2}} & Gravitational acceleration ($9.81$ \si{\metre\per\second\tothe{2}}) \\
	$G$ & \si{\metre\tothe{3}\per\kilogram\second\tothe{-2}} & Gravitational constant ($6.673 \times 10^{-11}$ \si{\metre\tothe{3}\per\kilogram\second\tothe{-2}}) \\
	$\mathbf{I}$ & \si{\kilogram\metre\tothe{2}} & Moment of inertia \\
	$\mathbf{\hat{i}}$ & \si{\metre} & Horizontal unit vector \\
	$\mathbf{\hat{j}}$ & \si{\metre} & Vertical unit vector \\
	$j$ & \si{\newton\second} & Impulse (scalar) \\
	$\mathbf{J}$ & \si{\newton\second} & Impulse (vector) \\
	$L$ & \si{\metre} & Length \\
	$m$ & \si{\kilogram} & Mass \\
	$n$ & unitless & Number of particles in a rigid body \\
	$\mathbf{n}$ & \si{\metre} & Collision normal vector \\
	$\boldsymbol{\omega}$ & \si{\radian\per\second} & Angular velocity \\
	$\mathbf{p}$ & \si{\metre} & Position \\
	$\boldsymbol{\phi}$ & \si{\radian} & Orientation \\
	$r$ & \si{\metre} & Distance \\
	$\mathbf{r}$ & \si{\metre} & Displacement \\
	$t$ & \si{\second} & Time \\
	$\tau$ & \si{\newton\metre} & Torque \\
	$\boldsymbol{\theta}$ & \si{\radian} & Angular displacement \\
	$\mathbf{v}$ & \si{\metre\per\second} & Velocity \\
	
	\bottomrule
\end{tabularx}

\subsection{Abbreviations and Acronyms}
\begin{tabular}{l l} 
	\toprule		
	\textbf{Symbol} & \textbf{Description}\\
	\midrule 
	IM & Instance Model\\
	R & Requirement\\
	T & Test\\
	TBD & To be determined\\
	2D & Two-dimensional\\
	SRS & System Requirement Specification\\
	\bottomrule
\end{tabular}\\

%\wss{symbols, abbreviations or acronyms -- you can simply reference the %SRS tables, if appropriate}

\newpage

\tableofcontents

\listoftables

\listoffigures

\newpage

\pagenumbering{arabic}

This document provides a high-level verification and validation plan for \progname - a 2D Rigid Body Game Physics Library. This document is based on the System Requirement Specification(SRS) document located in the following project repository link: \url{https://github.com/smiths/caseStudies/tree/master/CaseStudies/gamephys}. It discusses the verification and validation requirements for \progname, and describes the test strategies and methods that will be used to evaluate the software. The verification and validation of the software utilizes review, analysis, and testing method to determine whether the software product complies with the specifed requirements. These requirements include both functional and non-functional.
%This document ... \wss{provide an introductory blurb and roadmap of the
 % Verification and Validation plan}

\wss{The text is better for version control, and for reading in other editors,
  if you use a hard-wrap at 80 characters}

\section{General Information}

\subsection{Summary}
The software being tested is \progname. It is a 2D rigid body game physics library designed to simulate the interaction between rigid bodies in the game space. It will simulate the movement of objects in space, the behavoiur of objects when there is a collision and it tracks a history of the velocity, position and orientation of each object. Since physics libraries are an important part of video game development, game developers would be able to make use of \progname in their products. 
%\wss{Say what software is being tested.  Give its name and a brief %overview of its general functions.}

\subsection{Objectives}

The purpose of verification and validation activitiy is to find bugs and defects in the TamiasMini2D software and also to determine if it has met all the required functionality. It is also to verify that the software meets the required standard and the end product conforms with the software requirements based on the SRS. The objectives of System VnV activities for \progname are to:
  \begin{itemize}
	\item Build confidence in software correctness and performance.
	\item Verify the maintainability of the software, based on the product's ability to be easily enhanced, modified
	  and reused.
	\item Verify and demonstrate the ease of use and learning of the software.
  \end{itemize}
	

%\wss{State what is intended to be accomplished.  The objective will be %around the qualities that are most important for your project.  You might %have something like: ``build confidence in the software correctness,''
%``demonstrate adequate usability.'' etc.  You won't list all of the %qualities,just those that are most important.}

\subsection{References}

\wss{You should introduce the references, not just include a link.}
\olu{updated}
\begin{itemize}
	\item[1.] Software Requirement Specification for \progname : \url {https://github.com/smiths/caseStudies/blob/gamephy_finaldoc/CaseStudies/gamephys/docs/SRS/GamePhysicsSRS.pdf}
\end{itemize}

%\wss{Reference relevant documentation.  This will definitely include your SRS}

\section{Plan}
	
\subsection{Verification and Validation Team}
The verification and validation team consists of a one member team: Olu Owojaiye
%\wss{Probably just you.  :-)}

\subsection{SRS Verification Plan}
The SRS for the project will be reviewed by Dr.\ Smith \wss{\LaTeX{} has a rule that it inserts two spaces at the end of a sentence.  It detects a sentence as a period followed by a capital letter.  This comes up, for instance, with Dr. Smith.  Since the period after Dr.\ isn't actually the end of a sentence, you need to tell \LaTeX{} to insert one space.  You do this either by Dr.\ Smith (if you don't mind a line-break between Dr.\ and Smith), or Dr.~Spencer Smith (to force \LaTeX{} to not insert a line break).} and coursemate Karol Serkis;
feedback will be provided. Some SRS feedback for this project have already been provided and addressed using github issue tracker. Once the software has been implemented, the SRS will be reviewed to ensure that software meets all the specified requirements and more feedback will be provided via github issue tracker.

\wss{You can be specific about which classmates are going to review your documents; the specific assignments are in Repos.xlsx.}
\olu{updated}

%\wss{List any approaches you intend to use for SRS verification. This may %just be ad hoc feedback from reviewers, like your classmates, or you may %have something more rigorous/systematic in mind..}

\subsection{Design Verification Plan}
To ensure that the Design Specification has been properly specified and meets
software requirements, Dr. Smith and coursemates Robert White and Hanae Zlitni will be verifying the software design. The Module Guide and Module Interface Specification located at \url{https://github.com/smiths/caseStudies/tree/gamephy_MG/CaseStudies/gamephys/docs/Design/MG} and \url{https://github.com/smiths/caseStudies/tree/gamephy_MG/CaseStudies/gamephys/docs/Design/MIS} respectively will
contain information about the software design. Feedback is expected to be
provided by reviewers via github issue tracker. \wss{You should have links to MG and MIS.} \olu{updated with links}
%\wss{Plans for design verification}

\subsection{Implementation Verification Plan}
 The implementation of \progname{} will involve inspection of the software to ensure that all the required features have been implemented successfully and are functional. Once the development activities are completed, Dr.\ Smith and one of my CAS741 coursemate will be assigned to perform the implementation verification activities. The software will be installed by the reviewer and system test cases specified in Section 5 will be executed as well as unit test cases located at \url{https://github.com/smiths/caseStudies/blob/gamephy_UnitVnV/CaseStudies/gamephys/docs/VnVPlan/UnitVnVPlan/UnitVnVPlan.pdf}. Reviewers are expected to verify both functional and non-functional requirements specified below except otherwise stated. Exploratory testing can also be performed by testers. Any implementation verification issues will be reported and tracked via gitHub issue tracker and these issues will be resolved in order of severity by myself. After the issues raised have been fixed, they will be sent back to the reviewer(s) for re-verification.
%\wss{You should at least point to the tests listed in this document and %the unit testing plan.}

\subsection{Software Validation Plan}

\progname{} will be validated at a future phase with existing physics library Pymunk. 
%\wss{If there is any external data that can be used for validation, you %should point to it here.  If there are no plans for validation, you %should state that here.}

\wss{You should the reason why
  there is no software validation plan.}
\olu{updated}

\section{System Test Description}
	
\subsection{Tests for Functional Requirements}

%\wss{Subsets of the tests may be in related, so this section is divided %into different areas.  If there are no identifiable subsets for the %tests, this level of document structure can be removed.}


\subsubsection{Translational Motion Testing}
	
\paragraph{}

\begin{enumerate}

\item{TC1: test\_system\_horizontal\_translation\_noforce()\\}
Description: This test computes the position and velocity of a 2D rigid body that is static over 4 secs. Gravity is not applied. The object does not move, hence velocity and position do not change. Varying positive input values can be used in TC1 for each of the parameters

Control: Automatic
					
Initial State: NA
					
Input:\\
      \hspace*{1.3cm} $\mathbf{p_i}$$\mathbf{(t_0)}$ = (50, 50) ;This  is the initial position of body - (x,y) coordinate position\\
       \hspace*{1.3cm}$\mathbf{v_i}$$\mathbf{(t_0)}$ = 0, 0\\
       \hspace*{1.3cm}$\mathbf{F}$$\mathbf{}$ = 0, 0\\
       \hspace*{1.3cm}$\mathbf{m}$$ \mathbf{}$ = 10\\
       \hspace*{1.3cm}$\mathbf{g}$$\mathbf{}$ = 0 (acceleration due to gravity does not apply on a static body)\\			
Output:\\  
	     $\mathbf{p_i}$$\mathbf{(t_1)}$ = (50, 50);
         $\mathbf{v_i}$$\mathbf{(t_1)}$ = (0, 0)\\
         $\mathbf{p_i}$$\mathbf{(t_2)}$ = (50, 50);
         $\mathbf{v_i}$$\mathbf{(t_2)}$ = (0, 0);\\
         $\mathbf{p_i}$$\mathbf{(t_3)}$ = (50, 50);
         $\mathbf{v_i}$$\mathbf{(t_3)}$ = (0, 0);\\
         $\mathbf{p_i}$$\mathbf{(t_4)}$ = (50, 50);
         $\mathbf{v_i}$$\mathbf{(t_4)}$ = (0, 0);\\    
					
How test will be performed: Unit testing with Pytest
Ref. source: \url{https://www.calculatorsoup.com/calculators/physics/displacement_v_a_t.php}
					
\item{TC2: test\_system\_horizontal\_translation\_right()\\}

Description: This test computes the position and velocity of a 2D rigid body moving at constant acceleration to the right over 4 secs. Gravity does not apply.
 
Control: Automatic
					
Initial State: NA
					
Input:\\ 
		$\mathbf{p_i}$$\mathbf{(t_0)}$ = (20, 20) this is the (x,y) coordinate position\\
	   $\mathbf{v_i}$$\mathbf{(t_0)}$ = 0\\
	   $\mathbf{F}$$\mathbf{}$ = (100, 0)\\
	   $\mathbf{m}$$ \mathbf{}$ = 10\\ 	   
	  
Output:\\
	    $\mathbf{p_i}$$\mathbf{(t_1)}$ = (22.84 20.00);
		$\mathbf{v_i}$$\mathbf{(t_1)}$ = (10.50, 0.0)\\ 
		$\mathbf{p_i}$$\mathbf{(t_2)}$ = (31.02, 20.0);
		$\mathbf{v_i}$$\mathbf{(t_2)}$ = (20.83, 0.0)\\ 
		$\mathbf{p_i}$$\mathbf{(t_3)}$ = (44.54, 20.00);
		$\mathbf{v_i}$$\mathbf{(t_3)}$ = (31.16, 0.0)\\
		$\mathbf{p_i}$$\mathbf{(t_4)}$ = (60.66, 20.00);
		$\mathbf{v_i}$$\mathbf{(t_4)}$ = (40.16, 0.0)\\  
					
How test will be performed: Unit testing with pytest

\wss{Where did the answers come from?  The reader won't be able to verify what
  you are saying.  You are verifying for one point in time.  You can accomplish
  more by verifying the full history of the change in position.  You should
  provide the closed form solutions for position change under constant
  acceleration.  You then get can run the simulation until the position is zero
  and verify that the positions are correct.  To get one number for your test
  you can use the Euclidean (or other) norm of your vector and then divide by the norm of the expected result.}
\olu{updated result with history of velocity and position over a period of time}

\item{TC3: test\_system\_horizontal\_translation\_left()\\}

Description: This test computes the position and velocity of a 2D rigid body moving at constant acceleration to the left over 4 secs. Gravity does not apply.

Control: Automatic

Initial State: NA

Input:\\ 
$\mathbf{p_i}$$\mathbf{(t_0)}$ = (950, 50) this is the (x,y) coordinate position\\
$\mathbf{v_i}$$\mathbf{(t_0)}$ = 0\\
$\mathbf{F}$$\mathbf{}$ = (-100, 0)\\
$\mathbf{m}$$ \mathbf{}$ = 10\\ 	   

Output:\\
$\mathbf{p_i}$$\mathbf{(t_1)}$ = (947.16 50.00);
$\mathbf{v_i}$$\mathbf{(t_1)}$ = (-10.50, 0.0)\\ 
$\mathbf{p_i}$$\mathbf{(t_2)}$ = (938.98, 50.0);
$\mathbf{v_i}$$\mathbf{(t_2)}$ = (-20.83, 0.0)\\ 
$\mathbf{p_i}$$\mathbf{(t_3)}$ = (925.46, 50.00);
$\mathbf{v_i}$$\mathbf{(t_3)}$ = (-31.16, 0.0)\\
$\mathbf{p_i}$$\mathbf{(t_4)}$ = (909.66, 50.00);
$\mathbf{v_i}$$\mathbf{(t_4)}$ = (-40.16, 0.0)\\  

How test will be performed: Unit testing with pytest

\item{TC4: test\_system\_free\_falling\\}

Description: This test computes the position and velocity of bodies falling from height over time. Force of gravity is in effect. The position and velocity is calculated over time. 

Control: Automatic

Initial State: NA

Input: \\
Body1: $\mathbf{p_i}$$\mathbf{(t_0)}$ = (100.0, 50.0); $\mathbf{v_i}$$\mathbf{(t_0)}$ = 0\\
Body2: $\mathbf{p_i}$$\mathbf{(t_0)}$ = (300.0, 50.0); $\mathbf{v_i}$$\mathbf{(t_0)}$ = 0\\
Body3: $\mathbf{p_i}$$\mathbf{(t_0)}$ = (600.0, 50.0); $\mathbf{v_i}$$\mathbf{(t_0)}$ = 0\\
$\mathbf{m}$$ \mathbf{}$ = 100.0;
$\mathbf{g}$$\mathbf{}$ = 9.8 (acceleration due to gravity in space)\\
Mass(m) for each body is 100\\

Output:\\
Body1:\\
 $\mathbf{p_i}$$\mathbf{(t_1)}$ = (100.0, 52.78);
 $\mathbf{v_i}$$\mathbf{(t_1)}$ = (0.0, 10.289)\\
 $\mathbf{p_i}$$\mathbf{(t_2)}$ = (100.0, 60.80);
 $\mathbf{v_i}$$\mathbf{(t_2)}$ = (0.0, 20.41)\\
 $\mathbf{p_i}$$\mathbf{(t_3)}$ = (100.0, 74.05);
 $\mathbf{v_i}$$\mathbf{(t_3)}$ = (0.0, 30.54)\\
 $\mathbf{p_i}$$\mathbf{(t_4)}$ = (100.0, 74.05);
 $\mathbf{v_i}$$\mathbf{(t_4)}$ = (0.0, 39.54)\\
 
 Body2:\\
 $\mathbf{p_i}$$\mathbf{(t_1)}$ = (300.0, 52.78);
 $\mathbf{v_i}$$\mathbf{(t_1)}$ = (0.0, 10.289)\\
 $\mathbf{p_i}$$\mathbf{(t_2)}$ = (300.0, 60.80);
 $\mathbf{v_i}$$\mathbf{(t_2)}$ = (0.0, 20.41)\\
 $\mathbf{p_i}$$\mathbf{(t_3)}$ = (300.0, 74.05);
 $\mathbf{v_i}$$\mathbf{(t_3)}$ = (0.0, 30.54)\\
 $\mathbf{p_i}$$\mathbf{(t_4)}$ = (300.0, 89.85);
 $\mathbf{v_i}$$\mathbf{(t_4)}$ = (0.0, 39.54)\\
 
 Body3:\\
 $\mathbf{p_i}$$\mathbf{(t_1)}$ = (600.0, 52.78);
 $\mathbf{v_i}$$\mathbf{(t_1)}$ = (0.0, 10.289)\\
 $\mathbf{p_i}$$\mathbf{(t_2)}$ = (600.0, 60.80);
 $\mathbf{v_i}$$\mathbf{(t_2)}$ = (0.0, 20.41)\\
 $\mathbf{p_i}$$\mathbf{(t_3)}$ = (600.0, 74.05);
 $\mathbf{v_i}$$\mathbf{(t_3)}$ = (0.0, 30.54)\\
 $\mathbf{p_i}$$\mathbf{(t_4)}$ = (600.0, 89.85);
 $\mathbf{v_i}$$\mathbf{(t_4)}$ = (0.0, 39.54)\\

How test will be performed: Unit testing with pytest

\item{TC5: test\_system\_free\_bodies\_forces\\}

Description: This test computes the position and velocity of floating bodies over time by applying force (F) on each body. We assume that the force of gravity is not in effect in this case. The position and velocity is calculated over time. 

Control: Automatic

Initial State: NA

Input: \\
Body1: $\mathbf{p_i}$$\mathbf{(t_0)}$ = (0.0, 50.0); $\mathbf{v_i}$$\mathbf{(t_0)}$ = 0; $\mathbf{(m_1)}$ = 1; $\mathbf{F}$ =(-300, 0) \\
Body2: $\mathbf{p_i}$$\mathbf{(t_0)}$ = (300.0, 50.0); $\mathbf{v_i}$$\mathbf{(t_0)}$ = 0; $\mathbf{(m_2)}$ = 4000; $\mathbf{F}$ = (0, 80000)\\
Body3: $\mathbf{p_i}$$\mathbf{(t_0)}$ = (600.0, 50.0); $\mathbf{v_i}$$\mathbf{(t_0)}$ = 0;
$\mathbf{m_3}$$ \mathbf{}$ = 10.0, $\mathbf{F}$ = (1000, -2000)\\


Output:\\
Body1:\\
$\mathbf{p_i}$$\mathbf{(t_1)}$ = (-85.233, 50.0);
$\mathbf{v_i}$$\mathbf{(t_1)}$ = (-314.97, 0)\\
$\mathbf{p_i}$$\mathbf{(t_2)}$ = (-330.60, 50.0);
$\mathbf{v_i}$$\mathbf{(t_2)}$ = (-624.94, 0.0)\\
$\mathbf{p_i}$$\mathbf{(t_3)}$ = (-736.10, 50.0);
$\mathbf{v_i}$$\mathbf{(t_3)}$ = (-934.91, 0.0)\\
$\mathbf{p_i}$$\mathbf{(t_4)}$ = (-1219.76, 50.0);
$\mathbf{v_i}$$\mathbf{(t_4)}$ = (-1204.88, 0.0)\\

Body2:\\
$\mathbf{p_i}$$\mathbf{(t_1)}$ = (300.0, 55.68);
$\mathbf{v_i}$$\mathbf{(t_1)}$ = (0.0, 21.00)\\
$\mathbf{p_i}$$\mathbf{(t_2)}$ = (300.0, 72.04);
$\mathbf{v_i}$$\mathbf{(t_2)}$ = (0.0, 41.66)\\
$\mathbf{p_i}$$\mathbf{(t_3)}$ = (300.0, 99.07);
$\mathbf{v_i}$$\mathbf{(t_3)}$ = (0.0, 62.33)\\
$\mathbf{p_i}$$\mathbf{(t_4)}$ = (300.0, 131.32);
$\mathbf{v_i}$$\mathbf{(t_4)}$ = (0.0, 80.33)\\

Body3:\\
$\mathbf{p_i}$$\mathbf{(t_1)}$ = (628.0, -6.82);
$\mathbf{v_i}$$\mathbf{(t_1)}$ = (104.99, -209.98)\\
$\mathbf{p_i}$$\mathbf{(t_2)}$ = (710.20.0, -170.40);
$\mathbf{v_i}$$\mathbf{(t_2)}$ = (208.31, -416.63)\\
$\mathbf{p_i}$$\mathbf{(t_3)}$ = (845.37, -440.74);
$\mathbf{v_i}$$\mathbf{(t_3)}$ = (311.64, -623.27)\\
$\mathbf{p_i}$$\mathbf{(t_4)}$ = (1006.58, -763.17);
$\mathbf{v_i}$$\mathbf{(t_4)}$ = (401.63, -803.25)\\

How test will be performed: Unit testing with pytest

\item{TC6: test\_system\_projectile\_45deg\\}

Description:``Projectile motion is a form of motion experienced by an object or particle (a projectile) that is thrown near the Earth's surface and moves along a curved path under the action of gravity only''. The rigid object is falling from a height specified and angle in the input. In the horizontal direction, velocity is constant.

Control: Automatic

Initial State: NA

Input:\\
$\mathbf{p_i}$$\mathbf{(t_0)}$ = (0.0, 0.0); $\mathbf{v_i}$$\mathbf{(t_0)}$ = (848.50, 0); $\mathbf{m}$ = 10; $\mathbf{g}$ =(0, -100)(acceleration due to gravity) \\

Output:\\
$\mathbf{p_i}$$\mathbf{(t_1)}$ = (14.14, 14.14);
$\mathbf{v_i}$$\mathbf{(t_1)}$ = (848.52, 846.90)\\
$\mathbf{p_i}$$\mathbf{(t_2)}$ = (452.50, 424.09);
$\mathbf{v_i}$$\mathbf{(t_2)}$ = (848.53, 743.54)\\
$\mathbf{p_i}$$\mathbf{(t_3)}$ = (890.86, 780.67);
$\mathbf{v_i}$$\mathbf{(t_3)}$ = (848.53, 640.22)\\
$\mathbf{p_i}$$\mathbf{(t_4)}$ = (1329.23, 1083.86);
$\mathbf{v_i}$$\mathbf{(t_4)}$ = (848.52, 536.89)\\ 
$\mathbf{p_i}$$\mathbf{(t_5)}$ = (1767.59, 1333.68);
$\mathbf{v_i}$$\mathbf{(t_5)}$ = (848.52, 433.57)\\ 
$\mathbf{p_i}$$\mathbf{(t_6)}$ = (2205.95, 1530.11);
$\mathbf{v_i}$$\mathbf{(t_6)}$ = (848.52, 330.25)\\ 
$\mathbf{p_i}$$\mathbf{(t_7)}$ = (2644.31, 1673.1758);
$\mathbf{v_i}$$\mathbf{(t_7)}$ = (848.52, 226.92)\\ 
$\mathbf{p_i}$$\mathbf{(t_8)}$ = (3082.67, 1762.86);
$\mathbf{v_i}$$\mathbf{(t_8)}$ = (848.52, 123.60)\\
$\mathbf{p_i}$$\mathbf{(t_9)}$ = (3521.04, 1799.16);
$\mathbf{v_i}$$\mathbf{(t_9)}$ = (848.52, 20.28)\\ 
$\mathbf{p_i}$$\mathbf{(t_10)}$ = (3959.40, 1782.09);
$\mathbf{v_i}$$\mathbf{(t_10)}$ = (848.52, -83.05)\\   

How test will be performed: Unit testing with pytest\\
Ref. source:  \url{https://en.wikipedia.org/wiki/Projectile_motion},
\url{https://www.amesweb.info/Physics/Projectile-Motion-Calculator.aspx}

%\end{enumerate}

\wss{Same comment as for previous test.  I believe all of the equations you will
  need for your closed form solutions are at:
  \url{https://en.wikipedia.org/wiki/Projectile_motion}}

\wss{You are using $x$ and $y$ in the conventional orientation, but you should
  also have tests where your projectile is thrown in the opposite direction and
  when your projectile has a non-zero velocity in the $y$ direction.  You want
  to make sure that there isn't an error in any of the coordinate directions.
  If everything is down and to the left you won't notice errors with motion up
  or down.}
\olu{updated testcase}

%\begin{enumerate}
\item{TC7: test\_system\_projectile\_60deg\\}

Description: See T6 above(launched at $60\,^{\circ}$ angle)

Control: Automatic

Initial State: NA

Input:\\
$\mathbf{p_i}$$\mathbf{(t_0)}$ = (50.0, 50.0); $\mathbf{v_i}$$\mathbf{(t_0)}$ = (150.0, 260.0); $\mathbf{m}$ = 1000; $\mathbf{g}$ =(0, -40) (acceleration due to gravity) \\

Output:\\
$\mathbf{p_i}$$\mathbf{(t_1)}$ = (52.50, 54.32);
$\mathbf{v_i}$$\mathbf{(t_1)}$ = (150.0, 259.141)\\
$\mathbf{p_i}$$\mathbf{(t_2)}$ = (130.0, 177.19);
$\mathbf{v_i}$$\mathbf{(t_2)}$ = (150.0, 217.81)\\
$\mathbf{p_i}$$\mathbf{(t_3)}$ = (207.48, 278.69);
$\mathbf{v_i}$$\mathbf{(t_3)}$ = (150.0, 176.48)\\
$\mathbf{p_i}$$\mathbf{(t_4)}$ = (284.97, 358.84);
$\mathbf{v_i}$$\mathbf{(t_4)}$ = (150.0, 135.15)\\ 
$\mathbf{p_i}$$\mathbf{(t_5)}$ = (362.47, 417.65);
$\mathbf{v_i}$$\mathbf{(t_5)}$ = (150.0, 93.82)\\ 
$\mathbf{p_i}$$\mathbf{(t_6)}$ = (439.96, 455.10);
$\mathbf{v_i}$$\mathbf{(t_6)}$ = (150.0, 52.49)\\  

How test will be performed: Unit testing with pytest\\
Ref. source:  \url{https://en.wikipedia.org/wiki/Projectile_motion},
\url{https://www.amesweb.info/Physics/Projectile-Motion-Calculator.aspx}
\end{enumerate}

\subsubsection{Rotation of 2D Rigid Body}
This test is to simulate the rotation of a 2D rigid body about its axis by applying torque.
\paragraph{}
\begin{enumerate}
\item{TC8: test\_system\_rotation\_around\_axis()\\}
	
	Description: In rotational motion of 2D rigid bodies, Torque $\tau$ is the force which produces rotation. It has magnitude and direction.This test can also be used for multiple set of rigid bodies.As constant torque is applied over time the angular orientation increases.
	
	Control: Automatic
	
	Initial State: NA
	
	Input:\\
    \hspace*{1.3cm}$\mathbf{m}$$\mathbf{}$ = 0.5\\
	\hspace*{1.3cm}$\mathbf{g}$$\mathbf{}$ = (0, 0)\\
	\hspace*{1.3cm}$\phi$$_i\mathbf{(t_0)}$ = 0\\
	\hspace*{1.3cm}$\tau$$\mathbf{}$ = 100 \\
	
	Output:\\
	$\phi$$\mathbf{(t_1)}$ = 0.0\\
	$\phi$$\mathbf{(t_2)}$ = 6.82\\
	$\phi$$\mathbf{(t_3)}$ = 26.45\\
	$\phi$$\mathbf{(t_4)}$ = 58.89\\
	$\phi$$\mathbf{(t_4)}$ = 97.58\\

	
	How test will be performed: Unit testing with Pytest\\
Ref source:\url{http://hyperphysics.phy-astr.gsu.edu/hbase/rotq.html} 

\item{TC9: test\_system\_angular\_velocity\\}

Description: This is to calculate the angular velocity $\omega$ of a rotating body over time when torque $\tau$ is applied.

Control: Automatic

Initial State: NA

Input:\\
\hspace*{1.3cm}$\mathbf{m}$$\mathbf{}$ = 0.5\\
\hspace*{1.3cm}$\mathbf{g}$$\mathbf{}$ = (0, 0)\\
\hspace*{1.3cm}$\phi$$_i\mathbf{(t_0)}$ = 0\\
\hspace*{1.3cm}$\omega$$_i\mathbf{(t_0)}$ = 0\\
\hspace*{1.3cm}$\tau$$\mathbf{}$ = 100 \\

Output:\\
$\omega$$\mathbf{(t_1)}$ = 0.40\\
$\omega$$\mathbf{(t_2)}$ = 12.80\\
$\omega$$\mathbf{(t_3)}$ = 25.20\\
$\omega$$\mathbf{(t_4)}$ = 37.60\\
$\omega$$\mathbf{(t_5)}$ = 48.40\\


How test will be performed: Unit testing with Pytest\\
Ref source: \url{http://hyperphysics.phy-astr.gsu.edu/hbase/rotq.html}
\end{enumerate}

\wss{Nice to see rotation tests not being forgotten.  As before though, I would
  like to know how you come up with your output  answers.}

\subsubsection{Collision of 2D Rigid Body Simulation}
This test is to simulate the collision of 2D rigid bodies.
\paragraph{}
\begin{enumerate}
	
	\item{TC10: test\_system\_collision\_same\_mass\\}
	
	Description: This is to test the collision of a set of rigid bodies that have the same weight.\\
	
	Control: Automatic
	
	Initial State: NA
	
	Input:\\
	Body1: $\mathbf{p_i}$$\mathbf{(t_0)}$ = (0.0, 50.0); $\mathbf{v_i}$$\mathbf{(t_0)}$ = 0; $\mathbf{m}$ = 1; $\mathbf{F}$ =(500, 0) \\ 
	Body2: $\mathbf{p_i}$$\mathbf{(t_0)}$ = (300.0, 50.0); $\mathbf{v_i}$$\mathbf{(t_0)}$ = 0; $\mathbf{m}$ = 100; $\mathbf{F}$ =(-500, 0) \\ 
	$\mathbf{C_R}$$\mathbf{}$ = 1 (this applies to both bodies) \\
	
	
	Output: \\
	Body1:\\
	$\mathbf{p_i}$$\mathbf{(t_1)}$ = (1.42, 50.0);
	$\mathbf{v_i}$$\mathbf{(t_1)}$ = (5.25, 0.0)\\
	$\mathbf{p_i}$$\mathbf{(t_2)}$ = (5.51, 50.0);
	$\mathbf{v_i}$$\mathbf{(t_2)}$ = (10.42, 0.0)\\
	$\mathbf{p_i}$$\mathbf{(t_3)}$ = (12.27, 50.0);
	$\mathbf{v_i}$$\mathbf{(t_3)}$ = (15.58, 0.0)\\
	$\mathbf{p_i}$$\mathbf{(t_4)}$ = (20.32, 50.0);
	$\mathbf{v_i}$$\mathbf{(t_4)}$ = (20.08, 0.0)\\
	
	Body2:\\
	$\mathbf{p_i}$$\mathbf{(t_1)}$ = (298.58, 50.0);
	$\mathbf{v_i}$$\mathbf{(t_1)}$ = (-5.25, 0.0)\\
	$\mathbf{p_i}$$\mathbf{(t_2)}$ = (294.49, 50.0);
	$\mathbf{v_i}$$\mathbf{(t_2)}$ = (-10.42, 0.0)\\
	$\mathbf{p_i}$$\mathbf{(t_3)}$ = (287.73, 50.0);
	$\mathbf{v_i}$$\mathbf{(t_3)}$ = (-15.58, 0.0)\\
	$\mathbf{p_i}$$\mathbf{(t_4)}$ = (279.67, 50.0);
	$\mathbf{v_i}$$\mathbf{(t_4)}$ = (-20.08, 0.0)\\
	 
	How test will be performed: Unit testing with Pytest
	Ref. source: collision.py
	
	\item{TC11: test\_system\_col\_impulse\_against\_static\_object\\}

Description: This is to test a set of rigid bodies that collide. This test case will test for collision of a set of dynamic object falling from a height with a static object. At collision bodies' velocity, position, angular velocity, orientation is calculated. Momentum is conserved. Multiple objects will be simulated to fall at different times fro a height so we can simulate collision.

Control: Automatic

Initial State: NA

Input:\\
Static Body1: $\mathbf{p_i}$$\mathbf{(t_0)}$ = (400, 560.0); $\mathbf{v_i}$$\mathbf{(t_0)}$ = 0; $\mathbf{m}$ = 10000 \\ 
Dynamic Body2: $\mathbf{p_i}$$\mathbf{(t_0)}$ = (20.0, 50.0); $\mathbf{v_i}$$\mathbf{(t_0)}$ = 0; $\mathbf{m}$ = 100 \\ 
$\mathbf{C_R}$$\mathbf{}$ = 1 (this applies to both bodies) \\


Output:  $\mathbf{v_k}$$\mathbf{(t)}$ = TBD \\
\hspace*{1.3cm}$\mathbf{p_k}$$\mathbf{(t)}$ = TBD\\
\hspace*{1.3cm}$\phi$$_k\mathbf{(t)}$ = TBD\\
\hspace*{1.3cm}$\omega$$_k\mathbf{(t)}$$ \mathbf{}$ = TBD\\
after t secs(t, TBD) 

How test will be performed: Unit testing with PyUnit

\end{enumerate}	
	
\wss{The TBDs should be filled in.  I get the impression that some of the
  physics is giving you trouble.  The following resource looks pretty good
  \url{https://www.myphysicslab.com/engine2D/collision-en.html}.  You could use
  this calculator
  \url{https://www.omnicalculator.com/physics/conservation-of-momentum} with
  your coefficient or restitution set to 1.0.  You just need the objects to not
  rotate after their collision.  Two spheres colliding should be fine.  You can
  also play around with problems where one mass is so large that that object
  will essentially be stationary.  I can also lend you a physics textbook, if
  that would be helpful.}

\subsection{Tests for Nonfunctional Requirements}

\subsubsection{Usability Test}
		
\paragraph{Usability test}

\begin{enumerate}

\item{TC7\\}

Type: Usability test
					
Initial State: 
					
Input/Condition: 
					
Output/Result: 
					
How test will be performed: Users/reviewers of TamiasMini2D will be asked to install the library and use it. They will be asked to complete the survey in the Appendix section for Usability. 
					
\end{enumerate}

\wss{Nice to see a usability test.  It is a bit simplistic, but that is fine for
  right now.}

\subsubsection{Correctness/Performance}

\paragraph{Correctness/Performance}

\begin{enumerate}

\item{TC8\\}

Type: Dynamic

Initial State: 

Input: 

Output: 

How test will be performed: Correctness/performance will be measured by comparing output results with ODEs related to each requirement/function.
\end{enumerate}

\wss{This isn't really complete.  For correctness, you can probably refer to
  your functional tests from the previous section.}

\subsubsection{Reusability}

\paragraph{Reusability test}

\begin{enumerate}
	
\item{TC9\\}

Type: Dynamic

Initial State: 

Input: 

Output: 

How test will be performed: Users/reviewers will be asked to see if they can extend the library and use for other purposes.
...
\end{enumerate}

\wss{A survey is an interesting way to measure this.}

\subsubsection{Understandability/Maintainability}

\paragraph{Understandability/Maintainability test}

\begin{enumerate}
	
	\item{TC9\\}
	
	Type: Dynamic
	
	Initial State: 
	
	Input: 
	
	Output: 
	
	How test will be performed: Users/reviewers will be asked to see check some see if they are able to find the space module and update the body parameters are desired. Users will be asked on the scale of 1 to 5 how easy it was to find the code, understand it and make changes.
	...
\end{enumerate}


\subsection{Traceability Between Test Cases and Requirements}
The purpose of the information in Table 1 below is to provide a mapping between the test cases and the requirements in the SRS for easy reference and verification.
%\wss{Provide a table that shows which test cases are supporting which %requirements.}
\begin{table}
	
	\caption{Requirements Traceability Matrix}
	
	\label{Table:Table_Traceability}  
	
	\begin{tabular}{|c|p{5cm}|p{5cm}|}
		
		\hline	
		
		\textbf{Testcase Number} & \textbf{Instance Models} & \textbf{CA Requirements}\\
		
		\hline 
		
		TC1& IM1         & R1, R2, R4, R5       \\ \hline
		
		TC2& IM1        & R1, R2, R4, R5       \\ \hline
		
		TC3& IM1        & R1, R2, R4, R5       \\ \hline
		
		
		
		TC4& IM2 & R1, R2, R4, R6   \\ \hline
		
		TC5& IM3 & R1, R3, R4, R7, R8   \\ \hline
		
		
		
		TC6& IM3 & R1, R3, R4, R7, R8   \\ \hline
		
		TC7&     & NFR3   \\ \hline
		
		
		
		TC8&       & NFR1, NFR2  \\ \hline
		
		TC9&       & NFR5   \\ \hline
		
		TC10&       & NFR4, NFR6   \\ \hline
		
		
		
		
	\end{tabular}\\
	
\end{table}

\section{Static Verification Techniques}
Code review and inspection will be used as the method for implementation verification.
%\wss{In this section give the details of any plans for static verification of
%the implementation.  Potential techniques include code walkthroughs, code
%inspection, static analyzers, etc.}
\wss{You can remove this section, since the details can be covered in Section
  4.4.  I've realized that I should remove this section from the template.}

\wss{You should also think about parallel testing.  Calculate the answers for
  some more complex simulations with existing software and compare these results
  to your results.}

\bibliographystyle{plainnat}

\bibliography {../../../refs/References}
\begin{itemize}
\item{http://www.physicstutorials.org/home/mechanics/1d-kinematics/projectile-motion?start=1}
\item{https://en.wikipedia.org/wiki/Projectile-motion}
\item{R. A. BROUCKE.  "Equations of motion of a rotating rigid body", Journal of Guidance, Control, and Dynamics, Vol. 13, No. 6 (1990), pp. 1150-1152.}
\item{https://github.com/smiths/caseStudies/blob/master/CaseStudies/gamephys/docs/SRS/GamePhysicsSRS.pdf}

\end{itemize}


\newpage

\section{Appendix}

This is where you can place additional information.

\subsection{Symbolic Parameters}

The definition of the test cases will call for SYMBOLIC\_CONSTANTS.
Their values are defined in this section for easy maintenance.

\subsection{Usability Survey Questions?}

%\wss{This is a section that would be appropriate for some projects.}
\begin{enumerate}

\item {On the scale of 1 - 5, 1 being very difficult and 5 being very easy, How easy was it to install the program using the installation guide?

Comment on what can be improved:}

\item On a scale of 1 - 5, how easy were you able to update the parameters in a space? e.g change the velocity of a body

\item Did the program return the expected output based on the testcase and input values? 

If no, please add comments explaining issues encountered:


\end{enumerate}

\end{document}